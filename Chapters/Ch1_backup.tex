\chapter{Introduction}
\label{ch1}


% Part 1   Background and Vision of Swarm robotics
% Part 2   Limitations
% Part 3   Contributiion of this Thesis

\section{Background and Vision of Swarm robotics}

% Part 1
%  P1 What is swarm robotics
%  P2 scalability and fault tolerance
%  P3 Sensing, Decision-Making, and Acting as a Collective
%  P4 Programmability and Design Flexibility
%  P5 Human–Swarm Interaction
%  P6 Applications of Swarm Robotics
%  P7 Summary


%  P1 What is swarm robotics
%     s1 inspired
%     s2 individual are simple
%     s3 but together complex
%     s4 Swarm robotics learn from that --> to robots
%     s5 it is promising
Swarm Robotics draws inspiration from the collective behaviors of social animals in nature, such as colonies of ants, flocks of birds, schools of fish \textcolor{red}{cite}.
In these systems, individuals have limited capabilities.
However, through simple local interaction rules, they exhibit complex global behaviors and accomplish tasks that are difficult for single individual to achieve.
Swarm Robotics, trying to learn from this mechanism, connects multiple autonomous robots through local communication and collaboration to form a robot system with overall intelligence.
In recent years, swarm robotics has shown unique advantages in scenarios such as unknown environment exploration and large-scale collaborative operations.

%  P2 scalability and fault tolerance
One of the core advantages of a swarm robotic system is scalability.
Key feature of a swarm is scalability, a swarm can increase without cost
The other is fault tolerance, a swarm operates with failures.

%  P3 Sensing, Decision-Making, and Acting as a Collective
In the meantime, a swarm should operate like normal single robot systems.
It should sense the envrionment, making decisions, and react.

%  P4 Programmability and Design Flexibility
It should also easy to program and design

%  P5 Human–Swarm Interaction
Human operator should interact with the swarm easily.

%  P6 Applications of Swarm Robotics
With such feature, swarm systems can operation in many applications, like search and rescue, xxx

%  P7 Summary
In summary, Swarm Robotics is powerful.


\section{Limiations}

% Part 2
%    P1 however, can't achieve all of above
%    P2 problem of Sensing, Decision-Making, and Acting as a Collective
%    P3 problem of Programmability and Design Flexibility
%    P4 problem of Human–Swarm Interaction
%    P5 a centralized system is easy, but not scalable or fault tolerance
%    P6 summary

% P1 however, can't achieve all of above
However, current swarm systems faces great challenges.
% P2 problem of Sensing, Decision-Making, and Acting as a Collective
Because no global information, a swarm can not sense the envrionment as a whole, make decision, and coordinate what to do
% P3 problem of Programmability and Design Flexibility
Because of Micro-Macro problem, one cannot design and program the swarm easily.
% P4 problem of Human–Swarm Interaction
Human operator can't communicate to everyone and interact with the swarm
% P5 a centralized system is easy, but not scalable or fault tolerance
A centralized system is easy to do those, but they are not scalable or fault tolerance.
% P6 Summary
In summary, swarm is facing challenges.

\section{Goals and Contributions of This Thesis}

This paper aims to push one step further.
This thesis presents SoNS
%2
Chapter 2 introduces the related work
%3
Chapter 3 presents the design and establishment of the SoNS framework, focusing on dynamic topology formation and distributed sensing.
%4
Chapter 4 details the implementation and evaluation of SoNS in a flying drone swarm, highlighting its capabilities in formation control and navigation.
%5
Chapter 5 demonstrates robot autonomy (code transfer and mission adaptation) and human–swarm interaction.
%6
Chapter 6 provides discussions, limitations, and directions for future research.

