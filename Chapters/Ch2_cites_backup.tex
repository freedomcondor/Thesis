   \textcolor{red}{still to read}
    \cite{arrigoni2018bearing}
    \cite{trinh2018bearing}
    \cite{trinh2021finite}

    \cite{tay1985generating} \textcolor{red}{too many, consider drop}
    \cite{eren2012formation} 
    \cite{trinh2019minimal}
    \cite{karimian2017theory}
    \cite{hou2016elementary}
    \cite{carboni2014rigidity} maybe not about bearing, only rigidity, \textcolor{red}{consider drop}

\subsection{coverage}:

\cite{wang2011coverage} survey : fixed sensor
\cite{galceran2013survey} survey : moving sensor , Coverage Path Planning(CPP)

sensor network
\cite{luo2018adaptive}
\cite{santos2019decentralized}
\cite{siligardi2019robust} talks about connectivity, coverage is only part of it

environmental monitoring, collective perception
\cite{schmickl2006collective} : collective perception
\cite{baxter2007multi} search and rescue
\cite{lima2017cellular} foraging

\cite{almadhoun2019survey} a survey of multi-robot covrage path planning
\cite{avellar2015multi}
\cite{julia2012comparison} : single robot? path planning coverage

area divided, each robot sweep one block
\cite{rekleitis2008efficient}
\cite{scherer2015autonomous}

offline, path optimization
\cite{nazarahari2019multi}
\cite{thabit2018multi}
\cite{yu2016optimal}

offline construct map
\cite{mirzaei2011cooperative}

online construct map
\cite{ge2005complete}
\cite{miki2018multi} abstract not talking about map, but about search and action \textcolor{red}{consider drop}

about map broadcast
\cite{marjovi2009multi}
\cite{albani2017field}

randomwalking coverage:
\cite{mcguire2019minimal}
\cite{kegeleirs2019random}
\cite{huang2019exploration}
\cite{ichikawa1999characteristics}
\cite{pang2021effect}
\cite{khaluf2018collective}
\cite{zia2017cognitive} cellular-automata for city coverage, maybe not randomwalking

pheromone based
\cite{koenig2001terrain}
\cite{schroeder2017efficient}
\cite{deshpande2017robot}
\cite{maftuleac2015local} maybe not about pheromone, only flying robots
\cite{stirling2010energy} not about pheromone, only flying robots

formation sweep :

centralized:
\cite{wang1991navigation} only formation, not about coverage
\cite{din2018behavior} 
\cite{campbell2012review} a survey, about boat obstacle avoidance

self-organized: 
MNS et al
\cite{jamshidpey2020multi}
\cite{jamshidpey2023reducing}
\cite{mathews2017mergeable}
\cite{zhang2023self}
\cite{zhu2020formation}
\cite{zhu2024self} \textcolor{red}{not correct places}

\subsection{others}:
\cite{howard2006multi} mapping SLAM
\cite{psaraftis2016dynamic} path optimization, about vehicle routing

\cite{tuci2018cooperative} transportation
\cite{robin2016multi} search and rescue


\subsection{complex network}

\textcolor{red}{complex network is a too big area} check if swarm robot enough, if yes, check if flat :: check the point of include

\textcolor{blue}{from first paragraph of Sinan's paper about fault tolerance, talking about connectivity of a swarm network}

\cite{kirst2016dynamic}
\cite{zavlanos2011graph}

if connectivity is not reliable, system convergence and performance guarantees can be compromised: why this is important to the Thesis
\cite{cortes2008distributed} consensus
\cite{de2006decentralized}
\cite{moreau2005stability}
\cite{olfati2007consensus} consensus

\subsection{fault tolerance}
\cite{bjerknes2013fault}
\cite{tarapore2017generic}
\cite{winfield2006safety}
\cite{strobel2018managing} \textcolor{red}{incomplete}
\cite{pini2011task}
\cite{o2023predictive}
\cite{oladiran2019fault}

fault detection
\cite{tarapore2019fault}
\cite{khaldi2017monitoring}

exogenous fault detection
\cite{khadidos2015exogenous} nothing special with fault tolerance
\cite{millard2016exogenous} 
\cite{millard2013towards}

replacing/repairing the failed robots without pausing the mission
\cite{christensen2009fireflies} detection
\cite{varadharajan2020swarm} relay, maybe related to MNS, dynamic communication

\subsection{IFS, maybe not needed}
\cite{edition2000authoritative}
\cite{zhou2019review}
\cite{niu2021distributed}
\cite{sheng2021intermittent}
\cite{zhang2021intermittent}
\cite{syed2016novel}

\subsection{path planning}

UAV motion planning

\cite{quan2020survey} surveys recent state-of-the-art UAV motion planning methods, covering path finding and trajectory optimization techniques along with their motivations, formulations, and real-world applications.
single drone

\cite{zhou2022swarm} introduces a fully autonomous swarm of palm-sized flying robots equipped with an efficient onboard trajectory planner that enables real-time, collision-free navigation and coordination in cluttered natural environments like dense forests.

\cite{wang2022geometrically} proposes an optimization-based framework for multicopter trajectory planning that efficiently handles geometric and dynamic constraints through a novel trajectory representation and constraint elimination technique, achieving high-quality, real-time solutions.

\cite{wang2025unlocking} presents a system that enables quadcopters to autonomously generate and execute complex aerobatic maneuvers using a discrete maneuver representation, spatial-temporal trajectory optimization, and yaw compensation strategies in cluttered environments.

\cite{han2019fiesta} presents FIESTA, a fast incremental mapping system for building global Euclidean Signed Distance Fields (ESDFs) that enables real-time motion planning for aerial robots through efficient obstacle updates and high-performance map maintenance.
a bit detailed




%-------------------------------------------------------------
% P1 start summary :
% Although swarm researches developed over the past decades, researches focus on aspect.
% there lacks a systematic solution to cover overall applications in real scenario.

% P2 an ideal swarm system should be :
%   easy to deploy, change(reprogram)
%   use only local information
%   morphology
%   as a whole, sense the environment, make decision and react (autonomous).

% P3 however decentralized and centralized methods each can fulfill part of those.

% P4 This chapter gives a review of each of those


%subsection : Hybrid, hierarchical
% As centralized - decentralized each has its own pros and cons
% there are researches try to engage hybrid or hierarchical reserachs

%subsection : Task assignment in formation

%subsection : swarm autonomous

%subsection : reprogramming and human swarm interaction

%-----------------------------------------------------------


%Chapter 1 described some concrete problems and why they are important and difficult.
%In this chapter, we review and discuss literatures about them.

% a swarm should be
%As discussed in Ch1, a swarm should be
%    easy to deploy
%    use only local information
%    as a whole, sense the environment and react.

% however difficult
%However, with these constraints, many tasks are difficult to complete.

%\section{self-organization hierarchy}

%Many researches try to combine centralized and decentralized.
%They go hierarchical, with hierarchical, information can easily flow

%\section{Task assignment in formation}

%formation is researched in control theory.

%Task assignment makes efficient formation
%Task assignment calculated hungarian, network flow, n-flex algorithm

%distributed task assignment, each robot calculate a part of it, but calcualtion increases

%\section{swarm autonomous}

%what's in the builderbot paper.

%\section{reprogramming and human swarm interaction}

%what's in the builderbot paper.

%2.1 集群机器人的核心控制范式:从完全分布式到层级化混合架构

%2.1.1 纯分布式集群(Swarm)的理论基础与典型方法
%无领导集群的自组织机制(如基于局部规则的群体行为涌现)
%优势: scalability、容错性、适应动态环境的鲁棒性
%代表性研究:蚁群优化、鸟群模型(Boids)及其在机器人集群中的应用

%2.1.2 集中式控制(Centralized)的技术路径与适用场景
%全局信息感知与统一决策的实现方式(如基于中央服务器的路径规划)
%优势:任务精度高、行为可预测性强、易于编程与调试
%局限性:单点故障风险、扩展性瓶颈、对通信带宽依赖高

%2.1.3 混合架构与层级化领导模式(Hybrid, Hierarchical, Leadership)的演进
%动态层级的构建逻辑:临时领导者选举、角色分配与权限转移机制
%代表性方案:分布式领导(如基于局部优势的动态层级)与半集中式控制(如分区协调)
%核心价值:平衡分布式的灵活性与集中式的高效性

%2.2 任务场景导向的控制策略对比:集群与集中式方案的适用性分析

%2.2.1 队形控制(Formation Control)
%集群方案:基于局部距离 / 方位感知的自组织队形(如 Voronoi 图、人工势场法)
%优势:适应队形动态调整、个体故障不影响整体
%局限:精度依赖局部信息质量、大规模集群易出现累积误差
%集中式方案:全局路径规划与轨迹优化(如模型预测控制 MPC)
%优势:高精度队形保持、复杂图案生成能力强
%局限:计算负荷随规模指数增长、抗干扰能力弱

%2.2.2 探索与搜救任务(Exploration, Search and Rescue)
%集群方案:基于区域覆盖的分布式探索(如随机行走 + 信息共享)
%优势:无盲区覆盖、适应未知环境、多目标并行处理
%局限:任务协调效率低、全局信息整合滞后
%集中式方案:基于全局地图的任务分配(如 A * 算法 + 任务优先级排序)
%优势:任务规划最优性高、资源调度高效
%局限:依赖完整环境先验知识、难以应对突发障碍

%2.2.3 无人机娱乐表演(Drone Entertainment Shows)
%集群方案:基于预编程规则的分布式同步(如时间触发的轨迹对齐)
%优势:单无人机故障不中断整体表演、小规模部署灵活
%局限:复杂图案生成难度大、实时调整能力差
%集中式方案:全局动画分解与个体轨迹预生成
%优势:支持高精度复杂图案、时间同步性强
%局限:对通信延迟敏感、扩展性受限于中央计算能力

%2.2.4 其他典型任务的扩展分析
%物流协同运输:负载分配策略的集群 vs 集中式对比
%环境监测:数据采集与融合方式的效率差异

%2.3 集群编程与重编程的方法论:离线范式与核心挑战

%2.3.1 传统集群编程的离线模式
%行为规则预定义:基于个体算法的群体行为映射(如强化学习训练局部策略)
%仿真验证与参数调优:通过物理引擎模拟优化群体行为参数(如 Swarmulator 等工具)
%部署流程:统一固件更新与任务指令预装(无在线调整能力)

%2.3.2 离线编程的核心技术路径
%基于模板的行为组合:模块化规则库(如避障、聚集、跟随模块的组合调用)
%宏观行为到微观规则的转化方法:解决 “微 - 宏映射” 问题的数学建模(如控制论方法、群体动力学)

%2.3.3 离线范式的局限性隐含
%环境适应性差:预编程规则难以应对未预期场景
%扩展性瓶颈:大规模集群的参数调优成本呈指数增长

%2.4 集群自主性(Swarm Autonomy)的定义与评价维度

%2.4.1 自主性的层级划分
%基础自主性:个体故障自修复、简单环境适应(如避障)
%高级自主性:群体目标动态调整、跨任务自转换、人机协同决策

%2.4.2 现有方案的自主性边界
%纯分布式集群:高个体自主性但群体目标僵化
%集中式集群:群体目标可控但个体自主性缺失
%混合架构:在动态目标调整与自主决策方面的探索进展

%2.4.3 自主性与可控性的平衡难题
%高自主性带来的行为不可预测性风险
%人类干预与集群自主决策的接口设计挑战

%2.5 本章小结与研究定位

%现有研究的核心局限:混合架构的动态性不足、大规模任务的精度与效率难以兼顾、在线重编程能力缺失
%本研究(SoNS)的切入点:基于动态层级的混合架构如何突破上述局限,为集群自主性与可编程性提供新路径

%2.1 Core Control Paradigms in Swarm Robotics: From Fully Distributed to Hierarchical Hybrid Architectures
%2.1.1 Theoretical Foundations and Typical Methods of Purely Distributed Swarms
%Self-organization mechanisms in leaderless swarms (e.g., emergence of collective behavior based on local rules)
%Advantages: Scalability, fault tolerance, and robustness in adapting to dynamic environments
%Representative studies: Ant Colony Optimization, the Boids model, and their applications in robotic swarms​
%2.1.2 Technical Paths and Applicable Scenarios of Centralized Control​
%Implementation of global information perception and unified decision-making (e.g., path planning based on a central server)​
%Advantages: High task precision, strong behavior predictability, and ease of programming and debugging​
%Limitations: Single-point failure risk, scalability bottlenecks, and high dependence on communication bandwidth​
%2.1.3 Evolution of Hybrid Architectures and Hierarchical Leadership Models​
%Logic for constructing dynamic hierarchies: Mechanisms for temporary leader election, role assignment, and authority transfer​
%Representative solutions: Distributed leadership (e.g., dynamic hierarchies based on local superiority) and semi-centralized control (e.g., partitioned coordination)​
%Core value: Balancing the flexibility of distributed systems and the efficiency of centralized systems​
%2.2 Task-Scenario-Oriented Comparison of Control Strategies: Analysis of the Applicability of Swarm vs. Centralized Solutions​
%2.2.1 Formation Control​
%Swarm solutions: Self-organized formations based on local distance/orientation perception (e.g., Voronoi diagrams, artificial potential field method)​
%Advantages: Adaptable to dynamic formation adjustments; individual failures do not affect the whole system​
%Limitations: Precision depends on the quality of local information; cumulative errors tend to occur in large-scale swarms​
%Centralized solutions: Global path planning and trajectory optimization (e.g., Model Predictive Control, MPC)​
%Advantages: High-precision formation maintenance and strong capability for generating complex patterns​
%Limitations: Computational load grows exponentially with scale; weak anti-interference ability​
%2.2.2 Exploration and Search and Rescue Missions​
%Swarm solutions: Distributed exploration based on area coverage (e.g., random walk + information sharing)​
%Advantages: Blind-spot-free coverage, adaptability to unknown environments, and parallel processing of multiple targets​
%Limitations: Low task coordination efficiency; lag in global information integration​
%Centralized solutions: Task assignment based on global maps (e.g., A* algorithm + task priority ranking)​
%Advantages: High optimality of task planning and efficient resource scheduling​
%Limitations: Dependence on complete prior environmental knowledge; difficulty in responding to unexpected obstacles​
%2.2.3 Drone Entertainment Shows​
%Swarm solutions: Distributed synchronization based on preprogrammed rules (e.g., time-triggered trajectory alignment)​
%Advantages: Single-drone failures do not interrupt the overall performance; flexible small-scale deployment​
%Limitations: High difficulty in generating complex patterns; poor real-time adjustment capability​
%Centralized solutions: Global animation decomposition and pre-generation of individual trajectories​
%Advantages: Support for high-precision complex patterns and strong temporal synchronization​
%Limitations: Sensitivity to communication latency; scalability limited by central computing capacity​
%2.2.4 Extended Analysis of Other Typical Tasks​
%Collaborative logistics transportation: Comparison of swarm vs. centralized solutions in load distribution strategies​
%Environmental monitoring: Efficiency differences in data collection and fusion methods​
%2.3 Methodologies for Swarm Programming and Reprogramming: Offline Paradigms and Core Challenges​
%2.3.1 Offline Modes of Traditional Swarm Programming​
%Predefinition of behavioral rules: Mapping of collective behavior based on individual algorithms (e.g., reinforcement learning for training local policies)​
%Simulation verification and parameter tuning: Optimization of collective behavior parameters via physics engine simulations (e.g., tools like Swarmulator)​
%Deployment process: Unified firmware updates and pre-installation of task instructions (no online adjustment capability)​
%2.3.2 Core Technical Paths of Offline Programming​
%Template-based behavior composition: Modular rule libraries (e.g., combined invocation of obstacle avoidance, aggregation, and following modules)​
%Methods for converting macro-level behavior to micro-level rules: Mathematical modeling to address the "micro-macro mapping" problem (e.g., control theory methods, collective dynamics)​
%2.3.3 Implied Limitations of the Offline Paradigm​
%Poor environmental adaptability: Preprogrammed rules struggle to handle unanticipated scenarios​
%Scalability bottlenecks: Parameter tuning costs for large-scale swarms grow exponentially​
%2.4 Definition and Evaluation Dimensions of Swarm Autonomy​
%2.4.1 Hierarchical Classification of Autonomy​
%Basic autonomy: Individual fault self-repair, simple environmental adaptation (e.g., obstacle avoidance)​
%Advanced autonomy: Dynamic adjustment of collective goals, cross-task self-transformation, human-swarm collaborative decision-making​
%2.4.2 Autonomy Boundaries of Existing Solutions​
%Purely distributed swarms: High individual autonomy but rigid collective goals​
%Centralized swarms: Controllable collective goals but lack of individual autonomy​
%Hybrid architectures: Ongoing research progress in dynamic goal adjustment and autonomous decision-making​
%2.4.3 Challenges in Balancing Autonomy and Controllability​
%Risk of unpredictable behavior caused by high autonomy​
%Challenges in designing interfaces for human intervention and swarm autonomous decision-making​
%2.5 Chapter Summary and Research Positioning​
%Core limitations of existing research: Insufficient dynamics of hybrid architectures, difficulty in balancing precision and efficiency for large-scale tasks, and lack of online reprogramming capabilities​
%Entry point of this research (SoNS): How hybrid architectures based on dynamic hierarchies can overcome the above limitations and provide a new path for swarm autonomy and programmability
