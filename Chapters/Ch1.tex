\chapter{Introduction}
\label{ch1}

% logic:
% swarm robotics in general

%     swarm is good, 
%     scalability, fault tolerance, ...

%     however, downsides
%          micro macro problem
%          global information, analyzation
%          
%     centralized is easy but not scalable ...

%     To combine these two, hybrid approaches ...

%     concrete problem is:
%     
%     self-organization hierarchy
%     Task assignment with local information
%     swarm behavior design and adaption
%     human swarm interaction.

%     in this paper I solve those problems.
%     1. method to arrange hierarchy local information.
%     2. based on hierachy, I make efficitent task assignment with local information
%     3. easy to design, DBT for learning and HSI


% P1. Swarm is good
% P2. but swarm is hard to design, hard to control, not autonomous
% P3. centralized is easy, but not scalable, not fault tolerant

% P4. In this paper, we bring centralize to self-organization.
%     but to do that, face some concrete problems

% P5. self-organization hierarchy
% P6. Task assignment in formation with local information
% P7. swarm behavior design and autonomous
% P8. Human swarm interaction.

% P9. In this thesis, I create SoNS, which solves:   
%   avoid I, and avoid we in Ch1 2 6
%   This thesis proposes .... 
%      1. hierarchy establishment,
%      2. Task assignment formation
%      3. Behavior tree management, and DBT to autonomous
%      4. Human swarm interaction

%break P9 to multiple paragraphs / subsection / bullet list
%section 1.1
% say what contribution of each paper

%--------------------------------------------------------------------

% P1 swarm is good
%     s1. swarm intelligence inspired by ..
%     s2. in these animals, local interaction
%     s3. swarm robotics follow these
%     s4. It has many applications
%1
Swarm intelligence is inspired by the collective behaviors and collaborative strategies observed in various biological systems such as flocks of birds, schools of fish, or colonies of ants.
%2
These natural systems exhibit intelligent, adaptive, and self-organized patterns of cooperation without centralized control, where global behavior emerges from local interactions among individuals \textcolor{red}{[cite]}.
%\cite{bonabeau1999swarm, swarmintelligence},
%3
Swarm robotics follows this principle of local interaction.
%4
This decentralized approach enables a group of robots to exhibit scalability, fault tolerance, and adaptability as seen in the biological systems mentioned above \textcolor{red}{[cite]},
%\cite{dorigo2021swarm, brambilla2013swarm, rubenstein2014programmable}.
These characteristics are highly suitable for tasks such as
environmental monitoring \textcolor{red}{cite},
navigation and transportation \textcolor{red}{cite},
self-assembly \textcolor{red}{cite},
construction \textcolor{red}{cite},
bio-hybrid interaction \textcolor{red}{cite},
and etc.

% P2 But
%     s1. despite, it has problems
%     s2. macro micro problem, hard to design
%     s3. trial and error, pre-fixed, so unknow situation --> not autonomous
%     s4. another aspect of autonmous: due to local info,  sensing, analyzing, actuation,
%     s5. due to local info, human swarm interaction.

%1
Despite the promising capabilities of swarm robotics, several challenges hinder its broader deployment in complex, real-world scenarios.
%2
One of the primary difficulties lies in the so-called “macro–micro” gap: it is often nontrivial to design low-level agent behaviors that can reliably lead to the desired global swarm behavior \textcolor{red}{[cite]}.
%3
Designing such systems typically relies on extensive trial-and-error tuning or pre-defined behavior patterns, which may lack the flexibility to handle unforeseen environmental changes or mission objectives—thus limiting the autonomy of the swarm \textcolor{red}{[cite]}.
%4
Moreover, the autonomy of individual robots is constrained by their reliance on local sensing and information, which complicates tasks that require global situational awareness or complex coordination strategies \textcolor{red}{[cite]}.
%5
This reliance on local information also poses challenges for effective human–swarm interaction, as operators usually need global communication to command the whole swarm in real time \textcolor{red}{[cite]}.

% P3, on the other hand, centralized is good at those
%     s1. On the other hand, centralized is good at those
%     s2. as a whole system, close loop of sensing, analyzing, actuation
%     s3. human - interaction is there due to full connectivity
%     s4. but it lose scalability, fault tolerance.
%1
On the other hand, centralized robotic systems excel in areas where swarm robotics typically struggle.
%2
By treating the robot team as a single integrated system, centralized approaches enable a closed-loop cycle of global sensing, centralized data analysis, and coordinated actuation \textcolor{red}{[cite]}.
%3
Furthermore, centralized systems often maintain full network connectivity, allowing for more intuitive and direct human–robot interaction and supervision \textcolor{red}{[cite]}.
%4
However, these advantages come at the cost of scalability and fault tolerance, as the system may become a single point of failure and struggle to adapt in distributed or dynamic environments \textcolor{red}{[cite]}.

% P4. we try to hybrid
%     s1. works try to hybrid these.
%     s2. xxx did xxx
%     s3. xxx did xxx
%     s4. but these attempts face these problems:
%1
To overcome the limitations of purely centralized or decentralized systems, recent research has explored hybrid approaches that aim to combine the strengths of both paradigms.
%2
For instance, \textcolor{red}{Author A et al. proposed a hybrid control architecture... }
%3
Similarly, \textcolor{red}{Author B et al. introduced a another hierarchical system...}
%4
However, although these hybrid strategies to some extend solve some problem, but they still face several concrete challenges.

% P5. Self-organized hierachy
%     s1. heirachy is good, but challenge
%     s2. some literature in leader follower..
%     s3. existing all fixed.
%     s4. with a local info system, it is hard to manage the hierachy.
%1
First, hierarchy is a primary strategy to combine centralized or decentralized methods, but establishing and maintaining such hierarchies in swarm systems presents significant challenges.
%2
Prior work in leader–follower models \textcolor{red}{[cite]} and role-based agent assignment \textcolor{red}{[cite]} has shown some success in managing group behavior through structured organization.
%3
However, most of these systems rely on pre-defined, static hierarchies, where leaders or roles are designated in advance and remain fixed throughout the operation.
%4
In swarm systems that depend primarily on local information and decentralized decision-making, it remains difficult to dynamically form and adjust hierarchies in a self-organized and robust manner \textcolor{red}{[cite]}.

% P6. formation, task assignment
%     s1. formation is basic
%     s2. efficient formation relies on task assignment
%     s3. but good task assignment needs global
%     s4. xxx and xxx they go circles, not efficient
%     s5. xxx and xxx, they districtuted, but not scalable
%     s6. remains a problem,
%1
Second, formation control is one of the most fundamental and widely studied problems in swarm robotics.
%2
Effective swarm formation maintenance often relies on appropriate task assignment mechanisms, where agents are allocated roles or positions so that the travel cost for the swarm is small.
%3
However, achieving optimal or even efficient task assignment typically requires access to global information, which is inconsistent with the constraints of local sensing and decentralized control.
%4
Some decentralized approaches \textcolor{red}{[cite]} attempt to achieve the formation regardless of task assignment, where each robot move very far away until randomly finds a positon.
%5
Other approaches \textcolor{red}{[cite]} attempt to solve formation and task allocation in a distributed way, but the computation cost for each robot still grows with the swarm size, which violates the principle of scalability.
%5
As a result, robust and scalable task assignment for distributed formation control remains an open challenge in swarm robotics.

% P7. swarm behavior design and autonomous
%     s1. fixed program, 
%     s2. Swarm needs to make decisions on its own
%1
Third, as mentioned above, swarm is not autonomy.
%2
With pre-fixed programs, swarm can't deal with unexpected situations.
\textcolor{red}{xxxx}

% P8. Human swarm interaction.
Last, it is difficult for human to steer or re-program a swarm on the fly.
\textcolor{red}{xxxx}


% P10. In this paper, I propose SoNS
%     s1. In this these, I propse SoNS, which brings centralization to SO
%     s2. In SoNS, robot follow strictly local interaction.
%     s3. but robots coordinate with neighbors and form a hierachical topology, information flows
%     s4. harnessing from this, I show efficient task assignment for formation
%     s5. I show easy design for robot autonomous
%     s6. I show easy way for human swarm interaction
%1
This thesis proposes a novel framework named Self-organizing Nervous Swarm (SoNS), which aims to bring the benefits of centralized coordination into a fully decentralized, self-organizing swarm system.
%2
In SoNS, each robot follows strictly local interaction rules based on nearby sensing and communication, maintaining the principles of swarm intelligence.
%3
In the meantime, through continuous coordination with neighboring agents, the swarm spontaneously forms a hierarchical communication topology, allowing information to flow across the topology of the hierarchy in a guided way.
%4
Harnessing this emergent structure, efficient task assignment can be demonstrated for formation control can be achieved without requiring global knowledge.
%5
Furthermore, the SoNS framework is shown to simplify the design of autonomous behaviors, allowing robots to transfer codes to each other and re-program themselves to adapt to dynamic tasks and environments.
%6
Lastly, this thesis shows that SoNS facilitates scalable human–swarm interaction, where a human operator steer and reprogram the whole swarm by communicating with only one robot.




\textcolor{blue}{bio-hybrid, doesn't belong here, maybe move to Ch1}
\textcolor{red}{There are also researches on bio-hybrid research. Researches tries to understand the self-organizing from nature
    \cite{wahby2018autonomously}
    \cite{halloy2007social}
    \cite{buhl2006disorder}
    \cite{detrain2008collective}
    \cite{theraulaz1998origin}
}.
