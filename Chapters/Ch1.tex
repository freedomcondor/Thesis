\chapter{Introduction}
\label{ch1}

\textcolor{blue}{autonomous robots}

Autonomous robots are expected to perceive their environment, make decisions independently, and operate without continuous human control.
In such systems, humans primarily take a supervisory role, intervening only when necessary to adjust high-level objectives or ensure safety.

\textcolor{blue}{swarm robotics}

Swarm robotics is a decentralized robotic paradigm inspired by collective behaviors observed in natural systems such as insect colonies and animal groups
    \cite{bonabeau1999swarm}.
In these systems, complex global behaviors emerge from simple local interactions, without relying on centralized control or global information.
By following the same principle, swarm robotic systems naturally exhibit desirable properties including scalability, fault tolerance, and adaptability
    \cite{dorigo2014swarm,
          dorigo2020reflections,
          dorigo2021swarm}.
As the swarm size increases, performance can be maintained without centralized bottlenecks, while the failure of individual robots does not critically affect the overall system behavior.
These characteristics make swarm robotics particularly attractive for large-scale, distributed, and dynamic real-world applications, such as
    environmental monitoring
    \cite{talamali2021less},
    navigation and transportation
    \cite{dorigo2013swarmanoid},
    self-assembly
    \cite{rubenstein2014programmable},
    and collective construction
    \cite{team2012designing, petersen2019review}.

\textcolor{blue}{However}

However, swarm robotic systems typically rely on local sensing and local communication, with each robot having access to only partial information about the global state.
As a result, reaching consistent swarm-level decisions can be slow or inefficient.
The lack of centralized control mechanism also complicates human swarm interaction.
These limitations restrict the level of autonomy and manageability that swarm systems can achieve in complex and dynamic tasks.

\textcolor{blue}{hierarchy}

Hierarchical organization provides a promising approach to addressing the paradox between centralized and decentralized multi-robot systems.
By introducing structured information flow, hierarchical approaches enable limited forms of global coordination while preserving local interactions among robots.
Such structures can improve collective decision making, support scalable human supervision, and facilitate coordinated behavior at the swarm level.
Importantly, when designed appropriately, hierarchical methods can retain the scalability and robustness that characterize swarm robotics.

Existing work has already proved that hierarchical structures can be used to improve swarm robotics.
MNS tries to form a centralized control structure in a decentralized way.
    \cite{mathews2017mergeable}
Later work extend this idea to formation and coverage.
    \cite{zhang2023self}
    \cite{jamshidpey2020multi}
    \cite{jamshidpey2024centralization}
    \cite{jamshidpey2023reducing}
    \cite{ouguz2025proactive}
    \textcolor{red}{my paper ?}
    \cite{zhu2020formation}
    \cite{zhu2024self}

\textcolor{blue}{This thesis}

This thesis proposes a self-organizing hierarchical framework for swarm robotics.
The objective is to unify autonomous decision making, scalable coordination, and human–swarm interaction under strict local-information constraints.
The proposed framework demonstrates how hierarchical structures can emerge and be maintained without centralized control.
Through this approach, the thesis shows how hierarchy can be leveraged as an enabling mechanism rather than a violation of swarm principles.
