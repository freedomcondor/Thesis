\chapter{Conclusions and Discussions}
\label{ch6}

% History: 
% Version 1
%  limitations:
%      1. hierarchy establishment,
%           in hierarchy, error accumulates 
%           brain decides everything

%      2. Task assignment formation
%           fully optimization can not be achieved
%           there are some cases unsolved

%      3. Behavior tree management, and DBT to autonomous
%           cannot invent code on their own

% future direction:
%       1. solve time-delay and error accumulation
%       2. more robots to be brain collectively
%       3. task assignment
%       4. invent code
%       

%-------------------------------------------------------
%
%1. Key Contributions Summary
%    1.1 Proposed the SoNS to address the "decentralized vs. centralization" contradiction in swarm robotics; enables autonomous dynamic hierarchical architecture (Sci-Ro paper).
%    1.2 Applied the SoNS to 1,000 UAVs; verified the feasibility of large-scale high-velocity formation under non-central control (1000 drone paper).
%    1.3 Integrated DBTs into the SoNS; achieved the first runtime reprogrammable swarm (supporting runtime code transfer and reprogramming); enhanced swarm autonomy and human-swarm interaction (builderbot paper).
%    1.4 Significance: Bypasses the micro-macro problem to facilitate swarm behavior design; enables human-swarm interaction/reprogramming in large-scale scenarios; improves swarm autonomy.

%2. Current Limitations
%    2.1 Communication dependence: The SoNS relies heavily on communication. In case of communication failure (even if recovered after communication is restored), the SoNS fails to operate.
%    2.2 Error accumulation and time latency: Error accumulation was demonstrated in the Sci-Ro paper; the 1000 drone paper adopted a truss mechanism to reduce error accumulation, but further investigation is still required.
%    2.3 Human-Swarm Interaction: Code transfer needs verification—risks exist if invalid code is transferred by operators.

%3. Future Research Directions
%    3.1 Operation under weak communication conditions: Develop strategies to enable SoNS function under limited or unstable communication.
%    3.2 Mitigation of error accumulation and time latency: Conduct in-depth research to further reduce error accumulation and optimize time latency.
%    3.3 Integration of LLM and code transfer: Combine large language models (LLMs) with code transfer to enable direct LLM-based swarm reprogramming.
%    3.4 Cross-Domain Use: Test the SoNS in cross-medium swarms (e.g., air-ground-water robot collaboration) and emergency response scenarios.


\section{Summary of Contributions}

% P1 Sci-ro paper
% s1, this thesis proposed SoNS
This thesis proposed Self-organizing Nervous System (SoNS), a dynamic hierarchical architecture for robot swarms.
% s2, propsed how to manupulate SoNS
The thesis described algorithm to enable autonomous establishment, maintenance, and reconfiguration of SoNS.
% it solves centralized - decentralized paradox.
This design resolves the tension between decentralized scalability/fault tolerance and centralized coordination efficiency, laying the groundwork for practical swarm applications.

% P2 1000 drone paper
% expand sons to 1000 drones
By applying SoNS to a swarm of 1,000 UAVs, SoNS is proved to be able to demonstrate feasible large-scale, high-velocity formation control under non-centralized management.
% it shows large scale without ground control, real field application.
It lays fundamental proof for next generation of real field applications such as drone light shows, with large scale and without centralized ground control.

% P3 Builderbot paper
% SoNS use DBT to enable code transfer
By integrating Dynamic Behavior Trees (DBTs), SoNS firstly enable runtime code transfer and reprogramming to swarm robotics.
% it makes autonomy and HSI
This advancement enhances both swarm autonomy, adapting to unforeseen environments, and human-swarm interaction, for safety or on-demand adjustments.

% P4 significance
Collectively, SoNS ease the design of the swarm behavior despite the micro-macro problem (simplifying behavior design by unifying individual programming and group behavior), enable reliable human-swarm interaction in large swarms, and improve swarm autonomy—filling critical gaps in existing research.


\section{Limitations}

% P1 However...
While the SoNS advances swarm robotics, it faces practical limitations that constrain broader deployment.

% P2 Communication reliability
First, the SoNS relies heavily on stable inter-robot communication.
Temporary communication failures disrupt SoNS operation.
Although SoNS demonstrates great power to quickly recover after communication failure, it still exposes vulnerabilities in low-connectivity environments.

% P3 error accumulation
Second, errors caused by sensor noise and time latency accumulates along the communication chain across the swarm.
Chapter 3 documented error accumulation as formation grow larger and larger.
Although Chapter 4 introduces a truss mechanism to reduce this issue, residual errors and associated latency are still observable and requiring further investigation.

% P4 code transfer, what if bad code?
Third, code transfer feature Chapter 5 lacks validation for invalid code risks.
If operators transfer flawed or malicious code, the entire swarm’s functionality could be compromised, as current designs lack integrity checks or error-handling for such scenarios.

\section{Future Directions}

% P1 future mainly aim to solve above mentioned limitations.
Future work would mainly target the identified limitations to expand SoNS’ capabilities and real-world applicability.

% P2 weak communication
To ease communication reliability, one possible direction is to lower frame rate.
That would pose challenge for velocity-based control.
Robots would need to predict other robots movement and constructs reaction, that may be need game theory.
\textcolor{red}{other options}

% P3
For error accumulation and latency, advanced solutions—such as machine learning-based predictive error compensation or optimized hierarchical update logic—could be a possible direction to further improve formation precision and mission reliability for large or long-duration swarms.

% P4
Integrating large language models (LLMs) with SoNS’ code transfer would simplify human-swarm interaction: operators can issue natural language commands (e.g., "Form a search grid"), which the LLM converts into executable code, eliminating manual coding and making SoNS accessible to non-experts.

% P5
Chapter 3 demonstrates a cross-domain scenario with drones and ground robots.
Further testing would validate SoNS in heterogeneous cross-medium swarms (air-ground-water robots) and high-stakes emergency scenarios (post-earthquake rescue, wildfire monitoring), accelerating its transition from lab demonstrations to practical deployment.