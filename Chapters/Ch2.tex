\chapter{Related Work}
\label{ch2}

\section{Introduction}

The field of swarm intelligence often draws inspiration from the collective behaviors of social animals such as ants, birds, and fish
    \cite{bonabeau1999swarm}.  % the book of Swarm intelligence
Early models such as the Reynolds' Boids model of flocking have demonstrated that simple local behavioral rules can give rise to coherent large-scale patterns
    \cite{reynolds1987flocks}.
Swarm intelligence is based on self-organization: the principle that individuals interact with each other locally without centralized control, and complex global behavior emerges.
Swarm robotics aims to leverage the principles of self-organization to construct decentralized robot systems that are scalable, adaptive, and fault-tolerant
    \cite{dorigo2014swarm,       % scholarpedia on swarm robotics
          dorigo2020reflections, 
          dorigo2021swarm}.    %past, present and future
Self-organized systems have shown potential in many applications, including
    environmental monitoring
    \cite{talamali2021less},
    navigation and transportation
    \cite{dorigo2013swarmanoid},
    self-assembly
    \cite{rubenstein2014programmable},
    and collective construction
    \cite{team2012designing, petersen2019review}.
Owing to the absence of centralized control, such systems can scale up to vast numbers of robots without demanding increasing capabilities from each individual robot.

However, despite substantial progress, it remains a fundamental scientific question how to design low-level individual robot behaviors that result in desired high-level collective outcomes.
The micro-macro problem refers to this challenge
    \cite{hamann2018swarm,dorigo2021swarm}.
Although various theoretical frameworks have been proposed
    \cite{hamann2008framework,
          hamann2010space,
          %hamann2018swarm,
          hamann2013towards}
and decentralized controllers have been demonstrated for specific tasks
    \cite{dorigo2004evolving,
          nouyan2009teamwork,
          dorigo2013swarmanoid,
          rubenstein2014programmable,
          li2019decentralized},
purely decentralized swarms are often difficult to design and manage, and might take a long time to complete the desired task%, especially for tasks that require global information, such as environmental analysis and decision making.
    \cite{dorigo2021swarm,
          kengyel2015potential}. % pure flat is not good

The limitations of purely decentralized control motivate research into hybrid and hierarchical approaches that integrate selective forms of centralization into decentralized systems.
These systems seek to bypass the micro-macro problem,
                      ease the design of swarm behaviors and coordination strategies,
                      and improve the manageability of the system,
while preserving scalability and fault tolerance.

This chapter reviews prior work in hybrid and hierarchical swarm control,
compares the existing literature on decentralized and centralized approaches to different tasks,
and discusses aspects of swarm manageability, including human-swarm interaction, swarm re-programming, and swarm autonomy.

\section{Heterogeneity and leadership}

Flat and homogeneous robot swarms, often inspired by examples from nature,
    \cite{buhl2006disorder,
          detrain2008collective,
          theraulaz1998origin},
assume that all agents are identical, follow the same behavioral rules, and interact with each other in the same way
    \cite{viragh2014flocking,
          vasarhelyi2018optimized,
          floreano2008bio,
          csahin2004swarm,
          beni1988concept}.
By contrast, heterogeneous swarms consist of individuals serving different roles or having different behaviors.
These systems,
with individuals of different physical types or with different information,
can outperform purely homogeneous systems, depending on the task requirements
    \cite{kengyel2015potential}. % hetero is better
For example, individuals with more information can influence the behavior of others for faster convergence or higher adaptability during tasks such as flocking or collective decision-making
    \cite{firat2020self, % use informal individual to cue the whole swarm to aggreate to a desired shelter
          prasetyo2018best,  % some stubborn individuals
          balazs2020adaptive}.

In another type of heterogeneous swarm, a subset of individuals are assigned explicit leadership roles
    \cite{gu2009leader,
          amraii2014explicit}.
In these approaches, the leaders are usually pre-defined and fixed after deployment, while the rest of the coordination remains self-organized and based on local interactions. 
Each follower is assigned a leader and receives information from it, such as position references.
In this way, information flows in a direct manner from informed leaders to uninformed followers, thereby easing the design of the behaviors and reducing task completion times.
Many tasks can benefit from this type of organization, for example flocking
    \cite{dalmao2011cucker,
          jia2019modelling,
          pignotti2018flocking}.
%\textcolor{red}{consider drop:}
%These leaders may be vulnerable to adversarial detection, a concern explored in    \cite{zheng2020adversarial}.


%In some works of self-assembly or formation
%    \cite{desai1999control,
%          li2019decentralized,
%          rubenstein2014programmable},
%although leaders and followers are not explicitly pre-assigned, each robot implicitly selects a leader upon joining the system and adjusts its position based on that leader’s state, effectively forming a leader–follower hierarchy.

%and thereby improving efficiency in tasks
%Such approaches can be regarded as highly heterogeneous systems in which each individual assumes its own role.

Existing leader--follower approaches can be organized into two-level or multi-level systems, formed by fixed leader--follower pairs.
However, this results in a system with a predefined topology, which suffers from low scalability and the presence of single points of failure.
No existing approach has provided a general mechanism for self-organizing hierarchy—that is, adaptively forming and maintaining multi-level structures without predefined topologies or fixed leadership roles.
%As task complexity increases, self-organizing hierarchy becomes increasingly relevant. However, 
Constructing hierarchy in a self-organized manner remains a major challenge
    \cite{dorigo2020reflections}. 

\section{Self-organized versus centralized or predetermined control}

Purely self-organized swarms are generally scalable and fault-tolerant,
but their behaviors are often difficult to design,
and they often exhibit inferior performance.
By contrast, systems with centralized or predetermined control are typically easier to design and can complete tasks more quickly and with higher accuracy by exploiting global information, but they lack scalability and fault tolerance.
This section reviews the different control approaches and their advantages and disadvantages across several key example tasks in multi-robot systems.

\subsection{Multi-robot coverage}

Coverage is a key aspect of multi-robot coordination for many applications.
The goal of multi-robot coverage is to ensure that a target region is fully observed, sensed, or explored by a group of agents.
Effective coverage is thus essential for applications such as environmental monitoring, search and rescue, precision agriculture, and surveillance.
There are two types of coverage: area coverage (maintaining spatial distribution over a region) and sweep coverage (systematically traversing a region to ensure full visitation).
A wide range of coverage approaches have been developed, and most can be classified into three main categories: sensor dispersion for area coverage, random exploration for sweep coverage, and predetermined trajectories for sweep coverage
    \cite{wang2011coverage,     % survey
          galceran2013survey}.  


\textbf{Sensor Dispersion.~~}
In the sensor dispersion approach to area coverage, sensors spread over the environment to cover the target area,
typically by robots dispersing to well-separated positions that remain largely static once coverage is achieved,
with a few notable exceptions in dynamic environments
    \cite{santos2019decentralized}.
In these approaches, although sensor localization might rely on GPS or distance-based measurements, the interactions among the robots are typically decentralized.
Robots use mechanisms such as potential fields, local repulsion, or other local policies to maintain appropriate separation and avoid sensing overlap.
Such methods can offer scalability and robustness due to the robots' decentralized interactions
    \cite{santos2019decentralized,
          luo2018adaptive,
          spanogianopoulos2017fast,
          siligardi2019robust}.
However, these methods require a large number of robots to achieve coverage, especially in large-scale environments.

\textbf{Random Exploration.~~}
In the random exploration approach to sweep coverage, robots sweep the environment in parallel, moving randomly while also performing some type of collision avoidance.
The motion control in these approaches is fully decentralized, but sometimes centralized coordination is used for other steps occurring after coverage, such as fusing collected information.
Random exploration approaches to sweep coverage is easy to implement because the behaviors are very simple and only minimal communication and sensing information is used during operation. 
However, they often suffer from long completion times and inefficient coverage, due to redundancy and non-uniform exploration
    \cite{huang2019exploration,
          ichikawa1999characteristics,
          mcguire2019minimal}.

\textbf{Predetermined Sweeps.~~}
When using predetermined trajectories for sweep coverage, all robots' paths are pre-calculated to ensure full coverage of a known region, often using S-shaped trajectories.
These approaches use explicit predetermined roles and global map knowledge, which yields fast and predictable coverage, but usually lacks adaptability and can suffer from communication bottlenecks and single points of failure
\cite{almadhoun2019survey, avellar2015multi}.
There are also mixed approach variants:
for example, \cite{scherer2015autonomous} uses fully predetermined motion trajectories,
but the robots stream data using a relay network formed in a self-organized manner.

To summarize, area coverage approaches based on sensor dispersion are typically decentralized, robust, and scalable, but can require a prohibitively large number of robots to achieve satisfactory coverage.
In contrast, sweep coverage approaches can use a relatively small number of robots to explore a large area.
While random exploration approaches are robust and scalable, they often suffer from long completion times and inefficient coverage.
Conversely, predetermined sweep approaches can achieve fast and uniform coverage, but usually lack adaptability and fault tolerance.

\subsection{Multi-robot Path Planning}

Path planning is important for many multi-robot system applications.
Existing path planners include centralized and/or predetermined offline approaches as well as distributed online approaches.

\textbf{Centralized Offline Planning}
Similar to predetermined approaches to sweep coverage, centralized path planners use global map knowledge and compute predetermined collision-free trajectories for all robots before deployment.
Multi-robot path optimization usually employs heuristics such as genetic algorithms or particle swarm optimization 
    \cite{nazarahari2019multi,
          thabit2018multi,
          yu2016optimal,
          kushleyev2013towards}.
These methods can produce high-quality paths, but because the paths are fully predetermined, they lack fault tolerance, flexibility, and adaptability.

\textbf{Distributed Online Planning}
Although most path planners are offline, there are a few existing examples of distributed online path planners, such as EGO-Swarm
    \cite{zhou2020ego,
          zhou2021ego,
          zhou2022swarm}.
The EGO-Swarm planner uses local sensing and onboard optimization, %Fast distributed trajectory generation is achieved by predicting nearby robots' short-horizon behavior and optimizing smooth, dynamically feasible paths at high frequency.
enabling agile navigation in cluttered environments. However, the generated solutions are not perfectly optimal, in contrast to what could in principle be achieved using offline global optimization.  
Also, although the trajectory generation is decentralized, the start and goal positions of each robot are predetermined, which limits scalability and reduces robustness to agent failures.

In summary, centralized offline path planners can use global information to generate high-quality paths but lack flexibility, adaptability, and robustness to robot failures.
Distributed online planners are much more flexible, but still fall short of providing fully scalable and fault-tolerant mechanisms for both defining and reaching goal positions.


\subsection{Formation control}

Formation control is another key topic in multi-robot systems
    \cite{liu2018survey,
          oh2015survey}, % surveys
on which there is a vast amount of literature with substantial overlap with path planning. Because path planning is discussed in the previous section, this section focuses on the positioning and role assignment aspects of formation control.

Both centralized and decentralized formation control approaches can use relative positioning, for example based on distance sensors, ultraviolet markers, or fiducial markers
    \cite{walter2018fast, %localization of UaVs using ultraviolet 
          ulrich2022towards}.  %fast fiducial marker with full 6 dof pose estimation
When using relative positioning, stable rigid-body formations can be achieved using distance-based constraints
    \cite{yang2018growing,
          mehdifar2018finite,
          stacey2015passivity,
          anderson2018rigid,
          oh2011adistance,
          oh2011bdistance}
%which maintains formation using distance measurements,
or bearing-based constraints
%which allows formation maintenance using only bearing measurements, with two designated leaders controlling scale, rotation, and translation while other robots maintain relative bearings
    \cite{zhao2019bearing,
          zhao2015bearing,
          zhao2015translational,
          schiano2016rigidity,
          li2020adaptive}.
%          li2021adaptive,
%          li2020bearing,
%          zhao2021finite,
%          zhang2022distributed,
%          zhang2023bearing}.            % for water surface
%However, these methods assume that each robot is preassigned a fixed role in the formation.
However, these methods generally assume that each robot is pre-assigned a fixed role in the formation.
During deployment,
the interaction topology among the robots is predefined and fixed, and robots move by referencing the positions of the robots they are connected to in this predefined topology.
They offer little support for dynamic or fault-tolerant operation, emphasizing control-theoretic guarantees over flexibility or adaptivity.
%
%On the other hand, a key feature of swarm robotics is the ability to deploy robots without pre-assigned roles.
%Moreover, for fault tolerance, the swarm should be able to reassign roles in response to unexpected failures.
%These considerations highlight the importance of role assignment.
%In formation tasks, efficient role assignment can reduce unnecessary crossing movements among robots and thereby speed up formation convergence.
In addition, in large swarms and/or swarms prone to agent failures, pre-assigning roles is often impractical.
%Some self-assembly approaches allow robots to autonomously select vacant positions, but these methods are often inefficient due to random search
%    \cite{rubenstein2014programmable,
%          li2019decentralized}.

There are both centralized and decentralized approaches to dynamic role assignment.
Centralized approaches compute globally optimal role assignments by leveraging full knowledge of robot positions and formation targets
    \cite{rm2020review,
          macalpine2015scram,
          agarwal2018simultaneous,
          ravichandar2020strata,
          akella2020assignment},
ensuring minimal total displacement and/or collision-free paths.
%
Distributed assignment approaches instead divide computational responsibilities among robots,
using, e.g., consensus mechanisms, market-based coordination, or distributed optimization algorithms
    \cite{mosteo2017optimal,
          burger2012distributed,
          chopra2017distributed,
          zavlanos2007distributed,
          alonso2016distributed,
          michael2008distributed,
          montijano2014efficient,
          wang2020shape}.
In many such approaches, although computation is distributed, the per-robot computational and memory burden still typically grows with system size.
For example, some methods require each robot to solve a portion of a global optimization problem, such that the calculation dimensionality of each portion still increases with the number of agents.
Other methods rely, for example, on bidding or consensus schemes in which each robot must maintain information about all robots in the system.
That is why, despite some aspects of these approaches being distributed, they still struggle to scale efficiently with system size.
Distributed approaches that are specifically designed to handle faulty or interchangeable agents also remain limited
    \cite{kambayashi2018distributed}.

%In addition, switching formations have been explored: that is, a series of target formations are predefined and the system is triggered to switch between them 
%    \cite{desai1999control,
%          desai2002modeling}.
%However, these methods still typically rely on centralized commands to trigger the formation switches, rather than the system adapting autonomously.

In summary, formation control approaches that incorporate centralization or predetermined roles simplify optimization and achieve high-quality formations, but suffer from limited robustness and scalability.
Existing approaches to distributed role assignment typically impose large computational or communication burdens on individual robots, which still hinders scalability and fault tolerance in large swarms.


\subsection{Commercial drone entertainment shows}

Drone light shows are becoming popular in commercial entertainment.
So far, these drone shows have used fully centralized and predetermined approaches to robot coordination,
where global trajectories are precomputed offline and any commands that are issued during the show are broadcast to all agents from a ground station.
This can provide precise synchronization and visually coherent formations with very many robots
    \cite{waibel2017drone,
          ang2018high}.
However, the resulting systems are vulnerable to a single point (ground station) of failure and to communication failures and bottlenecks,
lacking the scalability or adaptability typically associated with decentralized systems.

\section{Swarm behavior design and manageability}

A fully autonomous multi-robot system should be able to make decisions and adapt to a changing environment without requiring human intervention in order to accomplish its goals. 
At the same time, these systems must remain manageable: human operators should still have the option to supervise the system and optionally intervene if they desire.
This goal is particularly challenging for swarm systems, since robots self-organize using only local information, which often leads to long convergence times, makes collective behaviors difficult to design, and makes it difficult for human commands to propagate through the swarm.
This section reviews the state of the art in swarm autonomy and manageability.

\subsection{Swarm autonomy}

%Autonomy is often regarded as a central long-term goal of robot systems—the ability for robots to operate without requiring human supervision or intervention.

Despite significant advances in individual robot autonomy
    \cite{cangelosi2022cognition,
          vernon2014artificial},
achieving autonomy in swarm robotics remains an open challenge.
Existing swarm robotics research primarily focuses on a single aspect of autonomy such as action, perception, and adaptation
    \cite{heinrich2022swarm},
but other aspects such as learning or anticipation are still largely absent,
and swarm robotics lacks an overall framework to unify all aspects of autonomy.

Currently, the most-studied aspect of swarm autonomy is best-of-$n$ collective decision making, of which the binary (best-of-two) case is a special instance
    \cite{khaluf2019neglected, % decision making
          valentini2017best,   % best of N
          dorigo2014self,
          valentini2016collective,
          shan2020collective,
          ebert2018multi,
          prasetyo2018best,       % best of N
          prasetyo2019collective, % best of N
          valentini2015efficient}. % best of N
These works demonstrate how simple local rules allow groups to reach agreement without centralized control.
Some works investigate specific aspects such as convergence speed, decision accuracy, and robustness under noise or malicious information
    \cite{shan2020collective,            % bayesian
          bartashevich2019benchmarking,  % 
          shan2021discrete, %\textcolor{red}{too tech}
          strobel2018managing}. %\textcolor{red}{ref incomplete} % byzenting
Despite this progress, decision making in swarms generally remains an isolated task rather than being integrated into a broader understanding of the mission context.
Many elements are missing in terms of autonomy,
for example the ability to detect that a new collective decision is required, or to autonomously adapt the decision-making process to conditions that were not explicitly studied beforehand
    \cite{khaluf2019neglected}.
Current frameworks provide no unified mechanisms for learning new behaviors online, anticipating future events, or autonomously restructuring task workflows.

The SAE J3016 standard
    \cite{sae2021automated}
defines autonomy levels for self-driving cars, but is used here because no standards explicitly for robot swarms have yet been defined.
Using the SAE J3016 standard, existing robot swarms would be barely classified as having collective behaviors equivalent to the individual behaviors described in {\it Level 2 (Partial automation)},
as existing robot swarms are able to
execute intended tasks, but a human operator is still required to continuously monitor the environment and system and decide when to intervene to correct or stop the system.
Also, human intervention in the swarm's operation would itself still be a challenge, as discussed in the following section.

\subsection{Human-swarm interaction}

Although robot swarms are generally capable of executing their intended tasks,
human operators still play a crucial role in supervising the system,
issuing high-level commands and reprogramming the swarm when necessary.
However, as swarm size increases, direct communication with all the robots in the swarm becomes infeasible.
This section reviews existing research on human–swarm interaction and swarm online programming.

Current human-swarm interaction (HSI) research mainly focuses on two aspects:
the design of human side of the HSI interfaces,
and the mechanisms through which human commands sent to a small subset of robots can influence the collective behavior of the swarm
    \cite{siean2023opportunities,
          kolling2015human}.
Existing research on HSI interfaces explores manipulating the swarm using gesture control
    \cite{alonso2015gesture,
          podevijn2013gesturing,
          macchini2021personalized}, % leap motion, control master drone. Other drones follow by Renolds
haptic devices
    \cite{lee2013semiautonomous},
or wearable systems
    \cite{jarvis2025first}.
Among these works,
operators usually broadcast commands to all robots directly
    \cite{ayanian2014controlling,
          abioye2025user}
which can lead to scalability issues due to the communication burden as the swarm size grows.
For decentralized methods where humans communicate with only one or a few robots, existing research focuses on how a subset of the robots can influence the rest of the swarm
    \cite{podevijn2013gesturing,  % gesture, subset of the swarm
          zhou2016assistive,      %     joysticks
          lee2013semiautonomous,  % haptic feedback
          kolling2013human}.      % select (human select some robots) and beacon (human put a beacon and influence nearby robots) 
While these methods improve scalability,
they often suffer from slow convergence to the desired swarm-wide behavior
and also make the design of swarm-wide behaviors challenging.
In general, there is a lack of a framework that simultaneously addresses scalability, convergence efficiency, and ease of behavior design in human–swarm interaction.

\subsection{Online programming for robot swarms}

When supervising a robot swarm, one important aspect of human–swarm interaction is the ability to re-program the swarm easily and online.
However, swarm reprogramming faces two challenges.
First, designing controllers for swarm systems is often a lengthy trial-and-error process
    \cite{hamann2018swarm,
          brambilla2013swarm}
that becomes more challenging when swarms must adapt to new tasks or environments.
Second, delivering new programs to all robots often requires the operator to communicate with the entire swarm, which can lead to scalability issues.
This section reviews approaches from offline, centralized online, and to decentralized online reprogramming.


Typically, robot swarm programs are pre-designed and downloaded offline to all the robots before deployment, for example, as in
    \cite{rubenstein2014programmable,
          valentini2016collective,
          werfel2014designing,
          dorigo2013swarmanoid}.
However, these offline methods cannot react to unexpected situations such as task changes during deployment.
%Notably, automatic design methods falls in this category
%    \cite{francesca2014automode,
%          francesca2016automatic,
%          birattari2019automatic,
%          salman2024automatic}.
%They demonstrate how optimization and evolutionary algorithms can produce robust swarm behaviors without manual tuning.
%Yet the programs are pre-generated.
Automatic design methods demonstrate how approaches such as optimization and evolutionary algorithms can produce robust swarm behaviors without manual tuning
    \cite{francesca2014automode,
          francesca2016automatic,
          birattari2019automatic,
          salman2024automatic}.
However, these methods still typically rely on extensive trial-and-error processes prior to deployment and are therefore most often applied in an offline manner.
New behaviors can in principle be designed offline and then uploaded to robots during operation. However, delivering and updating these programs across all robots remains challenging.

Most existing approaches to online reprogramming are centralized.
Centralized online reprogramming techniques update robot behaviors after deployment through a global communication channel.
Over-the-air programming frameworks
    \cite{zyrianoff2024over,
          abadie2024robotap}
enable operators to push new software to all robots.
Large-scale drone light shows
    \cite{waibel2017drone,
          ang2018high}
represent a widely deployed example of centralized online control.
Although they can coordinate thousands of robots,
they typically allow only predefined trajectories and very simple commands such as taking off or emergency landing.
Decentralized reprogramming methods update robot controllers without relying on a central broadcaster.
Many studies in wireless sensor networks
    \cite{xie2011design,
          wang2006reprogramming}
demonstrate the feasibility of distributed software updates.
However, in swarm robotics studies, in which over-the-air programming is used to update only a subset of the robots, the rest of the swarm is slow to converge on the new update
    \cite{de2009energy,
          varadharajan2018over},
and it remains challenging to coordinate consistent behavior across all the robots.
To summarize, a unified framework that enables a human operator to reprogram swarm behavior in a scalable and efficient manner has not yet been developed.

%\textcolor{red}{Behavior tree, move later}
\iffalse

\subsubsection{Behavior tree}

Behavior trees (BT) have been increasingly explored as a tool for structuring robot behaviors in a modular and hierarchical manner.
The compositional structure makes it possible to modify or update parts of a controller without redesigning the entire behavior,
which is particularly appealing in the context of online robot reprogramming.

Behavior trees (BTs) provide a modular and hierarchical framework for representing robot behaviors
    \cite{colledanchise2018behavior,
          iovino2022survey},
which have been adopted in several multi-robot applications
    \cite{colledanchise2016advantages,
          jeong2022behavior}.

Due to their compositionality and readability, BTs offer a potential solution to key challenges in swarm autonomy and human–swarm interaction. 
They allow robots to adapt their behavior online, enable hierarchical organization of complex tasks, and facilitate interaction with human operators through modular control structures. 

Evolving BTs via evolutionary algorithms—such as genetic programming
    \cite{jones2018evolving,
          kuckling2022automode}
or grammatical evolution
    \cite{neupane2019learning,
          kuckling2022automode}—
enables automated creation of swarm behaviors.
Recent advances support online evolution as well
    \cite{jones2019onboard,
          venkata2023kt},
allowing robots to adapt their BT controllers during deployment.

Dynamic BTs have also been investigated in game AI
    \cite{florez2008dynamic},
though their application to real-world robotics remains to be explored.

More recent work exploit behavior trees for knowledge transfer in multi-robot systems as a adaptability mechanism.
For instance, Venkata et al.\ \cite{venkata2023kt} introduce the KT-BT framework, which uses behavior trees and a grammar-based encoding (stringBT) to achieve decentralized transfer of acquired skills, improving collective performance during online adaptation.

\fi

\section{Fault tolerance}

Swarm systems show some degree of inherent fault tolerance because of their redundancy and absence of single points of failure
    \cite{dorigo2014swarm,
          hamann2018swarm}.
Most commonly-studied swarm tasks, such as coverage or collective decision making, can tolerate a subset of robots failing.
Robots simply continue to operate as normal, without knowing or caring about failed peers
    \cite{bjerknes2013fault,
          khadidos2015exogenous}.
Despite these advantages, fault tolerance in robot swarms remains far from a solved problem.
Existing approaches often struggle with limitations such as fault detection and diagnosis under partial observability, communication and sensing failures, malicious or Byzantine agents,
and performance degradation caused by cascading or correlated failures
    \cite{hamann2018swarm}.
Among these challenges, intermittent faults are a difficult and representative example case.
An intermittent fault (IF) is a failure that occurs sporadically: it can appear, disappear, and reappear.
Its sporadic nature makes reliable detection difficult,
and repeated occurrences can significantly degrade overall swarm performance
    \cite{zhou2019review,
          niu2021distributed}.
Centralized methods are highly effective for identifying IFs, since they have access to system-wide data
    \cite{sheng2021intermittent,
          zhang2021intermittent,
          syed2016novel},
but  centralized methods are vulnerable to single points of failure and are not scalable.
Overall, although robot swarms benefit from some degree of inherent fault tolerance,
addressing certain fault types still relies on centralized mechanisms,
thereby undermining scalability and resilience.

\section{Summary}

In summary, centralized approaches offer an easy solution for designing and managing collective behavior and can often coordinate the robots to finish a task quickly,
while decentralized approaches can provide greater robustness and scalability.
Yet purely decentralized swarms are often difficult to design and manage, and may take a long time to complete the task, especially in scenarios in which global information is relevant.

Neither paradigm alone fully addresses the need for coordinating large robot swarms to perform dynamic tasks that require both adaptability and performance guarantees.
A self-organizing hierarchical framework is a strong candidate to bridge this gap.
The goal of this thesis is to show the design of such a self-organizing hierarchical system that
enables swarms to dynamically reconfigure their coordination structure,
                   balance local autonomy with global oversight,
               and achieve human manageability, all without sacrificing scalability, flexibility, and fault tolerance.


\iffalse
\section{Abandon}

\textcolor{blue}{from Ayros paper}

multi-robot fusion problems are well understood, and existing methods are powerful
centralized
\textcolor{red}{About collective sensing : make it its section in applications}
These are all surveys
\cite{yan2013survey} 
\cite{sun2017multi}
\cite{rizk2019cooperative}
\cite{li2021multi}
\textcolor{red}{consider drop this sensor fusion part}


another scenario, about the quality of the block
\cite{khaluf2017edge}  find edge of an area
\cite{wahby2019collective} aggreate to the bright area
\cite{khaluf2020construction} about construction, fusion sensing the density of the building
\cite{capitan2013decentralized} 

not fully de-centralized
\cite{mirzaei2007performance} fixed sensor  \textcolor{red}{a bit tech}
\cite{rodrigues2015beyond} 
\cite{stroupe2001distributed}
\cite{zadorozhny2013information}
\cite{sasaoka2016multi}
\cite{czarnetzki2010handling}
\cite{otte2016collective} This is one a bit different, swarm robot nerval network
\cite{kornienko2005cognitive}
\cite{giusti2012cooperative}
\fi
