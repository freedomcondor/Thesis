t \chapter{Related Work}
\label{ch2}

% skeleton
% 1 Hybrid, hierarchical, and leadership approaches in the literature   TODO from Sci-Ro paper, yuwei paper
% 2 Swarm vs. centralized approaches for different tasks (explain what is possible to do with both, for those tasks -- current limitations are implied)
%   1.1 formation control
%   1.2 exploration, search and rescue    TODO from Sci-Ro paper
%   1.3 drone entertainment shows...
%   1.4 (more?)
% 3 The process of swarm programming / re-programming (mostly about how this is done offline, don't worry about the limitations when trying to do this online)
% 4 Swarm autonomy    TODO from Builderbot paper and zotero

\section{Introduction}

\textcolor{blue}{what is swarm}

Swarm robotics draws inspiration from collective behaviors in social animals such as ants, birds, and fish
    \cite{bonabeau1999swarm}  % the book of Swarm intelligence
.
Early models like Reynolds' Boids demonstrated that simple local behavioral rules can give rise to coherent large-scale patterns
    \cite{reynolds1987flocks}
.
Swarm Robotics aims to leverage these principles to construct decentralized robotic systems that are scalable, adaptive, and fault tolerant
    \cite{dorigo2014swarm,    % scholarpedia on swarm robotics
          dorigo2020reflections,
          dorigo2021swarm}    %past, present and future
.
These decentralized systems have shown advantages in large-scale or spatially distributed tasks, including
    environmental monitoring
    \cite{talamali2021less},  \textcolor{red}{varify}, 
    navigation and transportation
    \cite{dorigo2013swarmanoid},
    self-assembly
    \cite{rubenstein2014programmable},
    and collective construction
    \cite{team2012designing, petersen2019review}
.

\textcolor{blue}{but swarm is not always efficient}

\textcolor{red}{Maybe remove this sentence: Despite significant progress, it remains a fundamental challenge to connect low-level individual robot behaviors to the desired high-level collective outcomes.}

Despite that various theoretical frameworks have been proposed
    \cite{hamann2008framework,
          hamann2010space,
          hamann2018swarm,
          hamann2013towards}
and decentralized controllers have been demonstrated for specific tasks
    \cite{dorigo2004evolving,
          nouyan2009teamwork,
          dorigo2013swarmanoid,
          rubenstein2014programmable,
          li2019decentralized}
.
However, purely decentralized swarms often face fundamental inefficiencies, particularly due to their reliance on strictly local information
    \cite{dorigo2021swarm}

\textcolor{blue}{Therefore, hierarchical}

These limitations motivate research into hybrid and hierarchical approaches that integrate selective forms of centralization 
    \cite{kengyel2015potential} % pure flat is not good

\textcolor{blue}{In this chapter, review}

This chapter reviews prior work in hybrid and hierarchical swarm control,
compares decentralized and centralized approaches across key swarm robotics tasks,
and examines related developments in formation control, coverage, manageability, re-programming, and swarm autonomy.

\section{Heterogeneous, Hybrid, and hierarchical Approaches}

\textcolor{blue}{flat swarms}

Flat and homogeneous swarms, inspired directly by natural self-organizing processes
    \cite{buhl2006disorder,
          detrain2008collective,
          theraulaz1998origin}
,
assume that all agents are identical and follow the same behavioral rules
    \cite{viragh2014flocking,
          vasarhelyi2018optimized,
          floreano2008bio,
          csahin2004swarm,
          beni1988concept}
.
Such systems excel in robustness and scalability but struggle with tasks that require differentiated roles, long-range coordination, or efficient global organization.

\textcolor{blue}{heterogeneous}

On the other hand, heterogeneous swarms or swarms with informed individuals can often outperform purely homogeneous systems
    \cite{kengyel2015potential} % hetero is better
.
Informed or persistent agents can influence the behavior of others during tasks such as flocking or collective decision-making
    \cite{firat2020self, % use informal individual to cue the whole swarm to aggreate to a desired shelter
          valentini2016collective,
          balazs2020adaptive}
    \textcolor{red}{varify: valentini2016collective}
,
Some works explicitly name these individuals as leaders
    \cite{gu2009leader}
    \cite{amraii2014explicit}
.
These leaders may be vulnerable to adversarial detection, a concern explored in 
    \textcolor{red}{consider drop}
    \cite{zheng2020adversarial}
.

\textcolor{blue}{hierarchical}

Hybrid or hierarchical approaches attempt to combine decentralized robustness with centralized efficiency.
They introduce temporary leaders, role assignments, or distributed authority transfer mechanisms to improve swarm efficiency in flocking
    \cite{dalmao2011cucker,
          jia2019modelling,
          pignotti2018flocking}
and self-assembly and formation
    \cite{li2019decentralized,
          divband2019photomorphogenesis}
.

However,
it remains challenging
to keep scalability and avoid a single point of failure
while constructing hierarchy with such mechanisms.
    \cite{dorigo2020reflections}
.
As tasks grow in complexity, researchers increasingly recognize the importance of *self-organized hierarchy formation*, an ability still largely absent in classical decentralized swarms.

\section{Swarm vs. centralized approaches for different tasks}

\textcolor{red}{the point is}
\textcolor{red}{1 we want to show there is a gap :  self-organized hierarchy}
\textcolor{red}{2 both sides have some downsides}

In this section, we will discuss the swarm vs. centralized approaches for classic swarm tasks like formation etc.

\subsection{Formation}

Formation control is one of the most actively studied topics in swarm robotics or multi-agent systems
    \cite{liu2018survey}% a survey
    \cite{oh2015survey}% a survey
.
A variety of formation control problems have been studied in the literature for different situation assumptions.

It relies on localization of neighbours
\cite{walter2018fast} %localization of UaVs using ultraviolet 
leD markers
\cite{ulrich2022towards} %fast fiducial marker with full 6 dof pose estimation

\subsubsection{Fixed roles}

These class usually focus in control theory.

\textcolor{blue}{They are decentralized in the sense of movement (robots move according to neighbours), but centralized in the sense of roles (predefined roles).}
These researches usually assumes fixed roles, each robot knows their target positions already.
They focus on control theory in velocity control or acceleration control to converge to the desired formation.


They talk about how to form a rigid body with each robot detects the distance or bearing of the predefined neighbours
    \cite{yang2018growing}
    \cite{mehdifar2018finite} 
    \cite{stacey2015passivity}.

\textcolor{red}{distance rigidity, cite something here}


\textcolor{blue}{bearing rigidity}
Bearing rigidity is a way to form a rigid body with only sense of bearing
    \cite{zhao2019bearing}.
In bearing rigidity, there are two leaders, other ones try to keep a certain bearing with neighbours, and the two leaders control the scale, rotation and translation of the formation.

Mathematical researches have developed theories on bearing rigidity from 2D to higher dimensions
    \cite{zhao2015bearing}.
Based on that, many researches focus on control theory methods 
    \cite{zhao2015translational}
    \cite{schiano2016rigidity} bearing formation controller
    \cite{li2020adaptive}
    \cite{li2021adaptive}
    \cite{li2020bearing}
    \cite{zhao2021finite}
    \cite{zhang2022distributed}
    \cite{zhang2023bearing} for water surface
.

\cite{desai1999control}
\cite{desai2002modeling}
talk about switching formations.


\subsubsection{Switching Roles}

\textcolor{blue}{Why role switching is important}
As talked in Chapter 1, one feature of swarm robotics is that the individuals in the swarm can self-organize their roles.
To predefine the roles of the robots in the formation is not always a good idea, especially when the swarm is large, or we consider the some individual may fail randomly.
One should be able to scatter the swarm randomly and let them configure their position on their own.

Researches on self-assembly satisfy this assumption
    \cite{rubenstein2014programmable}
    \cite{li2019decentralized}
.

But in those approaches, robots moves around to find a vacant spot, which is low efficient.
Researches has been seeking a mathematical model for efficient role assignment.

\textcolor{blue}{centralized assignment}
Centralized assignment can make theoretical best role assignment.
One needs to knows the positions of all the robots and the positions of the whole formation.
After that, we have algorthms to find the best role assignment.
    \cite{rm2020review}
    \cite{mosteo2017optimal} \textcolor{red}{varify}
    %     presents an optimal algorithm for assigning roles and positions to homogeneous robots in freely placeable formations by jointly computing formation translation, rotation, and role assignments to minimize total displacement, with provable correctness and a distributed implementation.
    \cite{macalpine2015scram}
    % introduces SCRAM, a scalable, collision-free role assignment framework that assigns interchangeable robots to target positions to minimize makespan while avoiding collisions, with applications to large-scale multi-robot systems.
    \cite{agarwal2018simultaneous}
    % presents an algorithm that simultaneously optimizes robot assignments and variable goal formation parameters by transforming the problem into a linear sum assignment problem, enabling collision-free trajectories with O(n³) computational complexity.
    \cite{ravichandar2020strata}
    % introduces STRATA, a unified framework for task assignment in large heterogeneous agent teams that accounts for continuous traits, species- and agent-level variability, and ensures tasks' trait requirements are met efficiently.
    \cite{akella2020assignment}
    % presents algorithms for optimally assigning interchangeable robots to variable goal formations by partitioning the formation parameter space into cost-invariant equivalence classes and efficiently solving linear bottleneck assignment problems for each class. (\textcolor{red}{centralized best time assignment})
.

\textcolor{blue}{distributed assignment}
There are also distributed assignment approaches.
In these approaches, each robot carries part of the calculation for the whole assignment algorithm (solving some columns of the matrix).
    \cite{burger2012distributed}
    % introduces a distributed simplex algorithm for solving degenerate linear programs and multi-agent assignment problems over asynchronous peer-to-peer networks, achieving consensus on optimal solutions with linear scalability in communication graph diameter.
    \cite{chopra2017distributed}
    % presents a distributed version of the Hungarian method that enables teams of robots to cooperatively solve assignment and routing problems with spatiotemporal constraints through local computations and communications over a peer-to-peer network.
    \cite{zavlanos2007distributed}
    % presents a distributed formation control framework that combines consensus algorithms, market-based coordination, and artificial potential fields to achieve permutation-invariant formations, ensuring convergence and scalability using only local agent information.
    \cite{alonso2016distributed}
    % proposes a distributed formation control approach that combines consensus and sequential convex optimization to enable teams of robots to navigate and reconfigure safely among static and dynamic obstacles in 2D and 3D environments.
    \cite{michael2008distributed}
    % introduces a distributed, market-based algorithm for dynamic multi-robot task assignment that guarantees efficient convergence to desired task distributions while enabling coordinated formation control, splitting, and merging behaviors.
    \cite{montijano2014efficient}
    % presents a distributed optimization framework for multi-robot formation control that simultaneously solves for optimal positions and role assignments, using averaging and a modified distributed simplex algorithm to guarantee globally optimal configurations.
    \cite{wang2020shape}
    % presents a distributed algorithm for homogeneous robot swarms that concurrently assigns goal locations and plans collision-free paths, enabling fast and reliable shape formation for up to thousands of robots.
.

None of them are scalable : calcualtion grows as the swarm scales up.

\cite{kambayashi2018distributed} follows a unique path to use virtual agents has potential for interchangeable, but can't handle faulty robots.

\subsection{Coverage}

For coverage, usually three classes:
    \cite{wang2011coverage} % a survey
    \cite{galceran2013survey} % survey
1. Make the sensors scatter and cover the area.
They happen in a decentralized way.
    \cite{santos2019decentralized}
    \cite{luo2018adaptive}
    \cite{spanogianopoulos2017fast}
    \cite{siligardi2019robust}
2. wandering exploration
    \cite{huang2019exploration} random
    \cite{ichikawa1999characteristics} random
    \cite{mcguire2019minimal} touch and walk along the wall.

3. textcolor{blue}{sweep to cover, S shape path}
The other is to sweep an area, this usually pre-defines the paths of each robot.
    \cite{almadhoun2019survey} % a survey, talked about sweep coverage
    \cite{avellar2015multi}

\cite{scherer2015autonomous} is a mix, centralized for path planning, after target, decentralized for connection establishment.

\subsection{Path Planning}

\textcolor{red}{path planning, goes to other places, maybe obstacle avoidance, or simply drop them.}
\textcolor{red}{Aryo include them as offline under coverage, but they are not really about coverage}

\textcolor{blue}{navigation, not really about coverage, only from A to B through obstacles}

For navigation : offline centralized way for path planning
    \cite{nazarahari2019multi}
    \cite{thabit2018multi}
    \cite{yu2016optimal}
    \cite{kushleyev2013towards}
.

online, decentralized
    \cite{zhou2020ego}
    \cite{zhou2021ego}
    \cite{zhou2022swarm}
.

\subsection{Light Show}

Light show is centralized
    \cite{waibel2017drone} light show
    \cite{ang2018high} light show

\section{Fault tolerance}

Fault tolerance is a default feature for swarm
    \cite{dorigo2014swarm}
    \cite{hamann2018swarm}
.
\textcolor{red}{cite/explainTha some normal fault tolerance, that is not for centralized, predetermined}

but a swarm is sucepetable to intermittent faults IFs
    \cite{zhou2019review}
    \cite{niu2021distributed}
.

To detect IFs, centralized methods are efficient
    \cite{sheng2021intermittent}
    \cite{zhang2021intermittent}
    \cite{syed2016novel}
.4

To make swarm also tolerant to IFs, hybrid methods are considered
    \textcolor{red}{cite Sinan's paper}
    \cite{ouguz2025proactive}

\section{Manageability} \textcolor{red}{adjust this word}

Human needs to supervise a swarm : interact, steer, reprogram.

However the difficulty is, when scale up, human operator can't communicate with all the robot.

\subsection{Human-swarm interaction}

survey:
\cite{siean2023opportunities}
\cite{kolling2015human}

Human interface side
\cite{jarvis2025first}
\cite{alonso2015gesture} 
\cite{abioye2025user} 100 humans managing drones, not steering

human to the whole swarm
\cite{ayanian2014controlling}
\cite{alonso2015gesture}
\cite{macchini2021personalized}

\cite{podevijn2013gesturing} gesture, subset of the swarm
\cite{zhou2016assistive} joysticks
\cite{lee2013semiautonomous} haptic feedback

\cite{kolling2013human}

\subsection{programming / re-programming}

\textcolor{red}{consider a better place for automatic design, go to swarm program section}
    other challenges include automation design:
    \cite{francesca2016automatic}
    \cite{birattari2019automatic}
    \cite{salman2024automatic}

\subsubsection {offline examples}

mostly offline :

usually, long trial and error: 
\cite{hamann2018swarm} a book
\cite{brambilla2013swarm} a review

automatic design:
 offline automatic design
\cite{francesca2014automode} AutoMoDe
\cite{francesca2016automatic} a survey
\cite{birattari2019automatic} a survey

examples of pre-design
\cite{rubenstein2014programmable} self-assembly
\cite{valentini2016collective} decision making
\cite{werfel2014designing} construction
\cite{dorigo2013swarmanoid} multi task mission

\subsubsection {online but centralized}
\cite{zyrianoff2024over} over the air
\cite{abadie2024robotap}

\textcolor{blue}{light show: too big, only simple commands}
\textcolor{red}{goes to online centralized, make it a paragraph}
\cite{waibel2017drone} light show
\cite{ang2018high} light show

\subsubsection {online, decentralized}
\cite{xie2011design} wireless sensor network
\cite{wang2006reprogramming} wireless sensor network a survey
but slow : consensus
\cite{de2009energy}
\cite{varadharajan2018over} 

\cite{venkata2023kt} introduces the KT-BT framework, using behavior trees and stringBT grammar, enabling multirobot knowledge transfer, with simulations showing improved group performance.

\section{Swarm Autonomy}

From builderbot paper :
What Is Cognitive Robotics?
\cite{cangelosi2022cognition} 
\cite{vernon2014artificial} 
\cite{heinrich2022swarm}
\cite{khaluf2019neglected}

level standard :
\cite{sae2021automated}

collective decision
\textcolor{red}{goes to task section}
\cite{valentini2017best} best of N
\cite{dorigo2014self}
\cite{valentini2016collective}
\cite{shan2020collective}

\textcolor{blue}{from Ayros paper}

\cite{valentini2015efficient} \textcolor{red}{incomplete reference, from Aryo paper 49, belongs to decision making, doesn't belong here}
% 100 kilobots decision making

multi-robot fusion problems are well understood, and existing methods are powerful
centralized
\textcolor{red}{About collective sensing : make it its section in applications}
These are all surveys
\cite{yan2013survey} 
\cite{sun2017multi}
\cite{rizk2019cooperative}
\cite{li2021multi}
\textcolor{red}{consider drop this fusion part}

check colors, many about decision making
\cite{strobel2018managing} \textcolor{red}{ref incomplete}
\cite{ebert2018multi} \textcolor{red}{decision making}
\cite{shan2020collective}
\cite{bartashevich2019benchmarking}
\cite{shan2021discrete} \textcolor{red}{too tech}

another scenario, about the quality of the block
\cite{prasetyo2018best}   best of N
\cite{prasetyo2019collective}  best of N
\cite{khaluf2017edge}  find edge of an area
\cite{wahby2019collective} aggreate to the bright area
\cite{khaluf2020construction} about construction, fusion sensing the density of the building
\cite{capitan2013decentralized} 

not fully de-centralized
\cite{mirzaei2007performance} fixed sensor  \textcolor{red}{a bit tech}
\cite{rodrigues2015beyond} 
\cite{stroupe2001distributed}
\cite{zadorozhny2013information}
\cite{sasaoka2016multi}
\cite{czarnetzki2010handling}
\cite{otte2016collective} This is one a bit different, swarm robot nerval network
\cite{kornienko2005cognitive}
\cite{giusti2012cooperative}

\subsection{behavior tree}

\textcolor{red}{goes to reprogram section}

Behavior tree general
\cite{colledanchise2018behavior}
\cite{iovino2022survey}

BT for multi-robot system
\cite{colledanchise2016advantages}
\cite{jeong2022behavior}

Evolving BT:

genetic programming
\cite{jones2018evolving} \textcolor{red}{evolving}
\cite{kuckling2022automode}

grammatical evolution
\cite{neupane2019learning}
\cite{kuckling2022automode}

Online Evolving BT:

\cite{jones2019onboard}
\cite{venkata2023kt}


\cite{florez2008dynamic} about dynamic BT, but haven't been applied to robots, for games
