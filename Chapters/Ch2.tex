\chapter{Related Work}
\label{ch2}

\section{Introduction}

Swarm intelligence often draws inspiration from the collective behaviors of social animals such as ants, birds, and fish
    \cite{bonabeau1999swarm}.  % the book of Swarm intelligence
Early models like Reynolds' Boids have demonstrated that simple local behavioral rules can give rise to coherent large-scale patterns
    \cite{reynolds1987flocks}.
Swarm intelligence is based on self-organization: the principle that individuals interact with each other locally without centralized control, and complex global behavior emerges.
Swarm robotics aims to leverage the principles of self-organization to construct decentralized robotic systems that are scalable, adaptive, and fault tolerant
    \cite{dorigo2014swarm,       % scholarpedia on swarm robotics
          dorigo2020reflections, 
          dorigo2021swarm}.    %past, present and future
Self-organized systems have shown potential in many applications, including
    environmental monitoring
    \cite{talamali2021less},
    navigation and transportation
    \cite{dorigo2013swarmanoid},
    self-assembly
    \cite{rubenstein2014programmable},
    and collective construction
    \cite{team2012designing, petersen2019review}.
Owing to the absence of centralized control, such systems can scale up to vast numbers of robots without demanding increasing capabilities from each individual robot.

However, despite significant progress, it remains a fundamental scientific question how to design low-level individual robot behaviors that result in desired high-level collective outcomes.
The micro-macro problem refers to this challenge
    \cite{dorigo2021swarm}.
Although various theoretical frameworks have been proposed
    \cite{hamann2008framework,
          hamann2010space,
          hamann2018swarm,
          hamann2013towards}
and decentralized controllers have been demonstrated for specific tasks
    \cite{dorigo2004evolving,
          nouyan2009teamwork,
          dorigo2013swarmanoid,
          rubenstein2014programmable,
          li2019decentralized},
yet purely decentralized swarms are often difficult to design and manage, and may take a long time to complete the desired task%, especially for tasks that require global information, such as environmental analysis and decision making.
    \cite{dorigo2021swarm,
          kengyel2015potential}. % pure flat is not good

These limitations motivate research into hybrid and hierarchical approaches that integrate selective forms of centralization into decentralized systems.
These systems seek to bypass the micro-macro problem,
                      ease the design of swarm behaviors and coordination strategies,
                      and improve the manageability of a decentralized system,
while preserving scalability and fault tolerance features.

This chapter reviews prior work in hybrid and hierarchical swarm control,
compares the existing literature on decentralized and centralized approaches in different tasks,
and discusses aspects of swarm manageability, including human-swarm interaction, swarm re-programming, and swarm autonomy.

\section{Heterogeneity and leadership}

Flat and homogeneous swarms, often inspired by examples from nature,
    \cite{buhl2006disorder,
          detrain2008collective,
          theraulaz1998origin},
assume that all agents are identical, follow the same behavioral rules, and interact with each other in the same way
    \cite{viragh2014flocking,
          vasarhelyi2018optimized,
          floreano2008bio,
          csahin2004swarm,
          beni1988concept}.


In contrast, heterogeneous swarms consist of individuals serving different roles or having different behaviors.
These systems,
with individuals in different physical types or with different information,
can outperform purely homogeneous systems, depending on the task requirements
    \cite{kengyel2015potential}. % hetero is better
For example, individuals with more information can influence the behavior of others during tasks such as flocking or collective decision-making
    \cite{firat2020self, % use informal individual to cue the whole swarm to aggreate to a desired shelter
          prasetyo2018best,  % some stubborn individuals
          balazs2020adaptive}.

In some existing approaches, a subset of individuals are assigned explicit leadership roles
    \cite{gu2009leader,
          amraii2014explicit}.
The leaders are usually pre-defined and fixed after deployment, while the rest of the coordination remains self-organized and based on local interactions. 
Each follower is assigned a leader and receives information from it, such as position references.
In this way, information flows in a more structured manner among robots, thereby easing the design of the behaviors and reducing task completion times.
Many tasks can benefit from this, for example flocking
    \cite{dalmao2011cucker,
          jia2019modelling,
          pignotti2018flocking}.
%\textcolor{red}{consider drop:}
%These leaders may be vulnerable to adversarial detection, a concern explored in    \cite{zheng2020adversarial}.
Some of these leader--follower approaches are organized into two-level systems.
Some other approaches impose a multi-level topological organization, in which the hierarchy is usually formed by fixed leader--follower pairs.

%In some works of self-assembly or formation
%    \cite{desai1999control,
%          li2019decentralized,
%          rubenstein2014programmable},
%although leaders and followers are not explicitly pre-assigned, each robot implicitly selects a leader upon joining the system and adjusts its position based on that leader’s state, effectively forming a leader–follower hierarchy.

%and thereby improving efficiency in tasks
%Such approaches can be regarded as highly heterogeneous systems in which each individual assumes its own role.

However,
because the topology is usually predefined, these systems suffer from low scalability and single points of failure.
No existing approach has provided a general mechanism for self-organizing hierarchy—that is, adaptively forming and maintaining hierarchical structures without predefined topologies or fixed leadership roles.
%As task complexity increases, self-organizing hierarchy becomes increasingly relevant. However, 
Constructing hierarchy in a self-organized manner remains a major challenge
    \cite{dorigo2020reflections}. 

\section{Self-organized versus centralized or predetermined control}

Purely self-organized swarms are highly scalable and fault tolerant,
but their behaviors are often difficult to design,
and they often exhibit inferior performance.
In contrast, systems with centralized or predetermined control are easier to design and can complete tasks more quickly and with higher accuracy by exploiting global information, but they lack scalability and fault tolerance.
This section reviews these different control approaches and their advantages and disadvantages across several key example tasks in multi-robot systems.

\subsection{Multi-robot coverage}

Coverage addresses a key aspect of multi-robot coordination for many applications.
The goal is to ensure that a target region is fully observed, sensed, or explored by a group of agents.
Effective coverage is thus essential for applications such as environmental monitoring, search and rescue, precision agriculture, and surveillance.
There are two types of coverage problems: area coverage (maintaining spatial distribution over a region) and sweep coverage (systematically traversing a region to ensure full visitation).
A wide range of approaches to these coverage problems have been developed, and most can be classified into three main categories: sensor dispersion for area coverage, random exploration and predetermined sweeps for sweep coverage
    \cite{wang2011coverage,     % survey
          galceran2013survey}.  


\textbf{Sensor Dispersion.~~}
One approach is to spread sensors over the environment to cover the target area,
typically by driving robots to well-separated positions that remain largely static once coverage is achieved,
with a few notable exceptions in dynamic environments
    \cite{santos2019decentralized}.
In these approaches, although sensor localization may rely on GPS or distance-based measurements, the interactions among the robots are typically decentralized.
Robots use mechanisms such as potential fields, local repulsion, or other adaptive local policies to maintain appropriate separation and avoid sensing overlap.
Such methods can offer scalability and robustness due to their decentralized interactions
    \cite{santos2019decentralized,
          luo2018adaptive,
          spanogianopoulos2017fast,
          siligardi2019robust}.
However, these methods may require a large number of robots to achieve coverage, particularly in large-scale environments.

\textbf{Random Exploration.~~}
In another type of approach, robots sweep the environment in parallel.
In these approaches, robots move mostly randomly and often also perform simple reactive collision avoidance.
The motion control in these approaches is fully decentralized, but sometimes centralized coordination is used for later steps, such as fusing collected information.
Since the sweep motion control is fully decentralized, it is easy to implement and requires only minimal communication and sensing information during operation, but often suffers from long completion times and inefficient coverage, due to redundancy and non-uniform exploration
    \cite{huang2019exploration,
          ichikawa1999characteristics,
          mcguire2019minimal}.

\textbf{Predetermined Sweeps.~~}
Sweep-based methods pre-calculate systematic paths, usually S-shaped trajectories, to ensure full coverage of a known region.
These strategies often require explicit predetermined roles and global map knowledge, which yields fast and predictable coverage, but usually lacks adaptability and can suffer from communication bottlenecks and single points of failure
\cite{almadhoun2019survey, avellar2015multi}.
There are also mixed approach variants,
for example, \cite{scherer2015autonomous} uses fully predetermined motion trajectories,
but the robots stream data using a relay network formed in a self-organized manner.

To summarize, while random exploration is robust and scalable, it often suffers from long completion times and inefficient coverage.
Conversely, predetermined and/or centrally coordinated methods can achieve fast and uniform coverage, but usually lack adaptability and fault tolerance.

\subsection{Path Planning}

For many multi-robot system applications, path planning is a key component.
Existing path planners include centralized and/or predetermined offline approaches as well as self-organized online approaches.

\textbf{Centralized Offline Planning}
Centralized planners assume global map knowledge and compute predetermined collision-free trajectories for all robots before deployment.
Multi-robot path optimization usually employs heuristics such as genetic algorithms or particle swarm optimization 
    \cite{nazarahari2019multi,
          thabit2018multi,
          yu2016optimal,
          kushleyev2013towards}.
These methods can produce high-quality paths, but they scale poorly with the number of robots and lack robustness to unexpected robot failures during execution.

\textbf{Distributed Online Planning}
There are a few existing examples of distributed online planners, such as EGO-Swarm
    \cite{zhou2020ego,
          zhou2021ego,
          zhou2022swarm}.
The EGO-Swarm planner uses local sensing and onboard optimization. Fast distributed trajectory generation is achieved by predicting nearby robots' short-horizon behavior and optimizing smooth, dynamically feasible paths at high frequency.
This approach enables agile navigation in cluttered environments, but the generated solutions are not perfectly optimal, in contrast to what could in principle be achieved using offline global optimization.  
Although the trajectory generation is decentralized, the start and goal positions of each robot are predetermined, which limits scalability and reduces robustness to agent failures.

In summary, centralized planners can use global information to generate high-quality paths but scale poorly and lack robustness to failures.
Also, existing distributed planners still fall short of providing fully scalable and fault-tolerant mechanisms for both defining and reaching goal positions.


\subsection{Formation control}

Formation control is another key topic in multi-robot systems
    \cite{liu2018survey,
          oh2015survey}. % surveys
Given the emphasis of swarm robotics on scalability and fault tolerance, this chapter focuses on decentralized formation control, which is also the focus of the majority of existing work from the control theory literature.
In addition, multi-robot formations are examined from the perspective of role assignment.

Decentralized formation controllers use relative positioning, for example based on distance sensors, ultraviolet markers, or fiducial markers
    \cite{walter2018fast, %localization of UaVs using ultraviolet 
          ulrich2022towards}.  %fast fiducial marker with full 6 dof pose estimation
These approaches typically focus on achieving stable rigid-body formations using distance-based constraints
    \cite{yang2018growing,
          mehdifar2018finite,
          stacey2015passivity,
          anderson2018rigid,
          oh2011adistance,
          oh2011bdistance},
%which maintains formation using distance measurements,
or bearing-based constraints
%which allows formation maintenance using only bearing measurements, with two designated leaders controlling scale, rotation, and translation while other robots maintain relative bearings
    \cite{zhao2019bearing,
          zhao2015bearing,
          zhao2015translational,
          schiano2016rigidity,
          li2020adaptive}.
%          li2021adaptive,
%          li2020bearing,
%          zhao2021finite,
%          zhang2022distributed,
%          zhang2023bearing}.            % for water surface
%However, these methods assume that each robot is preassigned a fixed role in the formation.
However, these methods assume that each robot is pre-assigned a fixed role in the formation.
During deployment,
the interaction topology among the robots is predefined and fixed, and robots move by referencing the positions of the robots they are connected to in this predefined topology.
They offer little support for dynamic or fault-tolerant operation, emphasizing control-theoretic guarantees over flexibility or adaptivity.


On the other hand, a key feature of swarm robotics is the ability to deploy robots without pre-assigned roles.
Moreover, for fault tolerance, the swarm should be able to reassign roles in response to unexpected failures.
These considerations highlight the importance of role assignment.
In formation tasks, efficient role assignment can reduce unnecessary crossing movements among robots and thereby speed up formation convergence.
However, in large swarms or environments prone to agent failures, pre-assigning roles is often impractical.
%Some self-assembly approaches allow robots to autonomously select vacant positions, but these methods are often inefficient due to random search
%    \cite{rubenstein2014programmable,
%          li2019decentralized}.

Centralized algorithms can compute globally optimal role assignments by leveraging full knowledge of robot positions and formation targets
    \cite{rm2020review,
          macalpine2015scram,
          agarwal2018simultaneous,
          ravichandar2020strata,
          akella2020assignment},
ensuring minimal total displacement and/or collision-free paths.

Distributed assignment strategies instead divide computational responsibilities among robots,
using, e.g., consensus mechanisms, market-based coordination, or distributed optimization algorithms
    \cite{mosteo2017optimal,
          burger2012distributed,
          chopra2017distributed,
          zavlanos2007distributed,
          alonso2016distributed,
          michael2008distributed,
          montijano2014efficient,
          wang2020shape}.
In many such approaches, although computation is distributed, the per-robot computational and memory burden still typically grows with system size.
For example, some methods require each robot to solve a portion of a global optimization problem, such that the calculation dimensionality of each portion still increases with the number of agents.
Other methods rely, for example, on bidding or consensus schemes in which each robot must maintain information about all robots in the system.
That is why, despite some aspects of these approaches being distributed, they still struggle to scale efficiently with system size.
Distributed approaches that are specifically designed to handle faulty or interchangeable agents also remain limited
    \cite{kambayashi2018distributed}.

In addition, switching formations have been explored: that is, a series of target formations are predefined and the system is triggered to switch between them 
    \cite{desai1999control,
          desai2002modeling}.
However, these methods still typically rely on centralized commands to trigger the formation switches, rather than the system adapting autonomously.

In summary, while centralized or predetermined approaches can ease the implementation of optimization algorithms,
they lack robustness and scalability.

\subsection{Light Show}

Drone light shows are becoming popular in urban life.
Existing shows use a fully centralized and predetermined approach to robot coordination,
where global trajectories are precomputed offline and any commands that are issued during the show are broadcast to all agents from a ground station.
This can provide precise synchronization and visually coherent formations with very many robots
    \cite{waibel2017drone,
          ang2018high}.
However, the resulting systems are vulnerable to a single point (ground station) of failure and to communication failures and bottlenecks,
lacking the scalability or adaptability typically associated with decentralized systems.

\section{Behavior design and manageability}

A fully autonomous multi-robot system should be able to make decisions and adapt to a changing environment without requiring human intervention in order to accomplish its goals. 
At the same time, these systems must remain manageable: human operators should still have the option to supervise the system and optionally intervene if they desire.
This goal is particularly challenging for swarm systems, since robots self-organize using only local information, which often leads to long convergence times, makes collective behaviors difficult to design, and makes it difficult for human commands to propagate through the swarm.
This section reviews the state of the art in swarm autonomy and manageability.

\subsection{Swarm autonomy}

Autonomy is often regarded as a central long-term goal of robotic systems—the ability for robots to operate without requiring human supervision or intervention.

Despite significant advances in individual robot autonomy
    \cite{cangelosi2022cognition,
          vernon2014artificial},
achieving autonomy in swarm robotics remains an open challenge.
Existing swarm research primarily focuses on a single aspect of autonomy such as action, perception, and adaptation
    \cite{heinrich2022swarm},
but other cognitive capabilities such as learning or anticipation are still largely absent,
and it lacks an overall framework to unify all the attributes.


%Collective decision making, as a classical task for swarm systems for environmental analysis, poses an important role in swarm autonomy.
%Usually, autonomy in robotic systems implies the requirement of the ability to make decisions without human intervention.
Generally, autonomy in robotic systems refers to the ability to make decisions without human intervention.
In swarm systems, such decisions must emerge from the behavior of individuals with only local information.
That is why collective decision making is one of the most-studied aspects of swarm autonomy.


The most-studied collective decision-making problem in swarm robotics is the best-of-N problem, of which the binary (best-of-two) case is a special instance
    \cite{khaluf2019neglected, % decision making
          valentini2017best,   % best of N
          dorigo2014self,
          valentini2016collective,
          shan2020collective,
          ebert2018multi,
          prasetyo2018best,       % best of N
          prasetyo2019collective, % best of N
          valentini2015efficient}, % best of N
These works demonstrate how simple local rules allow groups to reach agreement without centralized control.
Some works investigate specific aspects such as convergence speed, decision accuracy, and robustness under noise or malicious information
    \cite{shan2020collective,            % bayesian
          bartashevich2019benchmarking,  % 
          shan2021discrete, %\textcolor{red}{too tech}
          strobel2018managing}. %\textcolor{red}{ref incomplete} % byzenting


Despite this progress, decision making in swarms generally remains an isolated task rather than being integrated into a broader understanding of the mission context.
Many elements are missing in terms of autonomy
such as the swarm needs to detect that a new collective decision is required, or to autonomously adapt the decision-making process to conditions that were not explicitly studied beforehand
    \cite{khaluf2019neglected}.
Current frameworks provide no unified mechanisms for learning new behaviors online, anticipating future events, or autonomously restructuring task workflows.


The SAE J3016 standard
    \cite{sae2021automated}
defines autonomous level for robot systems.
Existing robot swarm systems would be barely classified as having collective behaviors equivalent to the individual behaviors described in {\it Level 2 (Partial automation)},
as existing robot swarm systems are able to
execute intended tasks, but a human operator is still required to continuously monitor the environment and system and decide when to intervene to correct or stop the system.
That would be a challenge for swarm systems as will be talked about in the following section.

\subsection{Human-swarm interaction}


Additionally, although robot swarms are generally capable of executing their intended tasks,
human operators still play a crucial role in supervising the system,
issuing high-level commands, and reprogramming the swarm when necessary.
However, as swarm size increases, direct communication with all the robots in the swarm becomes infeasible.
This section reviews existing research on human–swarm interaction and swarm online programming.


Current research mainly focuses on two aspects:
the design of human–swarm interfaces,
and the mechanisms through which human commands sent to a small subset of robots can influence the collective behavior of the swarm
    \cite{siean2023opportunities,
          kolling2015human}.


Some existing research explores interfaces to manipulate the swarm, such as gesture control
    \cite{alonso2015gesture,
          podevijn2013gesturing,
          macchini2021personalized}, % leap motion, control master drone. Other drones follow by Renolds
haptic devices
    \cite{lee2013semiautonomous},
or wearable systems
    \cite{jarvis2025first}.
Among these works,
operators usually broadcast commands to all robots directly
    \cite{ayanian2014controlling,
          abioye2025user}
which can lead to scalability issues due to the communication burden as the swarm size grows.


For decentralized methods where humans communicate with only one or a few robots, research focuses on how a subset of the swarm can influence the entire swarm
    \cite{podevijn2013gesturing,  % gesture, subset of the swarm
          zhou2016assistive,      %     joysticks
          lee2013semiautonomous,  % haptic feedback
          kolling2013human}.      % select (human select some robots) and beacon (human put a beacon and influence nearby robots) 
While these methods improve scalability,
they often suffer from slow convergence to the desired swarm-wide behavior
and also make the design of swarm-wide behaviors challenging.

In general, there is a lack of a framework that simultaneously addresses scalability, convergence efficiency, and ease of behavior design in human–swarm interaction.

\subsection{programming / re-programming}


When supervising the swarm, one important aspect of human–swarm interaction is the ability to re-program the swarm easily and online.
However, it faces two problems.
Firstly, designing controllers for swarm systems is often a lengthy trial-and-error process
    \cite{hamann2018swarm,
          brambilla2013swarm}.  
It becomes more challenging when swarms must adapt to new tasks or environments.
Secondly, delivering new programs to all robots often requires the operator to communicate with the entire swarm, which can lead to scalability issues.

This section reviews approaches from offline, centralized online, and to decentralized online reprogramming.


Classically,
programs are pre-designed and downloaded offline to all the robots in a swarm before deployment, for example,
    \cite{rubenstein2014programmable,
          valentini2016collective,
          werfel2014designing,
          dorigo2013swarmanoid}.
However, offline methods cannot react to unexpected situations such as task changes during deployment.
%Notably, automatic design methods falls in this category
%    \cite{francesca2014automode,
%          francesca2016automatic,
%          birattari2019automatic,
%          salman2024automatic}.
%They demonstrate how optimization and evolutionary algorithms can produce robust swarm behaviors without manual tuning.
%Yet the programs are pre-generated.
Automatic design methods demonstrate how approaches such as optimization and evolutionary algorithms can produce robust swarm behaviors without manual tuning
    \cite{francesca2014automode,
          francesca2016automatic,
          birattari2019automatic,
          salman2024automatic}.
%They usually fall in this category, as they typically require long trial-and-error optimization before deployment.
These methods typically rely on extensive trial-and-error processes prior to deployment and are therefore most often applied in an offline manner.


Even once behavior design is complete through extensive optimization, delivering and updating these programs across all robots remains challenging.

Centralized online reprogramming techniques update robot behaviors after deployment through a global communication channel.
Over-the-air programming frameworks
    \cite{zyrianoff2024over,
          abadie2024robotap}
enable operators to push new software to all robots.
Large-scale drone light shows
    \cite{waibel2017drone,
          ang2018high}
represent a widely deployed example of centralized online control.
Although they can coordinate thousands of robots,
they typically allow only predefined trajectories and very simple commands such as taking off or emergency landing.

Decentralized reprogramming methods update robot controllers without relying on a central broadcaster.
Some studies in wireless sensor networks
    \cite{xie2011design,
          wang2006reprogramming}
demonstrated the feasibility of distributed software updates.
However, studies in over-the-air program updating for robot swarms also highlight fundamental limitations such as slow propagation resulting from consensus mechanisms
    \cite{de2009energy,
          varadharajan2018over}.
        
While many studies originate from wireless sensor networks, they provide foundational insights for decentralized reprogramming in robotic swarms: distributed program updating often propagates slowly and it remains challenging to coordinate consistent behavior across all the robots.


To summarize, a unified framework that enables a human operator to reprogram swarm behavior in a scalable and efficient manner remains largely unexplored.

%\textcolor{red}{Behavior tree, move later}
\iffalse

\subsubsection{behavior tree}

Behavior trees (BT) have been increasingly explored as a tool for structuring robot behaviors in a modular and hierarchical manner.
The compositional structure makes it possible to modify or update parts of a controller without redesigning the entire behavior,
which is particularly appealing in the context of online robot reprogramming.

Behavior trees (BTs) provide a modular and hierarchical framework for representing robot behaviors
    \cite{colledanchise2018behavior,
          iovino2022survey},
which have been adopted in several multi-robot applications
    \cite{colledanchise2016advantages,
          jeong2022behavior}.

Due to their compositionality and readability, BTs offer a potential solution to key challenges in swarm autonomy and human–swarm interaction. 
They allow robots to adapt their behavior online, enable hierarchical organization of complex tasks, and facilitate interaction with human operators through modular control structures. 

Evolving BTs via evolutionary algorithms—such as genetic programming
    \cite{jones2018evolving,
          kuckling2022automode}
or grammatical evolution
    \cite{neupane2019learning,
          kuckling2022automode}—
enables automated creation of swarm behaviors.
Recent advances support online evolution as well
    \cite{jones2019onboard,
          venkata2023kt},
allowing robots to adapt their BT controllers during deployment.

Dynamic BTs have also been investigated in game AI
    \cite{florez2008dynamic},
though their application to real-world robotics remains to be explored.

More recent work exploit behavior trees for knowledge transfer in multi-robot systems as a adaptability mechanism.
For instance, Venkata et al.\ \cite{venkata2023kt} introduce the KT-BT framework, which uses behavior trees and a grammar-based encoding (stringBT) to achieve decentralized transfer of acquired skills, improving collective performance during online adaptation.

\fi

\section{Fault tolerance}

Swarm systems show some degree of inherent fault tolerance because of their redundancy and absence of single points of failure
    \cite{dorigo2014swarm,
          hamann2018swarm}.
Most commonly-studied swarm tasks, such as coverage or collective decision making, can tolerate a subset of robots failing.
Robots simply continue to operate as normal, without knowing or caring about failed peers
    \cite{bjerknes2013fault,
          khadidos2015exogenous}.

Despite these advantages, fault tolerance in robot swarms remains far from a solved problem.
Existing approaches often struggle with limitations such as fault detection and diagnosis under partial observability, communication and sensing failures, malicious or Byzantine agents,
and performance degradation caused by cascading or correlated failures
    \cite{hamann2018swarm}.

Among these challenges, intermittent fault (IF) is an example that poses a particularly difficult and representative case.
An IF is a failure that occurs sporadically. 
It can appear, disappear, and reappear.
Its sporadic nature makes reliable detection difficult,
and repeated occurrences can significantly degrade overall swarm performance
    \cite{zhou2019review,
          niu2021distributed}.
Centralized methods are highly effective for identifying IFs, since they have access to system-wide data
    \cite{sheng2021intermittent,
          zhang2021intermittent,
          syed2016novel},
but as discussed in the previous section, centralized methods are vulnerable to single point failures and not scalable.

Overall, although decentralized swarm systems benefit from some degree of inherent fault tolerance,
addressing certain fault types still relies on centralized mechanisms,
thereby undermining scalability and resilience.

\section{Summary}

Yet purely decentralized swarms are often difficult to design and manage, and may take a long time to complete the task, especially in tasks that require global information, such as environmental analysis and decision making.


In summary, centralized approaches offer an easy solution for designing and managing collective behavior and can often coordinate the robots to finish the task quickly,
while decentralized approaches can provide greater robustness and scalability.


However, neither paradigm alone fully addresses the need for coordinating large swarms to perform dynamic tasks that require both adaptability and performance guarantees.
A self-organizing hierarchical framework is a strong candidate to bridge this gap.
The goal of this Thesis is to show the design of such a self-organizing hierarchical system that
enables swarms to dynamically reconfigure their coordination structure,
                   balance local autonomy with global oversight,
               and achieve human-scalable manageability without sacrificing resilience.


\iffalse
\section{Abandon}

\textcolor{blue}{from Ayros paper}

multi-robot fusion problems are well understood, and existing methods are powerful
centralized
\textcolor{red}{About collective sensing : make it its section in applications}
These are all surveys
\cite{yan2013survey} 
\cite{sun2017multi}
\cite{rizk2019cooperative}
\cite{li2021multi}
\textcolor{red}{consider drop this sensor fusion part}


another scenario, about the quality of the block
\cite{khaluf2017edge}  find edge of an area
\cite{wahby2019collective} aggreate to the bright area
\cite{khaluf2020construction} about construction, fusion sensing the density of the building
\cite{capitan2013decentralized} 

not fully de-centralized
\cite{mirzaei2007performance} fixed sensor  \textcolor{red}{a bit tech}
\cite{rodrigues2015beyond} 
\cite{stroupe2001distributed}
\cite{zadorozhny2013information}
\cite{sasaoka2016multi}
\cite{czarnetzki2010handling}
\cite{otte2016collective} This is one a bit different, swarm robot nerval network
\cite{kornienko2005cognitive}
\cite{giusti2012cooperative}
\fi
