t \chapter{Related Work}
\label{ch2}

% skeleton
% 1 Hybrid, hierarchical, and leadership approaches in the literature   TODO from Sci-Ro paper, yuwei paper
% 2 Swarm vs. centralized approaches for different tasks (explain what is possible to do with both, for those tasks -- current limitations are implied)
%   1.1 formation control
%   1.2 exploration, search and rescue    TODO from Sci-Ro paper
%   1.3 drone entertainment shows...
%   1.4 (more?)
% 3 The process of swarm programming / re-programming (mostly about how this is done offline, don't worry about the limitations when trying to do this online)
% 4 Swarm autonomy    TODO from Builderbot paper and zotero

\section{Introduction}

\textcolor{blue}{what is swarm, swarm is good}

Swarm robotics draws inspiration from collective behaviors in social animals such as ants, birds, and fish
    \cite{bonabeau1999swarm}.  % the book of Swarm intelligence
Early models like Reynolds' Boids demonstrated that simple local behavioral rules can give rise to coherent large-scale patterns
    \cite{reynolds1987flocks}.
Swarm Robotics aims to leverage these principles to construct decentralized robotic systems that are scalable, adaptive, and fault tolerant
    \cite{dorigo2014swarm,    % scholarpedia on swarm robotics
          dorigo2020reflections,
          dorigo2021swarm}.    %past, present and future
These decentralized systems have shown advantages in large-scale or spatially distributed tasks, including
    environmental monitoring
    \cite{talamali2021less},
    navigation and transportation
    \cite{dorigo2013swarmanoid},
    self-assembly
    \cite{rubenstein2014programmable},
    and collective construction
    \cite{team2012designing, petersen2019review}.

\textcolor{blue}{but, macro, micro problem, and swarm is not always efficient}

However, despite significant progress, it remains a fundamental scientific question to connect low-level individual robot behaviors to the desired high-level collective outcomes, known as micro-marco problem
    \cite{dorigo2021swarm}.
Despite that various theoretical frameworks have been proposed
    \cite{hamann2008framework,
          hamann2010space,
          hamann2018swarm,
          hamann2013towards}
and decentralized controllers have been demonstrated for specific tasks
    \cite{dorigo2004evolving,
          nouyan2009teamwork,
          dorigo2013swarmanoid,
          rubenstein2014programmable,
          li2019decentralized},
yet purely decentralized swarms often face fundamental inefficiencies in environmental analyzing and decision making, particularly due to their reliance on strictly local information
    \cite{dorigo2021swarm,
          kengyel2015potential}. % pure flat is not good

\textcolor{blue}{Therefore, hierarchical}

These limitations motivate research into hybrid and hierarchical approaches that integrate selective forms of centralization into decentralized systems,
These systems seek to bypass the micro-macro problem and increase the global efficiency of a decentralized system and, in the meantime, keep the scalability and fault tolerance features.

\textcolor{blue}{In this chapter, we review}

This chapter reviews prior work in hybrid and hierarchical swarm control,
compares decentralized and centralized approaches across key swarm robotics tasks,
and examines related developments in formation control, coverage, manageability, re-programming, and swarm autonomy.

\section{Heterogeneous, Hybrid, and hierarchical Approaches}

\textcolor{blue}{flat swarms}

Flat and homogeneous swarms, as their classic nature inspiration
    \cite{buhl2006disorder,
          detrain2008collective,
          theraulaz1998origin},
assume that all agents are identical and follow the same behavioral rules
    \cite{viragh2014flocking,
          vasarhelyi2018optimized,
          floreano2008bio,
          csahin2004swarm,
          beni1988concept}.
Such systems excel in robustness and scalability but struggle with tasks that require differentiated roles, long-range coordination, or efficient global organization.

\textcolor{blue}{heterogeneous}

Opposite to that, heterogeneous swarms consist of individuals serving different roles.
These systems,
with individuals in different physical types or with different information,
can often outperform purely homogeneous systems
    \cite{kengyel2015potential}. % hetero is better

Informed or persistent agents can influence the behavior of others during tasks such as flocking or collective decision-making
    \cite{firat2020self, % use informal individual to cue the whole swarm to aggreate to a desired shelter
          valentini2016collective,
          balazs2020adaptive}.
\textcolor{red}{varify: valentini2016collective is not a good choice, find another decision making with stubborn individuals}
Some works explicitly name these individuals as leaders
    \cite{gu2009leader,
          amraii2014explicit}.
\textcolor{red}{consider drop:}
These leaders may be vulnerable to adversarial detection, a concern explored in
    \cite{zheng2020adversarial}.

\textcolor{blue}{hierarchical}

On the other hand, hierarchical approaches impose an additional topological organization on the swarm,
allowing information to flow in a more structured manner
and thereby improving efficiency in tasks
such as flocking
    \cite{dalmao2011cucker,
          jia2019modelling,
          pignotti2018flocking},
and self-assembly or formation
    \cite{li2019decentralized,
          divband2019photomorphogenesis}.
Such approaches can be regarded as highly heterogeneous systems in which each individual assumes its own role.

However,
maintaining scalability and avoiding single points of failure in these systems often require sophisticated mechanisms,
such as temporary leadership, explicit role assignments, or distributed authority-transfer schemes.
In other words, constructing hierarchy in a truly self-organized manner remains a major challenge
    \cite{dorigo2020reflections}.
As task complexity increases, researchers increasingly recognize the importance of self-organized hierarchy, a capability still largely absent in classical decentralized swarms.

\section{Swarm vs. centralized approaches for different tasks}

\textcolor{blue}{The point is}
\textcolor{blue}{1 we want to show there is a gap :  self-organized hierarchy}
\textcolor{blue}{2 both sides have some downsides}

Purely decentralized swarms are highly scalable and fault tolerant but often lack efficiency, especially when global information is required.
Centralized systems, by contrast, achieve high precision and optimal coordination but lose scalability and fault tolerance.
This section reviews these trade-offs across major tasks in multi-robot systems.

\subsection{Formation}

Formation control is one of the most fundamental research topics in multi-robot and swarm systems
    \cite{liu2018survey,
          oh2015survey}. % surveys
Most decentralized formation controllers rely on relative localization, using distance sensors, ultraviolet markers, or fiducial markers
    \cite{walter2018fast, %localization of UaVs using ultraviolet 
          ulrich2022towards}.  %fast fiducial marker with full 6 dof pose estimation

\subsubsection{Fixed-Role Approaches}

Classical methods assume that each robot is preassigned a fixed target position.
During deployment,
robots move according to the positions of predefined neighbors,
making motion control decentralized
but role specification effectively centralized.
These approaches achieve stable rigid-body formations using distance- or bearing-based constraints
    \cite{yang2018growing,
          mehdifar2018finite,
          stacey2015passivity}.
\textcolor{red}{distance rigidity.}
Bearing rigidity theory
    \cite{zhao2019bearing,
          zhao2015bearing}
further allows formation maintenance using only bearing measurements, with two designated leaders controlling scale, rotation, and translation while other robots maintain relative bearings
    \cite{zhao2015translational,
          schiano2016rigidity,
          li2020adaptive,
          li2021adaptive,
          li2020bearing,
          zhao2021finite,
          zhang2022distributed,
          zhang2023bearing}. % for water surface
Switching formations under rigid-body assumptions has also been explored
    \cite{desai1999control,
          desai2002modeling}.

Overall, these methods emphasize control-theoretic guarantees over flexibility or adaptivity, assuming fixed roles and full knowledge of the desired formation.

\subsubsection{Switching Roles}

\textcolor{blue}{Why role switching is important}

One required feature of swarm robotics is self-organized role assignment.
An efficient role assignment can reduce individual robot movements and thereby accelerate formation.
However, in large swarms or environments prone to agent failures, preassigning roles is often impractical.
Some self-assembly approaches allow robots to autonomously select vacant positions, but these methods are often inefficient due to random search
    \cite{rubenstein2014programmable,
          li2019decentralized}.

Centralized algorithms, by contrast, can compute globally optimal role assignments by leveraging full knowledge of robot and formation positions
    \cite{rm2020review,
          macalpine2015scram,
          agarwal2018simultaneous,
          ravichandar2020strata,
          akella2020assignment},
ensuring minimal total displacement or collision-free paths.

\textcolor{blue}{distributed assignment}

Distributed assignment strategies attempt to reduce centralization by dividing computational responsibilities among robots,
using consensus, market-based coordination, or distributed optimization to approximate optimal assignments
    \cite{mosteo2017optimal,
          burger2012distributed,
          chopra2017distributed,
          zavlanos2007distributed,
          alonso2016distributed,
          michael2008distributed,
          montijano2014efficient,
          wang2020shape}.
Despite these efforts, current distributed methods still struggle to scale efficiently with swarm size, as the calculation load of each individual scales up with the swarm scale.
Approaches handling faulty or interchangeable agents also remain limited
    \cite{kambayashi2018distributed}.

These observations highlight a critical gap in terms of role assignment in swarm formation:
while centralized approaches provide optimal coordination
and decentralized swarms provide robustness and scalability,
self-organized hierarchy for role allocation remain largely underexplored.
Developing such mechanisms is essential for achieving both efficiency and resilience in large-scale, formation tasks.

\subsection{Coverage}

\textcolor{blue}{what is coverage, why it is important}

Coverage is another classic task in multi-robot systems.
The goal is to ensure that a target region is fully observed, sensed, or explored by a group of agents.
Effective coverage is essential for applications such as environmental monitoring, search and rescue, precision agriculture, and surveillance.
A wide range of strategies has been developed, and most can be classified into three main categories
    \cite{wang2011coverage,     % survey
          galceran2013survey}.  % survey

\textbf{Decentralized Sensor Dispersion}
In these approaches, robots rely solely on local information to spread out and maximize area coverage.
The robots typically use potential fields, repulsion mechanisms, or adaptive local policies to maintain appropriate separation and avoid sensing overlap.
Such methods are inherently decentralized and offer strong scalability and robustness
    \cite{santos2019decentralized,
          luo2018adaptive,
          spanogianopoulos2017fast,
          siligardi2019robust}.

\textbf{Wandering or Random Exploration}
Another family of approaches relies on random motions to gradually explore the environment.
Robots may move randomly, rely on simple reactive collision responses.
These strategies are also decentralized.
They are easy to implement and require only minimal information but often suffer from slow convergence and redundant traversal
    \cite{huang2019exploration,
          ichikawa1999characteristics,
          mcguire2019minimal}.

\textbf{Predefined Sweep Trajectory-Based Coverage}
Sweep-based methods pre-calculate systematic paths, usually S-shaped trajectories, to ensure full coverage of a known region.
These strategies often require predefined roles, global map knowledge, or explicit task allocation, which introduces centralized components but yields efficient and predictable coverage
\cite{almadhoun2019survey, avellar2015multi}.

There are also mixed approaches. For example, \cite{scherer2015autonomous} employs centralized predefined sweep trajectories to generate coverage paths and then relies on decentralized mechanisms to establish communication links for environmental information collection.

To summarize, 
coverage tasks clearly show the dichotomy between decentralization and centralization.
While decentralized dispersion and exploration are robust and easy to scale, they often lack efficiency and global optimality.
Conversely, sweep-based and centrally coordinated methods achieve highly efficient coverage but depend on predefined assignments or global knowledge, limiting their adaptability in large, dynamic environments.

\subsection{Navigation Path Planning}

Path planning is a key component of multi-robot and swarm systems, especially in multi-drone systems.
Existing methods ranges from centralized global planners to decentralized online approaches.

\textbf{Centralized Offline Planning}
Centralized planners assume global map knowledge and compute collision-free trajectories for all robots before execution.
These methods provide high-quality or even optimal solutions but scale poorly with swarm size.
Multi-robot path optimization is usually NP-hard problem,
so researches employ heuristics and approximate solutions to generate global trajectories for robots
    \cite{nazarahari2019multi,
          thabit2018multi,
          yu2016optimal,
          kushleyev2013towards}.

\textbf{Decentralized Online Planning}
Decentralized planners operate using local sensing and real-time optimization, offering better scalability and robustness.
The EGO-Swarm family
    \cite{zhou2020ego,
          zhou2021ego,
          zhou2022swarm}
demonstrates fast, distributed trajectory generation by predicting nearby robots' short-horizon behavior and optimizing smooth, dynamically feasible paths at high frequency.
These approaches enable agile navigation in cluttered environments but lack global optimality guarantees and may struggle in tightly constrained spaces.
And similar to formation control, they do not consider task assignments.
In this term, the target position of each robot is pre-defined and fixed.

\subsection{Light Show}

Drone light shows are becoming popular in urban life.
It represent a fully centralized form of swarm coordination,
where global trajectories are precomputed offline and broadcast to all agents.
This guarantees precise synchronization and visually coherent formations at large scale
    \cite{waibel2017drone,
          ang2018high}.
However, such architecture relies heavily on global communication and predetermined paths.
It is vulnerable to a single point (ground station) of failure or communication failures,
lacking the scalability or adaptability typically associated with decentralized systems.

\section{Fault tolerance}

Fault tolerance is an inherent property of swarm system for its redundancy and absence of single points of failure
    \cite{dorigo2014swarm}
    \cite{hamann2018swarm}.

\textcolor{blue}{cite/explain some normal fault tolerance, that is not for centralized, predetermined}
Most classical swarm tasks, such as coverage or collective decision making, can tolerate a subset of robots failes.
Robots simplys continue to operate as normal, without knowing or caring about the failure peers.

\textcolor{blue}{However, IFs} \textcolor{red}{check what Sinan says, maybe rewrite this part}
However, intermittent faults (IFs) poses a unique challenge for swarm systems.
An IF is a failure that occurs for a short period of time, and then recovers.
If makes them difficult to detect and have bad influence to the swarm
    \cite{zhou2019review}
    \cite{niu2021distributed}.

Centralized methods are highly effective for identifying IFs, since they have access to system-wide data.
    \cite{sheng2021intermittent}
    \cite{zhang2021intermittent}
    \cite{syed2016novel}.

\section{Manageability} \textcolor{red}{adjust this word}

As a robot system, despite its autonomy, human operators needs to enforce supervision to the system: to interact, steer, and reprogram the system.
However for a swarm robot system, when scale up, human operator can't communicate with all the robot.
This section gives review in terms of Human-swarm interaction and swarm programming.

\subsection{Human-swarm interaction}

As swarms scale up, humans cannot interact with every robot individually.
Surveys
    \cite{siean2023opportunities,
          kolling2015human}
classify interfaces and strategies for human–swarm interaction.

A class of researches explore interfaces to manipulate the swarm such as gesture control
    \cite{alonso2015gesture,
          podevijn2013gesturing},
haptic devices
    \cite{lee2013semiautonomous},
or wearable systems
    \cite{jarvis2025first}.

Among these works, by default, operators are able to communicate with all the robots in the swarm to send commands collectively
    \cite{ayanian2014controlling,
          macchini2021personalized,
          abioye2025user}
\textcolor{red}{varify},
which will face scalability problem when the swarm scales up.

For decentralized methods where human communicate with only one robot or subset of the swarm, researches forcus on how to make the subset of the swarm influence the whole swarm
    \cite{podevijn2013gesturing, % gesture, subset of the swarm
          zhou2016assistive,  % joysticks
          lee2013semiautonomous}. % haptic feedback
These methods are not very efficient for swarm to converge.

\cite{kolling2013human} \textcolor{red}{verify}

\subsection{programming / re-programming}

usually, long trial and error: 
\cite{hamann2018swarm} a book
\cite{brambilla2013swarm} a review

\subsubsection {offline}

automatic design:
 offline automatic design
\cite{francesca2014automode} AutoMoDe
\cite{francesca2016automatic} a survey
\cite{birattari2019automatic} a survey
\cite{salman2024automatic}

examples of pre-program
\cite{rubenstein2014programmable} self-assembly
\cite{valentini2016collective} decision making
\cite{werfel2014designing} construction
\cite{dorigo2013swarmanoid} multi task mission

\subsubsection {online but centralized}
\cite{zyrianoff2024over} over the air
\cite{abadie2024robotap}

\textcolor{blue}{light show: too big, only simple commands}
\textcolor{red}{goes to online centralized, make it a paragraph}
\cite{waibel2017drone} light show
\cite{ang2018high} light show

\subsubsection {online, decentralized}
\cite{xie2011design} wireless sensor network
\cite{wang2006reprogramming} wireless sensor network a survey
but slow : consensus
\cite{de2009energy}
\cite{varadharajan2018over} 

\cite{venkata2023kt} introduces the KT-BT framework, using behavior trees and stringBT grammar, enabling multirobot knowledge transfer, with simulations showing improved group performance.

\subsubsection{behavior tree}

\textcolor{red}{goes to reprogram section}

Behavior tree general
\cite{colledanchise2018behavior}
\cite{iovino2022survey}

BT for multi-robot system
\cite{colledanchise2016advantages}
\cite{jeong2022behavior}

Evolving BT:

genetic programming
\cite{jones2018evolving} \textcolor{red}{evolving}
\cite{kuckling2022automode}

grammatical evolution
\cite{neupane2019learning}
\cite{kuckling2022automode}

Online Evolving BT:

\cite{jones2019onboard}
\cite{venkata2023kt}


\cite{florez2008dynamic} about dynamic BT, but haven't been applied to robots, for games


\section{Swarm Autonomy}

From builderbot paper :
What Is Cognitive Robotics?
\cite{cangelosi2022cognition} 
\cite{vernon2014artificial} 
\cite{heinrich2022swarm}
\cite{khaluf2019neglected}

level standard :
\cite{sae2021automated}

collective decision
\textcolor{red}{goes to task section}
\cite{valentini2017best} best of N
\cite{dorigo2014self}
\cite{valentini2016collective}
\cite{shan2020collective}

\textcolor{blue}{from Ayros paper}

\cite{valentini2015efficient} \textcolor{red}{incomplete reference, from Aryo paper 49, belongs to decision making, doesn't belong here}
% 100 kilobots decision making

multi-robot fusion problems are well understood, and existing methods are powerful
centralized
\textcolor{red}{About collective sensing : make it its section in applications}
These are all surveys
\cite{yan2013survey} 
\cite{sun2017multi}
\cite{rizk2019cooperative}
\cite{li2021multi}
\textcolor{red}{consider drop this fusion part}

check colors, many about decision making
\cite{strobel2018managing} \textcolor{red}{ref incomplete}
\cite{ebert2018multi} \textcolor{red}{decision making}
\cite{shan2020collective}
\cite{bartashevich2019benchmarking}
\cite{shan2021discrete} \textcolor{red}{too tech}

another scenario, about the quality of the block
\cite{prasetyo2018best}   best of N
\cite{prasetyo2019collective}  best of N
\cite{khaluf2017edge}  find edge of an area
\cite{wahby2019collective} aggreate to the bright area
\cite{khaluf2020construction} about construction, fusion sensing the density of the building
\cite{capitan2013decentralized} 

not fully de-centralized
\cite{mirzaei2007performance} fixed sensor  \textcolor{red}{a bit tech}
\cite{rodrigues2015beyond} 
\cite{stroupe2001distributed}
\cite{zadorozhny2013information}
\cite{sasaoka2016multi}
\cite{czarnetzki2010handling}
\cite{otte2016collective} This is one a bit different, swarm robot nerval network
\cite{kornienko2005cognitive}
\cite{giusti2012cooperative}

