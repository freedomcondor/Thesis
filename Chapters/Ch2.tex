\chapter{Related Work}
\label{ch2}

% skeleton
% 1 Hybrid, hierarchical, and leadership approaches in the literature   TODO from Sci-Ro paper, yuwei paper
% 2 Swarm vs. centralized approaches for different tasks (explain what is possible to do with both, for those tasks -- current limitations are implied)
%   1.1 formation control
%   1.2 exploration, search and rescue    TODO from Sci-Ro paper
%   1.3 drone entertainment shows...
%   1.4 (more?)
% 3 The process of swarm programming / re-programming (mostly about how this is done offline, don't worry about the limitations when trying to do this online)
% 4 Swarm autonomy    TODO from Builderbot paper and zotero

\section{Introduction}

\textcolor{blue}{what is swarm, swarm is good}

Swarm intelligence often draws inspiration from collective behaviors in social animals such as ants, birds, and fish
    \cite{bonabeau1999swarm}.  % the book of Swarm intelligence
Early models like Reynolds' Boids have demonstrated that simple local behavioral rules can give rise to coherent large-scale patterns
    \cite{reynolds1987flocks}.
Following these insights, the concept of self-organization refers to the principle that individuals interact with each other locally without centralized control, and complex global behavior emerges.
Swarm Robotics aims to leverage the principles of self-organization to construct decentralized robotic systems that are scalable, adaptive, and fault tolerant
    \cite{dorigo2014swarm,    % scholarpedia on swarm robotics
          dorigo2020reflections,
          dorigo2021swarm}.    %past, present and future
These self-organized systems have shown potential in many applications, including
    environmental monitoring
    \cite{talamali2021less},
    navigation and transportation
    \cite{dorigo2013swarmanoid},
    self-assembly
    \cite{rubenstein2014programmable},
    and collective construction
    \cite{team2012designing, petersen2019review}.
Owing to the absence of centralized control, such systems can be deployed over large spatial domains and can scale up to vast numbers of robots without demanding increasing capabilities of each individual robots.

\textcolor{blue}{but, macro, micro problem, and swarm is not always efficient}

However, despite significant progress, it remains a fundamental scientific question how to design low-level individual robot behaviors to result in the desired high-level collective outcomes.
The micro-macro problem refers to this challenge
    \cite{dorigo2021swarm}.
Despite that various theoretical frameworks have been proposed
    \cite{hamann2008framework,
          hamann2010space,
          hamann2018swarm,
          hamann2013towards}
and decentralized controllers have been demonstrated for specific tasks
    \cite{dorigo2004evolving,
          nouyan2009teamwork,
          dorigo2013swarmanoid,
          rubenstein2014programmable,
          li2019decentralized},
yet purely decentralized swarms are often difficult to design and manage, and may take long time to finish the task, especially in tasks that require global information, such as environmental analyzing and decision making.
    \cite{dorigo2021swarm,
          kengyel2015potential}. % pure flat is not good

\textcolor{blue}{Therefore, hierarchical}

These limitations motivate research into hybrid and hierarchical approaches that integrate selective forms of centralization into decentralized systems.
These systems seek to bypass the micro-macro problem,
                      ease the design of swarm behaviors and coordination strategies,
                      and improve the manageability of a decentralized system,
while preserving scalability and fault tolerance features.

\textcolor{blue}{In this chapter, we review}

This chapter reviews prior work in hybrid and hierarchical swarm control,
compares the existing literature on decentralized and centralized approaches in different tasks,
and discusses aspects of swarm manageability, including human-swarm interaction, swarm re-programming, and swarm autonomy.

\section{Heterogeneity and leadership}

\textcolor{blue}{flat swarms}

Flat and homogeneous swarms, as their classic nature inspiration
    \cite{buhl2006disorder,
          detrain2008collective,
          theraulaz1998origin},
assume that all agents are identical, follow the same behavioral rules, and interact with each other in the same way
    \cite{viragh2014flocking,
          vasarhelyi2018optimized,
          floreano2008bio,
          csahin2004swarm,
          beni1988concept}.
%Such systems excel in robustness and scalability but struggle with tasks that require differentiated roles, long-range coordination, or efficient global organization.

\textcolor{blue}{heterogeneous and hierarchy, the boundary is vague}

In contrast, heterogeneous swarms consist of individuals serving different roles or having different behaviors.
These systems,
with individuals in different physical types or with different information,
can outperform purely homogeneous systems, depending on the task requirements
    \cite{kengyel2015potential}. % hetero is better
For example, individuals with more information can influence the behavior of others during tasks such as flocking or collective decision-making
    \cite{firat2020self, % use informal individual to cue the whole swarm to aggreate to a desired shelter
          prasetyo2018best,
          balazs2020adaptive}.
\textcolor{red}{varify: prasetyo2018best}

In some works, some individuals are assigned explicit leadership roles
    \cite{gu2009leader,
          amraii2014explicit}.
The leaders are usually pre-defined and fixed after deployment, while the rest of the coordination remains self-organized and based on local interactions. 
Each follower is assigned a leader and gets information from it, such as position references.
In this way, information flows in a more structured manner among robots, thereby easing the design of the behaviors and reducing task completion times.
Many tasks can benefit from this, for example flocking
    \cite{dalmao2011cucker,
          jia2019modelling,
          pignotti2018flocking}.
%\textcolor{red}{consider drop:}
%These leaders may be vulnerable to adversarial detection, a concern explored in    \cite{zheng2020adversarial}.
Some of these leader--follower approaches are organized into two-level systems.
Some other approaches impose a multi-level topological organization, in which the hierarchy is usually formed by fixed leader--follower pairs.

%In some works of self-assembly or formation
%    \cite{desai1999control,
%          li2019decentralized,
%          rubenstein2014programmable},
%although leaders and followers are not explicitly pre-assigned, each robot implicitly selects a leader upon joining the system and adjusts its position based on that leader’s state, effectively forming a leader–follower hierarchy.

%and thereby improving efficiency in tasks
%Such approaches can be regarded as highly heterogeneous systems in which each individual assumes its own role.

\textcolor{blue}{but hierarchical is not self-organized : not scalable nor fault tolerant}

However,
because the topology organization is usually pre-defined, these systems suffer from low scalability and single points of failure.
No existing approach has provided a general mechanism for self-organizing hierarchy—that is, adaptively forming and maintaining hierarchical structures without predefined topology or fixed leadership.
%maintaining scalability and avoiding single points of failure in these systems often require sophisticated mechanisms,
%such as temporary leadership, explicit role assignments, or distributed authority-transfer schemes.
As task complexity increases, self-organizing hierarchy becomes increasingly relevant. However, constructing hierarchy in a truly self-organizing manner remains a major challenge
    \cite{dorigo2020reflections}. %However, a capability still largely absent in classical decentralized swarms.

\section{Self-organized versus centralized or predetermined control}

\textcolor{blue}{The point is : }
\textcolor{blue}{1 we want to show there is a gap }
\textcolor{blue}{2 both sides have some downsides }
\textcolor{blue}{ : so we need self-organizing hierarchy}

Purely self-organized swarms are highly scalable and fault tolerant,
but they are often difficult to design and
         may exhibit inferior performance, such as slower task completion, in certain tasks.
In contrast, systems with control that is centralized, predetermined, or both, are much easier to design.
They can achieve higher task optimality, for example through global information, but lose scalability and fault tolerance.
This section reviews these trade-offs across major tasks in multi-robot systems.

\subsection{Multi-robot coverage}

\textcolor{blue}{what is coverage, why it is important}

%While formation focuses on maintaining morphology of the swarm, 
Coverage addresses an aspect of swarm coordination.
The goal is to ensure that a target region is fully observed, sensed, or explored by a group of agents.
Depending on how space is partitioned and visited, coverage problems typically include area coverage (maintaining spatial distribution over a region) and sweep coverage (systematically traversing a region to ensure full visitation).
Effective coverage is essential for applications such as environmental monitoring, search and rescue, precision agriculture, and surveillance.
A wide range of strategies has been developed, and most can be classified into three main categories: sensor dispersion, random exploration, and predetermined sweeps
    \cite{wang2011coverage,     % survey
          galceran2013survey}.  % survey

\textbf{Sensor Dispersion}
The goal of sensor dispersion methods is to spread sensors over the environment to cover the target area,
typically by driving robots to well-separated positions that remain largely static once coverage is achieved,
with some notable exceptions in dynamic environments
    \cite{santos2019decentralized}.
In these approaches, although sensor localization may rely on GPS or range-based measurements, interactions among robots are decentralized.
Robots typically use potential fields, repulsion mechanisms, or adaptive local policies to maintain appropriate separation and avoid sensing overlap.
Such methods offer scalability and robustness due to their decentralized nature
    \cite{santos2019decentralized,
          luo2018adaptive,
          spanogianopoulos2017fast,
          siligardi2019robust}.

\textbf{Random Exploration}
Another family of approaches relies on random motions to gradually explore the environment.
In these approaches, robots move mostly randomly and often also perform simple reactive collision avoidance.
These searching strategies are mostly self-organized.
They are easy to implement and require only minimal information, but often suffer from long completion times and inefficient coverage, due to redundancy and non-uniform exploration
    \cite{huang2019exploration,
          ichikawa1999characteristics,
          mcguire2019minimal}.

\textbf{Predetermined Sweeps}
Sweep-based methods pre-calculate systematic paths, usually S-shaped trajectories, to ensure full coverage of a known region.
These strategies often require explicit predetermined roles and global map knowledge, which yields fast and predictable coverage, but usually lacks adaptability and can suffer from communication bottlenecks and single points of failure
\cite{almadhoun2019survey, avellar2015multi}.

There are also mixed approaches.
For example, \cite{scherer2015autonomous} uses predetermined motion trajectories,
and in the data streaming stage, the robots establish a communication network in a self-organized manner.

\textcolor{blue}{summarize: a gap}

To summarize, 
%coverage tasks clearly show the dichotomy between self-organization and centralization.
while random exploration is robust and scalable, it often suffer from long completion times and inefficient coverage.
Conversely, predetermined and/or centrally coordinated methods can achieve fast and uniform coverage, but usually lack adaptability and fault tolerance.

\subsection{Path Planning}

In multi-robot systems, path planning is a key component.
Existing methods range from centralized and/or predetermined global planners to self-organized online approaches.

\textbf{Centralized Offline Planning}
Centralized planners assume global map knowledge and compute predetermined collision-free trajectories for all robots before deployment.
Since multi-robot path optimization is usually an NP-hard problem,
these methods usually employ heuristics %and approximate solutions to generate high-quality global trajectories for robots
    \cite{nazarahari2019multi,
          thabit2018multi,
          yu2016optimal,
          kushleyev2013towards}.

\textbf{Distributed Online Planning}
There are a few existing examples of distributed online planners: e.g., the EGO-Swarm family
    \cite{zhou2020ego,
          zhou2021ego,
          zhou2022swarm}.
These existing planners use local sensing and onboard optimization.
They demonstrate fast, distributed trajectory generation by predicting nearby robots' short-horizon behavior and optimizing smooth, dynamically feasible paths at high frequency.
These approaches enable agile navigation in cluttered environments but the generated solutions are of course not perfectly optimal, in contrast to what could in principle be achieved using offline global optimization.  %guarantees and may struggle in tightly constrained spaces.
%And similar to formation control, they do not consider task assignments.
However, these methods focus on decentralized navigation and trajectory generation.
The start and goal positions of each robot are still specified externally,
rather than being assigned or adapted autonomously by the system,
which limits scalability and reduces robustness to agent failures.


\textcolor{blue}{summarize: a gap}

To conclude, although distributed approaches exist to generate high-quality solutions, both offline and online,
fully scalable and fault tolerable systems remains to be explored.

\subsection{Formation control}

Formation control is a key fundamental topic in multi-robot and swarm systems
    \cite{liu2018survey,
          oh2015survey}. % surveys
Decentralized formation controllers rely on relative localization, using distance sensors, ultraviolet markers, or fiducial markers
    \cite{walter2018fast, %localization of UaVs using ultraviolet 
          ulrich2022towards}.  %fast fiducial marker with full 6 dof pose estimation

These approaches typically focus on achieving stable rigid-body formations using distance-based constraints
    \cite{yang2018growing,
          mehdifar2018finite,
          stacey2015passivity,
          anderson2018rigid,
          oh2011adistance,
          oh2011bdistance},
%which maintains formation using distance measurements,
or bearing-based constraints
%which allows formation maintenance using only bearing measurements, with two designated leaders controlling scale, rotation, and translation while other robots maintain relative bearings
    \cite{zhao2019bearing,
          zhao2015bearing,
          zhao2015translational,
          schiano2016rigidity,
          li2020adaptive}.
%          li2021adaptive,
%          li2020bearing,
%          zhao2021finite,
%          zhang2022distributed,
%          zhang2023bearing}.            % for water surface

However, these methods assume that each robot is preassigned a fixed role in the formation.
During deployment,
the interaction topology among the robots is predefined and fixed, and robots move by referencing the positions of the robots they are connected to in this predefined topology.
They offer little support for dynamic or fault-tolerant operation, emphasizing control-theoretic guarantees over flexibility or adaptivity.

\textcolor{blue}{Why role switching is important}

On the other hand, a key feature of swarm robotics is the ability to deploy robots without pre-assigned roles.
Moreover, for fault tolerance, the swarm should be able to reassign roles in response to unexpected failures.
These considerations highlight the importance of role assignment.
In formation tasks, efficient role assignment can reduce unnecessary crossing movements among robots and thereby speed up formation convergence.
However, in large swarms or environments prone to agent failures, preassigning roles is often impractical.
%Some self-assembly approaches allow robots to autonomously select vacant positions, but these methods are often inefficient due to random search
%    \cite{rubenstein2014programmable,
%          li2019decentralized}.

Centralized algorithms can compute globally optimal role assignments by leveraging full knowledge of robot positions and formation targets
    \cite{rm2020review,
          macalpine2015scram,
          agarwal2018simultaneous,
          ravichandar2020strata,
          akella2020assignment},
ensuring minimal total displacement and/or collision-free paths.

\textcolor{blue}{distributed assignment}

Distributed assignment strategies instead divide computational responsibilities among robots,
using, e.g., consensus mechanisms, market-based coordination, or distributed optimization algorithms
    \cite{mosteo2017optimal,
          burger2012distributed,
          chopra2017distributed,
          zavlanos2007distributed,
          alonso2016distributed,
          michael2008distributed,
          montijano2014efficient,
          wang2020shape}.
In many such approaches, although computation is distributed, the per-robot computational and memory burden still grows with swarm size.
For example, some methods require each robot to solve a portion of a global optimization problem, but the calculation dimensionality of each portion still increases with the number of agents,
while others rely on bidding or consensus schemes in which each robot must maintain information about all the peers.
That is why, in despite of these distributed efforts, these methods still struggle to scale efficiently with swarm size.
Approaches handling faulty or interchangeable agents also remain limited
    \cite{kambayashi2018distributed}.

In addition, switching formations under rigid-body assumptions has also been explored,
where the swarm transitions between multiple predefined rigid formations based on environmental changes 
by updating inter-agent constraints while preserving overall formation rigidity
    \cite{desai1999control,
          desai2002modeling}.
However, these methods still rely on centralized commands to trigger formation switches, rather than allowing the swarm to adapt autonomously.

\textcolor{blue}{there is a gap}

In summary, while centralized or predetermined approaches can ease the implementation of optimization algorithms,
they lack robustness and scalability.
%self-organizing hierarchy for role allocation remain largely underexplored.
%Developing such mechanisms is essential for achieving both efficiency and resilience in large-scale, formation tasks.

\subsection{Light Show}

\textcolor{blue}{centralized : not fault tolerant, not scalable}

Drone light shows are becoming popular in urban life.
Existing shows use a fully centralized and predetermined approach to robot coordination,
where global trajectories are precomputed offline and any commands that are issued during the show are broadcast to all agents from a ground station.
This can provide precise synchronization and visually coherent formations with very many robots
    \cite{waibel2017drone,
          ang2018high}.
However, the resulting systems are vulnerable to a single point (ground station) of failure and to communication failures and bottlenecks,
lacking the scalability or adaptability typically associated with decentralized systems.

\section{Behavior design and manageability}

While the previous sections have focused on specific tasks such as coverage, formation control, and path planning, these tasks are typically considered in isolation. 
From a broader robot system perspective, a multi-robot system should not only execute individual tasks but also operate autonomously, making decisions and adapting to a changing environment without continuous human commands. 
At the same time, for practical deployment, these systems must remain manageable: human operators should be able to supervise and intervene when necessary.
This section reviews swarm autonomy and manageability.

\subsection{Swarm autonomy}

\textcolor{blue}{indivudal autonomy is good, but swarm autonomy is a challenge}

Autonomy is often regarded as a central long-term goal of robotic systems—the ability for robots to operate without continuous human supervision.

In despite of significant advances in individual robot autonomy
    \cite{cangelosi2022cognition,
          vernon2014artificial},
achieving autonomy in swarm robotics remains an open challenge.
% for swarm, good works 
Existing swarm researches primarily focus on single task such as action, perception, and adaptation
    \cite{heinrich2022swarm},
but higher level cognitive capabilities such as learning or anticipation are still largely absent,
and it lacks an overall framework to unify all the attributes.

\textcolor{blue}{Collective decision is a key component for swarm autonomy}

Collective decision making, as a classical task for swarm systems for environmental analysis, poses an important role in swarm automony.
Classical models for collective decision making, such as best-of-N and best-of-two
    \cite{khaluf2019neglected, % decision making
          valentini2017best,   % best of N
          dorigo2014self,
          valentini2016collective,
          shan2020collective,
          ebert2018multi,
          prasetyo2018best,       % best of N
          prasetyo2019collective, % best of N
          valentini2015efficient}, % best of N
demonstrate how simple local rules allow groups to reach agreement without centralized control.
Subsequent work investigates convergence speed, decision accuracy, and robustness under noise or malicious information
    \cite{shan2020collective,            % bayesian
          bartashevich2019benchmarking,  % 
          shan2021discrete, %\textcolor{red}{too tech}
          strobel2018managing}. %\textcolor{red}{ref incomplete} % byzenting

\textcolor{blue}{but Collective decision too simple, swarm needs higher level of cognition}

Despite these progress, decision-making in swarms generally remains a simple reactive task for environmental analysis,
rather than maintaining a broader understanding of the mission context.
Many elements are missing, in terms of automony,
    \cite{khaluf2019neglected},
such as the swarm needs to detect that a new collective decision is required, or to autonomously adapt the decision-making process to conditions that were not explicitly studied beforehand.
Current frameworks provide no unified mechanisms for learning new behaviors online, anticipating future events, or autonomously restructuring task workflows.

\textcolor{blue}{current swarm is roughly at level 2}

The SAE J3016 standard
    \cite{sae2021automated}
defines autonomous level for robot systems.
Existing robot swarm systems would be barely classified as having collective behaviors equivalent to the individual behaviors described in {\it Level 2 (Partial automation)},
as existing robot swarm systems are able to
execute intended tasks, but a human operator is still required to continuously monitor the environment and system and decide when to intervene to correct or stop the system.
That would be a challenge for swarm systems as will be talked about in the following section.

\subsection{Human-swarm interaction}

\textcolor{blue}{human-swarm interaction is needed}

As discussed in previous section, although robot swarms are generally capable of executing their intended tasks,
human operators still play a crucial role in supervising the system,
issuing high-level commands, and reprogramming the swarm when necessary.
However, as swarm size increases, direct communication with all the robots in the swarm becomes infeasible.
This section reviews existing research on human–swarm interaction and swarm online programming.

Current researches mainly focus on two sides : human interfaces and swarm influence.
    \cite{siean2023opportunities,
          kolling2015human}.

\textcolor{blue}{interface side: gesture control and etc}

A class of researches explore interfaces to manipulate the swarm such as gesture control
    \cite{alonso2015gesture,
          podevijn2013gesturing},
haptic devices
    \cite{lee2013semiautonomous},
or wearable systems
    \cite{jarvis2025first}.
Among these works, by default, operators are able to communicate with all the robots in the swarm to send commands collectively
    \cite{ayanian2014controlling,
          macchini2021personalized,
          abioye2025user}
\textcolor{red}{varify},
which will face scalability problem when the swarm scales up.

\textcolor{blue}{influence side: decentralized, but not efficient}

For decentralized methods where human communicate with only one robot or subset of the swarm, researches forcus on how to make the subset of the swarm influence the whole swarm
    \cite{podevijn2013gesturing, % gesture, subset of the swarm
          zhou2016assistive,  % joysticks
          lee2013semiautonomous}. % haptic feedback
While these methods improve scalability,
they often suffer from slow convergence to the desired swarm-wide behavior.

\cite{kolling2013human} \textcolor{red}{verify}

\textcolor{blue}{so, a gap}

In general, there is a lack of a framework that can issue commands to swarm robots with low latency while maintaining scalability.

\subsection{programming / re-programming}

\textcolor{blue}{People needs to easily (and online) program the swarm, but with challenges}

When supervising the swarm, the programmer needs to be able to re-program the swarm easily and online.
However, it faces two problems.
Firstly, designing controllers for swarm systems is often a long trial-and-error process
    \cite{hamann2018swarm,
          brambilla2013swarm}.  
It becomes more challenging when swarms must adapt to new tasks or environments.
Secondly, when the swarm scales up, it becomes difficult for the programmer to communicate with the swarm and deliver the new program.

This section reviews approaches from offline, centralized online, and to decentralized online reprogramming.

\textcolor{blue}{Offline Programming}

Classically, swarms are deployed in offline way.
Programs are pre-designed and pre-downloaded to all the robots before deployment, for example,
    \cite{rubenstein2014programmable,
          valentini2016collective,
          werfel2014designing,
          dorigo2013swarmanoid}.
However, offline methods cannot react to unexpected task changes during deployment.
Noteably, automatic design methods falls in this category.
    \cite{francesca2014automode,
          francesca2016automatic,
          birattari2019automatic,
          salman2024automatic}.
They demonstrate how optimization and evolutionary algorithms can produce robust swarm behaviors without manual tuning.
Yet the programs are pre-generated.

\textcolor{blue}{online but centralized}

Centralized online reprogramming techniques update robot behaviors after deployment through a global communication channel.
Over-the-air programming frameworks
    \cite{zyrianoff2024over,
          abadie2024robotap}
enable operators to push new software to all robots.
Large-scale drone light shows
    \cite{waibel2017drone,
          ang2018high}
represent a widely deployed example of centralized online control.
Although they can coordinate thousands of robots,
they typically allow only predefined trajectories and low-level commands.

\textcolor{blue}{online, decentralized}

Decentralized reprogramming methods update robot controllers without relying on a central broadcaster.
Early work in wireless sensor networks
    \cite{xie2011design,
          wang2006reprogramming}
shows the feasibility of distributed software updates.
However, they also highlights fundamental limitations such as slow propagation and high energy consumption resulting from consensus mechanisms
    \cite{de2009energy,
          varadharajan2018over}.

\textcolor{blue}{summarize}

To summarize, a unified framework that enables a human operator to reprogram swarm behavior in a scalable and efficient manner remains largely unexplored.

\subsubsection{behavior tree}

Behavior trees (BTs) provide a modular and hierarchical framework for representing robot behaviors
    \cite{colledanchise2018behavior,
          iovino2022survey},
which have been adopted in several multi-robot applications
    \cite{colledanchise2016advantages,
          jeong2022behavior}.

Due to their compositionality and readability, BTs offer a potential solution to key challenges in swarm autonomy and human–swarm interaction. 
They allow robots to adapt their behavior online, enable hierarchical organization of complex tasks, and facilitate interaction with human operators through modular control structures. 

Evolving BTs via evolutionary algorithms—such as genetic programming
    \cite{jones2018evolving,
          kuckling2022automode}
or grammatical evolution
    \cite{neupane2019learning,
          kuckling2022automode}—
enables automated creation of swarm behaviors.
Recent advances support online evolution as well
    \cite{jones2019onboard,
          venkata2023kt},
allowing robots to adapt their BT controllers during deployment.

Dynamic BTs have also been investigated in game AI
    \cite{florez2008dynamic},
though their application to real-world robotics remains to be explored.

More recent work exploit behavior trees for knowledge transfer in multi-robot systems as a adaptability mechanism.
For instance, Venkata et al.\ \cite{venkata2023kt} introduce the KT-BT framework, which uses behavior trees and a grammar-based encoding (stringBT) to achieve decentralized transfer of acquired skills, improving collective performance during online adaptation.



\section{Fault tolerance}

Fault tolerance is an inherent property of swarm systems for their redundancy and absence of single points of failures
    \cite{dorigo2014swarm}
    \cite{hamann2018swarm}.

\textcolor{red}{cite/explain some normal fault tolerance, that is not for centralized, predetermined}
Most classical swarm tasks, such as coverage or collective decision making, can tolerate a subset of robots failes.
Robots simply continue to operate as normal, without knowing or caring about the failure peers \textcolor{red}{cite something here}.

\textcolor{blue}{However, IFs}

However, intermittent faults (IFs) pose a unique challenge for swarm systems.
An IF is a failure that occurs for a short period of time, and then recovers.
Its sporadic nature makes it difficult to reliably detect,
and the repeated reoccurrence of IFs can often degrade overall swarm performance
    \cite{zhou2019review}
    \cite{niu2021distributed}.
\textcolor{red}{check what Sinan says, the argument here not strong enough}

Centralized methods are highly effective for identifying IFs, since they have access to system-wide data.
    \cite{sheng2021intermittent}
    \cite{zhang2021intermittent}
    \cite{syed2016novel},
but as discussed in the previous section, centralized methods are vulnerable to single point failures and not scalable.

\textcolor{blue}{Therefore, a gap}

Overall, both centralized and de-centralized swarm system have their own unique challenges in terms of fault tolerance.


\section{Summary}

In summary, centralized approaches offer global optimality and ease of control, while decentralized swarms provide robustness and scalability.
However, neither paradigm alone fully addresses the need for efficient coordination in large-scale, dynamic tasks that require both adaptability and performance guarantees.
A self-organizing hierarchical framework is therefore essential to bridge this gap,
enabling swarms to dynamically reconfigure their coordination structure,
                   balance local autonomy with global oversight,
               and achieve human-scalable manageability without sacrificing resilience.



\iffalse
\section{Abandon}

\textcolor{blue}{from Ayros paper}

multi-robot fusion problems are well understood, and existing methods are powerful
centralized
\textcolor{red}{About collective sensing : make it its section in applications}
These are all surveys
\cite{yan2013survey} 
\cite{sun2017multi}
\cite{rizk2019cooperative}
\cite{li2021multi}
\textcolor{red}{consider drop this sensor fusion part}


another scenario, about the quality of the block
\cite{khaluf2017edge}  find edge of an area
\cite{wahby2019collective} aggreate to the bright area
\cite{khaluf2020construction} about construction, fusion sensing the density of the building
\cite{capitan2013decentralized} 

not fully de-centralized
\cite{mirzaei2007performance} fixed sensor  \textcolor{red}{a bit tech}
\cite{rodrigues2015beyond} 
\cite{stroupe2001distributed}
\cite{zadorozhny2013information}
\cite{sasaoka2016multi}
\cite{czarnetzki2010handling}
\cite{otte2016collective} This is one a bit different, swarm robot nerval network
\cite{kornienko2005cognitive}
\cite{giusti2012cooperative}
\fi
