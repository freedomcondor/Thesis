\chapter{Related Work}
\label{ch2}

%-------------------------------------------------------------
% P1 start summary :
% Although swarm researches developed over the past decades, researches focus on aspect.
% there lacks a systematic solution to cover overall applications in real scenario.

% P2 an ideal swarm system should be :
%   easy to deploy, change(reprogram)
%   use only local information
%   morphology
%   as a whole, sense the environment, make decision and react (autonomous).

% P3 however decentralized and centralized methods each can fulfill part of those.

% P4 This chapter gives a review of each of those


%subsection : Hybrid, hierarchical
% As centralized - decentralized each has its own pros and cons
% there are researches try to engage hybrid or hierarchical reserachs

%subsection : Task assignment in formation

%subsection : swarm autonomous

%subsection : reprogramming and human swarm interaction

%-----------------------------------------------------------


Chapter 1 described some concrete problems and why they are important and difficult.
In this chapter, we review and discuss literatures about them.

% a swarm should be
As discussed in Ch1, a swarm should be
    easy to deploy
    use only local information
    as a whole, sense the environment and react.

% however difficult
However, with these constraints, many tasks are difficult to complete.

\section{self-organization hierarchy}

Many researches try to combine centralized and decentralized.
They go hierarchical, with hierarchical, information can easily flow

\section{Task assignment in formation}

formation is researched in control theory.

Task assignment makes efficient formation
Task assignment calculated hungarian, network flow, n-flex algorithm

distributed task assignment, each robot calculate a part of it, but calcualtion increases

\section{swarm autonomous}

what's in the builderbot paper.

\section{reprogramming and human swarm interaction}

what's in the builderbot paper.
