\chapter{Related Work}
\label{ch2}

% 1 Hybrid, hierarchical, and leadership approaches in the literature   TODO from Sci-Ro paper, yuwei paper
% 2 Swarm vs. centralized approaches for different tasks (explain what is possible to do with both, for those tasks -- current limitations are implied)
%   1.1 formation control
%   1.2 exploration, search and rescue    TODO from Sci-Ro paper
%   1.3 drone entertainment shows...
%   1.4 (more?)
% 3 The process of swarm programming / re-programming (mostly about how this is done offline, don't worry about the limitations when trying to do this online)
% 4 Swarm autonomy    TODO from Builderbot paper and zotero

\textcolor{blue}{what is swarm}

swarm is inspired from social animals like ants, birds ... \cite{bonabeau1999swarm}

starts from boids \cite{reynolds1987flocks}

robot swarms follows that principal, try to create a robots system that is scalable and fault tolerant.
\cite{dorigo2014swarm}
\cite{dorigo2020reflections} %reflections to swarm robotics.
\cite{dorigo2021swarm} %past, present and future

They useful in applications like 
Environmental monitoring \cite{talamali2021less}
navigation and transport \cite{dorigo2013swarmanoid}
self-assembly \cite{rubenstein2014programmable}
construction: \cite{team2012designing} \cite{petersen2019review}

biohybrid interaction 
\cite{wahby2018autonomously}
\cite{halloy2007social}

although fundamental questions still needs to be answered : Macro-micro problem:
\cite{hamann2008framework}
\cite{hamann2010space}
\cite{hamann2018swarm}
\cite{hamann2013towards} % swarm calculus
\cite{valentini2015efficient} \textcolor{red}{incomplete reference, from Aryo paper 49}

researches have already shown that robots can work in decentralized way
\cite{dorigo2004evolving}
\cite{nouyan2009teamwork}
\cite{dorigo2013swarmanoid}
\cite{rubenstein2014programmable}
\cite{li2019decentralized}

However, in many scenarios, due to partial information, decentralized shows its deficiency.
\cite{eberhardinger2018approach}
\cite{eberhardinger2018measuring}
\cite{kaddoum2010criteria}
\cite{nedic2018network} Network topology and communication-computation tradeoffs in decentralized optimization
\cite{jovanovic2016controller} Tradeoffs ...

Therefore, hierarchical.

\section{Hybrid, hierarchical, and leadership approaches in the literature}

\cite{dorigo2020reflections} it is challenging to self-organized hierarchy.

other Challenges include automation design:
\cite{francesca2016automatic}
\cite{birattari2019automatic}
\cite{salman2024automatic}
and heterogeneity 
\cite{kengyel2015potential}

flat swarm:
\cite{viragh2014flocking}
\cite{vasarhelyi2018optimized}

some books about swarm talking about flat swarm
\cite{beni1988concept}
\cite{bonabeau1999swarm}
\cite{csahin2004swarm}
\cite{floreano2008bio}

animals in flat system
\cite{buhl2006disorder}
\cite{detrain2008collective}
\cite{theraulaz1998origin}

some heterogeneous with some members:
\cite{firat2020self} : more informed 
\cite{valentini2016collective} : more communicative
\cite{balazs2020adaptive} : more persistent

with leaders:
\cite{gu2009leader}
\cite{amraii2014explicit}
\cite{zheng2020adversarial}
\cite{shan2020collective}
\cite{kaiser2022innate}

hierarchical
\cite{dalmao2011cucker}
\cite{pignotti2018flocking}
\cite{jia2019modelling}
\cite{divband2019photomorphogenesis}

\cite{zhou2022swarm} partially centralized, fixed structure

MNS:
\cite{mathews2017mergeable}
\cite{zhu2020formation}
\cite{zhang2023self} Yuwei's paper, Self-reconfigurable hierarchical for formation control
\cite{jamshidpey2020multi}
\cite{jamshidpey2024centralization}
\cite{jamshidpey2023reducing}

\section{Swarm vs. centralized approaches for different tasks}

\subsection{Formation}

Formation is a classic task for swarm.
\cite{liu2018survey} a survey
\cite{oh2015survey} a survey

It relies on localization of neighbours
\cite{walter2018fast} %localization of UaVs using ultraviolet 
leD markers
\cite{ulrich2022towards} %fast fiducial marker with full 6 dof pose estimation

\subsubsection{Formation control in control theory}

\cite{stacey2015passivity} proposes a passivity-based formation control framework for 3D dynamic vehicles that achieves local asymptotic stability using partial relative position measurements through virtual mechanical couplings and adaptive compensation.

\cite{desai1999control} presents a framework for coordinating a team of mobile robots to navigate and change formations in obstacle-filled environments using a leader-follower structure, control graphs, and optimal control\-based path planning.

\cite{spanogianopoulos2017fast} applies the fast-converging nPSO algorithm to enable rapid, collision-free optimal formation of UAV swarms in congested urban environments with large obstacles.

\cite{desai2002modeling} presents a graph-theoretic framework for modeling and controlling formations of nonholonomic mobile robots, enabling coordinated navigation, formation maintenance, and transitions between formations in obstacle-filled environments.

\cite{kushleyev2013towards} presents the design, modeling, and control of agile 75 g micro quadrotors capable of autonomous flight and tight formation coordination in constrained indoor environments, demonstrated with experiments on a 20-quadrotor swarm.

\cite{schiano2016rigidity} bearing formation controller

rigid body graph
\cite{yang2018growing}
\cite{mehdifar2018finite}

bearing
\cite{zhao2015translational}
\cite{zhao2015bearing}
\cite{li2020adaptive}
\cite{li2021adaptive}
\cite{li2020bearing}
\cite{zhao2021finite}
\cite{zhang2022distributed}
\cite{zhang2023bearing}
\cite{arrigoni2018bearing}
\cite{zhao2019bearing}
\cite{trinh2018bearing}
\cite{trinh2021finite}

bearing rigidity
\cite{tay1985generating}
\cite{eren2012formation}
\cite{trinh2019minimal}
\cite{karimian2017theory}
\cite{hou2016elementary}
\cite{carboni2014rigidity}

\subsubsection{target position assignment}

To form the formation efficiently, it is crucial to assign the robots to a good target position based on its current positions.

\cite{rm2020review} presents a concise review on the variant state-of-the-art dynamic task allocation strategies.

\cite{burger2012distributed} introduces a distributed simplex algorithm for solving degenerate linear programs and multi-agent assignment problems over asynchronous peer-to-peer networks, achieving consensus on optimal solutions with linear scalability in communication graph diameter.

\cite{chopra2017distributed} presents a distributed version of the Hungarian method that enables teams of robots to cooperatively solve assignment and routing problems with spatiotemporal constraints through local computations and communications over a peer-to-peer network.

\cite{akella2020assignment} presents algorithms for optimally assigning interchangeable robots to variable goal formations by partitioning the formation parameter space into cost-invariant equivalence classes and efficiently solving linear bottleneck assignment problems for each class. (\textcolor{red}{centralized best time assignment})

\cite{li2019decentralized} proposes a decentralized algorithm that enables robot swarms to progressively form target shapes represented as point clouds using local interactions guided by an acyclic directed graph, ensuring reliable convergence with minimal formation error.

\cite{zavlanos2007distributed} presents a distributed formation control framework that combines consensus algorithms, market-based coordination, and artificial potential fields to achieve permutation-invariant formations, ensuring convergence and scalability using only local agent information.

\cite{alonso2016distributed} proposes a distributed formation control approach that combines consensus and sequential convex optimization to enable teams of robots to navigate and reconfigure safely among static and dynamic obstacles in 2D and 3D environments.

\cite{michael2008distributed} introduces a distributed, market-based algorithm for dynamic multi-robot task assignment that guarantees efficient convergence to desired task distributions while enabling coordinated formation control, splitting, and merging behaviors.

\cite{kambayashi2018distributed} proposes a decentralized swarm formation control method using mobile software agents that mimic ant–pheromone communication, enabling robots to self-organize into formations through distributed, cooperative interactions.

\cite{montijano2014efficient} presents a distributed optimization framework for multi-robot formation control that simultaneously solves for optimal positions and role assignments, using averaging and a modified distributed simplex algorithm to guarantee globally optimal configurations.

\cite{mosteo2017optimal} presents an optimal algorithm for assigning roles and positions to homogeneous robots in freely placeable formations by jointly computing formation translation, rotation, and role assignments to minimize total displacement, with provable correctness and a distributed implementation.

\cite{macalpine2015scram} introduces SCRAM, a scalable, collision-free role assignment framework that assigns interchangeable robots to target positions to minimize makespan while avoiding collisions, with applications to large-scale multi-robot systems.

\cite{wang2020shape} presents a distributed algorithm for homogeneous robot swarms that concurrently assigns goal locations and plans collision-free paths, enabling fast and reliable shape formation for up to thousands of robots.

\cite{agarwal2018simultaneous} presents an algorithm that simultaneously optimizes robot assignments and variable goal formation parameters by transforming the problem into a linear sum assignment problem, enabling collision-free trajectories with O(n³) computational complexity.

\cite{ravichandar2020strata} introduces STRATA, a unified framework for task assignment in large heterogeneous agent teams that accounts for continuous traits, species- and agent-level variability, and ensures tasks' trait requirements are met efficiently.


\subsection{coverage}:

\cite{wang2011coverage} survey : fixed sensor
\cite{galceran2013survey} survey : moving sensor

sensor network
\cite{luo2018adaptive}
\cite{santos2019decentralized}
\cite{siligardi2019robust}

environmental monitoring, collective perception
\cite{schmickl2006collective} : 
\cite{baxter2007multi} search and rescue
\cite{lima2017cellular} foraging

\cite{julia2012comparison} : single robot? path planning coverage
\cite{almadhoun2019survey} a survey of multi-robot covrage path planning
\cite{avellar2015multi}

area divided, each robot sweep one block
\cite{rekleitis2008efficient}
\cite{scherer2015autonomous}

offline, path optimization
\cite{nazarahari2019multi}
\cite{thabit2018multi}
\cite{yu2016optimal}

offline construct map
\cite{mirzaei2011cooperative}

online construct map
\cite{ge2005complete}
\cite{miki2018multi}

about map broadcast
\cite{marjovi2009multi}
\cite{albani2017field}

randomwalking coverage:
\cite{kegeleirs2019random}
\cite{huang2019exploration}
\cite{ichikawa1999characteristics}
\cite{mcguire2019minimal}
\cite{pang2021effect}
\cite{khaluf2018collective}
\cite{zia2017cognitive}

pheromone based
\cite{koenig2001terrain}
\cite{schroeder2017efficient}
\cite{deshpande2017robot}
\cite{maftuleac2015local}
\cite{stirling2010energy}

formation sweep :

centralized:
\cite{campbell2012review}
\cite{wang1991navigation}
\cite{din2018behavior}

self-organized: 
MNS et al
\cite{jamshidpey2020multi}
\cite{jamshidpey2023reducing}
\cite{mathews2017mergeable}
\cite{zhang2023self}
\cite{zhu2020formation}
\cite{zhu2024self}

\subsection{others}:
\cite{howard2006multi} mapping
\cite{psaraftis2016dynamic} path optimization

\cite{tuci2018cooperative} transportation
\cite{robin2016multi} search and rescue



\subsection{complex network}
\cite{kirst2016dynamic}
\cite{zavlanos2011graph}

if connectivity is not reliable, system convergence and performance guarantees can be compromised
\cite{cortes2008distributed} consensus
\cite{de2006decentralized}
\cite{moreau2005stability}
\cite{olfati2007consensus} consensus

\subsection{fault tolerance}
\cite{bjerknes2013fault}
\cite{tarapore2017generic}
\cite{winfield2006safety}
\cite{strobel2018managing} \textcolor{red}{incomplete}
\cite{pini2011task}
\cite{o2023predictive}
\cite{oladiran2019fault}

fault detection
\cite{tarapore2019fault}
\cite{khaldi2017monitoring}

exogenous fault detection
\cite{khadidos2015exogenous}
\cite{millard2016exogenous}
\cite{millard2013towards}

replacing/repairing the failed robots without pausing the mission
\cite{christensen2009fireflies}
\cite{varadharajan2020swarm}

\subsection{IFS, maybe not needed}
\cite{edition2000authoritative}
\cite{zhou2019review}
\cite{niu2021distributed}
\cite{sheng2021intermittent}
\cite{zhang2021intermittent}
\cite{syed2016novel}

\subsection{path planning}

UAV motion planning

\cite{quan2020survey} surveys recent state-of-the-art UAV motion planning methods, covering path finding and trajectory optimization techniques along with their motivations, formulations, and real-world applications.

\cite{zhou2022swarm} introduces a fully autonomous swarm of palm-sized flying robots equipped with an efficient onboard trajectory planner that enables real-time, collision-free navigation and coordination in cluttered natural environments like dense forests.

\cite{han2019fiesta} presents FIESTA, a fast incremental mapping system for building global Euclidean Signed Distance Fields (ESDFs) that enables real-time motion planning for aerial robots through efficient obstacle updates and high-performance map maintenance.

\cite{wang2022geometrically} proposes an optimization-based framework for multicopter trajectory planning that efficiently handles geometric and dynamic constraints through a novel trajectory representation and constraint elimination technique, achieving high-quality, real-time solutions.

\cite{wang2025unlocking} presents a system that enables quadcopters to autonomously generate and execute complex aerobatic maneuvers using a discrete maneuver representation, spatial-temporal trajectory optimization, and yaw compensation strategies in cluttered environments.


\section{Controllability}

Human needs to supervise a swarm : interact, steer, reprogram.

However the difficulty is, when scale up, human operator can't communicate with all the robot.

\subsection{Human-swarm interaction}

survey:
\cite{siean2023opportunities}
\cite{kolling2015human}

Human interface side
\cite{jarvis2025first}
\cite{abioye2025user}
\cite{alonso2015gesture}

human to the whole swarm
\cite{ayanian2014controlling}
\cite{alonso2015gesture}
\cite{macchini2021personalized}

\cite{podevijn2013gesturing} gesture, subset of the swarm
\cite{zhou2016assistive} joysticks
\cite{lee2013semiautonomous} haptic feedback

\cite{kolling2013human}

\subsection{programming / re-programming}

\subsection {offline examples}

mostly offline :

usually, long trial and error: 
\cite{hamann2018swarm} a book
\cite{brambilla2013swarm} a review

automatic design:
\cite{francesca2014automode} AutoMoDe
\cite{francesca2016automatic}
\cite{birattari2019automatic} offline automatic design

examples of pre-design
\cite{rubenstein2014programmable} self-assembly
\cite{valentini2016collective} decision making
\cite{werfel2014designing} construction
\cite{dorigo2013swarmanoid} multi task mission

\subsection {online but centralized}
\cite{zyrianoff2024over} over the air
\cite{abadie2024robotap}

\subsection {light show: too big, only simple commands}
\cite{waibel2017drone} light show
\cite{ang2018high} light show

\subsection {online, decentralized}
\cite{xie2011design} wireless sensor network
\cite{wang2006reprogramming} wireless sensor network
but slow : consensus
\cite{de2009energy}
\cite{varadharajan2018over} 

\cite{venkata2023kt} introduces the KT-BT framework, using behavior trees and stringBT grammar, enabling multirobot knowledge transfer, with simulations showing improved group performance.

\section{Swarm Autonomous}

From builderbot paper :
What Is Cognitive Robotics?
\cite{cangelosi2022cognition} 
\cite{vernon2014artificial} 
\cite{heinrich2022swarm}
\cite{khaluf2019neglected}

level standard :
\cite{sae2021automated}

collective decision
\cite{valentini2017best} best of N
\cite{dorigo2014self}
\cite{valentini2016collective}

multi-robot fusion problems are well understood, and existing methods are powerful
These are all surveys
\cite{yan2013survey} 
\cite{sun2017multi}
\cite{rizk2019cooperative}
\cite{li2021multi}

check colors
\cite{strobel2018managing} \textcolor{red}{ref incomplete}
\cite{ebert2018multi}
\cite{shan2020collective}
\cite{bartashevich2019benchmarking}
\cite{shan2021discrete}

another scenario, about the quality of the block
\cite{prasetyo2018best}
\cite{prasetyo2019collective}
\cite{khaluf2017edge} 
\cite{wahby2019collective}
\cite{khaluf2020construction}
\cite{capitan2013decentralized}

not fully de-centralized
\cite{mirzaei2007performance} fixed sensor
\cite{rodrigues2015beyond}
\cite{stroupe2001distributed}
\cite{zadorozhny2013information}
\cite{sasaoka2016multi}
\cite{czarnetzki2010handling}
\cite{otte2016collective}
\cite{kornienko2005cognitive}
\cite{giusti2012cooperative}

\subsection{behavior tree}

Behavior tree general
\cite{colledanchise2018behavior}
\cite{iovino2022survey}

BT for multi-robot system
\cite{colledanchise2016advantages}
\cite{jeong2022behavior}

Evolving BT:

genetic programming
\cite{jones2018evolving}
\cite{kuckling2022automode}

grammatical evolution
\cite{neupane2019learning}
\cite{kuckling2022automode}

Online Evolving BT:

\cite{jones2019onboard}
\cite{venkata2023kt}


\cite{florez2008dynamic} about dynamic BT, but haven't been applied to robots



%-------------------------------------------------------------
% P1 start summary :
% Although swarm researches developed over the past decades, researches focus on aspect.
% there lacks a systematic solution to cover overall applications in real scenario.

% P2 an ideal swarm system should be :
%   easy to deploy, change(reprogram)
%   use only local information
%   morphology
%   as a whole, sense the environment, make decision and react (autonomous).

% P3 however decentralized and centralized methods each can fulfill part of those.

% P4 This chapter gives a review of each of those


%subsection : Hybrid, hierarchical
% As centralized - decentralized each has its own pros and cons
% there are researches try to engage hybrid or hierarchical reserachs

%subsection : Task assignment in formation

%subsection : swarm autonomous

%subsection : reprogramming and human swarm interaction

%-----------------------------------------------------------


%Chapter 1 described some concrete problems and why they are important and difficult.
%In this chapter, we review and discuss literatures about them.

% a swarm should be
%As discussed in Ch1, a swarm should be
%    easy to deploy
%    use only local information
%    as a whole, sense the environment and react.

% however difficult
%However, with these constraints, many tasks are difficult to complete.

%\section{self-organization hierarchy}

%Many researches try to combine centralized and decentralized.
%They go hierarchical, with hierarchical, information can easily flow

%\section{Task assignment in formation}

%formation is researched in control theory.

%Task assignment makes efficient formation
%Task assignment calculated hungarian, network flow, n-flex algorithm

%distributed task assignment, each robot calculate a part of it, but calcualtion increases

%\section{swarm autonomous}

%what's in the builderbot paper.

%\section{reprogramming and human swarm interaction}

%what's in the builderbot paper.

%2.1 集群机器人的核心控制范式:从完全分布式到层级化混合架构

%2.1.1 纯分布式集群(Swarm)的理论基础与典型方法
%无领导集群的自组织机制(如基于局部规则的群体行为涌现)
%优势: scalability、容错性、适应动态环境的鲁棒性
%代表性研究:蚁群优化、鸟群模型(Boids)及其在机器人集群中的应用

%2.1.2 集中式控制(Centralized)的技术路径与适用场景
%全局信息感知与统一决策的实现方式(如基于中央服务器的路径规划)
%优势:任务精度高、行为可预测性强、易于编程与调试
%局限性:单点故障风险、扩展性瓶颈、对通信带宽依赖高

%2.1.3 混合架构与层级化领导模式(Hybrid, Hierarchical, Leadership)的演进
%动态层级的构建逻辑:临时领导者选举、角色分配与权限转移机制
%代表性方案:分布式领导(如基于局部优势的动态层级)与半集中式控制(如分区协调)
%核心价值:平衡分布式的灵活性与集中式的高效性

%2.2 任务场景导向的控制策略对比:集群与集中式方案的适用性分析

%2.2.1 队形控制(Formation Control)
%集群方案:基于局部距离 / 方位感知的自组织队形(如 Voronoi 图、人工势场法)
%优势:适应队形动态调整、个体故障不影响整体
%局限:精度依赖局部信息质量、大规模集群易出现累积误差
%集中式方案:全局路径规划与轨迹优化(如模型预测控制 MPC)
%优势:高精度队形保持、复杂图案生成能力强
%局限:计算负荷随规模指数增长、抗干扰能力弱

%2.2.2 探索与搜救任务(Exploration, Search and Rescue)
%集群方案:基于区域覆盖的分布式探索(如随机行走 + 信息共享)
%优势:无盲区覆盖、适应未知环境、多目标并行处理
%局限:任务协调效率低、全局信息整合滞后
%集中式方案:基于全局地图的任务分配(如 A * 算法 + 任务优先级排序)
%优势:任务规划最优性高、资源调度高效
%局限:依赖完整环境先验知识、难以应对突发障碍

%2.2.3 无人机娱乐表演(Drone Entertainment Shows)
%集群方案:基于预编程规则的分布式同步(如时间触发的轨迹对齐)
%优势:单无人机故障不中断整体表演、小规模部署灵活
%局限:复杂图案生成难度大、实时调整能力差
%集中式方案:全局动画分解与个体轨迹预生成
%优势:支持高精度复杂图案、时间同步性强
%局限:对通信延迟敏感、扩展性受限于中央计算能力

%2.2.4 其他典型任务的扩展分析
%物流协同运输:负载分配策略的集群 vs 集中式对比
%环境监测:数据采集与融合方式的效率差异

%2.3 集群编程与重编程的方法论:离线范式与核心挑战

%2.3.1 传统集群编程的离线模式
%行为规则预定义:基于个体算法的群体行为映射(如强化学习训练局部策略)
%仿真验证与参数调优:通过物理引擎模拟优化群体行为参数(如 Swarmulator 等工具)
%部署流程:统一固件更新与任务指令预装(无在线调整能力)

%2.3.2 离线编程的核心技术路径
%基于模板的行为组合:模块化规则库(如避障、聚集、跟随模块的组合调用)
%宏观行为到微观规则的转化方法:解决 “微 - 宏映射” 问题的数学建模(如控制论方法、群体动力学)

%2.3.3 离线范式的局限性隐含
%环境适应性差:预编程规则难以应对未预期场景
%扩展性瓶颈:大规模集群的参数调优成本呈指数增长

%2.4 集群自主性(Swarm Autonomy)的定义与评价维度

%2.4.1 自主性的层级划分
%基础自主性:个体故障自修复、简单环境适应(如避障)
%高级自主性:群体目标动态调整、跨任务自转换、人机协同决策

%2.4.2 现有方案的自主性边界
%纯分布式集群:高个体自主性但群体目标僵化
%集中式集群:群体目标可控但个体自主性缺失
%混合架构:在动态目标调整与自主决策方面的探索进展

%2.4.3 自主性与可控性的平衡难题
%高自主性带来的行为不可预测性风险
%人类干预与集群自主决策的接口设计挑战

%2.5 本章小结与研究定位

%现有研究的核心局限:混合架构的动态性不足、大规模任务的精度与效率难以兼顾、在线重编程能力缺失
%本研究(SoNS)的切入点:基于动态层级的混合架构如何突破上述局限,为集群自主性与可编程性提供新路径

%2.1 Core Control Paradigms in Swarm Robotics: From Fully Distributed to Hierarchical Hybrid Architectures
%2.1.1 Theoretical Foundations and Typical Methods of Purely Distributed Swarms
%Self-organization mechanisms in leaderless swarms (e.g., emergence of collective behavior based on local rules)
%Advantages: Scalability, fault tolerance, and robustness in adapting to dynamic environments
%Representative studies: Ant Colony Optimization, the Boids model, and their applications in robotic swarms​
%2.1.2 Technical Paths and Applicable Scenarios of Centralized Control​
%Implementation of global information perception and unified decision-making (e.g., path planning based on a central server)​
%Advantages: High task precision, strong behavior predictability, and ease of programming and debugging​
%Limitations: Single-point failure risk, scalability bottlenecks, and high dependence on communication bandwidth​
%2.1.3 Evolution of Hybrid Architectures and Hierarchical Leadership Models​
%Logic for constructing dynamic hierarchies: Mechanisms for temporary leader election, role assignment, and authority transfer​
%Representative solutions: Distributed leadership (e.g., dynamic hierarchies based on local superiority) and semi-centralized control (e.g., partitioned coordination)​
%Core value: Balancing the flexibility of distributed systems and the efficiency of centralized systems​
%2.2 Task-Scenario-Oriented Comparison of Control Strategies: Analysis of the Applicability of Swarm vs. Centralized Solutions​
%2.2.1 Formation Control​
%Swarm solutions: Self-organized formations based on local distance/orientation perception (e.g., Voronoi diagrams, artificial potential field method)​
%Advantages: Adaptable to dynamic formation adjustments; individual failures do not affect the whole system​
%Limitations: Precision depends on the quality of local information; cumulative errors tend to occur in large-scale swarms​
%Centralized solutions: Global path planning and trajectory optimization (e.g., Model Predictive Control, MPC)​
%Advantages: High-precision formation maintenance and strong capability for generating complex patterns​
%Limitations: Computational load grows exponentially with scale; weak anti-interference ability​
%2.2.2 Exploration and Search and Rescue Missions​
%Swarm solutions: Distributed exploration based on area coverage (e.g., random walk + information sharing)​
%Advantages: Blind-spot-free coverage, adaptability to unknown environments, and parallel processing of multiple targets​
%Limitations: Low task coordination efficiency; lag in global information integration​
%Centralized solutions: Task assignment based on global maps (e.g., A* algorithm + task priority ranking)​
%Advantages: High optimality of task planning and efficient resource scheduling​
%Limitations: Dependence on complete prior environmental knowledge; difficulty in responding to unexpected obstacles​
%2.2.3 Drone Entertainment Shows​
%Swarm solutions: Distributed synchronization based on preprogrammed rules (e.g., time-triggered trajectory alignment)​
%Advantages: Single-drone failures do not interrupt the overall performance; flexible small-scale deployment​
%Limitations: High difficulty in generating complex patterns; poor real-time adjustment capability​
%Centralized solutions: Global animation decomposition and pre-generation of individual trajectories​
%Advantages: Support for high-precision complex patterns and strong temporal synchronization​
%Limitations: Sensitivity to communication latency; scalability limited by central computing capacity​
%2.2.4 Extended Analysis of Other Typical Tasks​
%Collaborative logistics transportation: Comparison of swarm vs. centralized solutions in load distribution strategies​
%Environmental monitoring: Efficiency differences in data collection and fusion methods​
%2.3 Methodologies for Swarm Programming and Reprogramming: Offline Paradigms and Core Challenges​
%2.3.1 Offline Modes of Traditional Swarm Programming​
%Predefinition of behavioral rules: Mapping of collective behavior based on individual algorithms (e.g., reinforcement learning for training local policies)​
%Simulation verification and parameter tuning: Optimization of collective behavior parameters via physics engine simulations (e.g., tools like Swarmulator)​
%Deployment process: Unified firmware updates and pre-installation of task instructions (no online adjustment capability)​
%2.3.2 Core Technical Paths of Offline Programming​
%Template-based behavior composition: Modular rule libraries (e.g., combined invocation of obstacle avoidance, aggregation, and following modules)​
%Methods for converting macro-level behavior to micro-level rules: Mathematical modeling to address the "micro-macro mapping" problem (e.g., control theory methods, collective dynamics)​
%2.3.3 Implied Limitations of the Offline Paradigm​
%Poor environmental adaptability: Preprogrammed rules struggle to handle unanticipated scenarios​
%Scalability bottlenecks: Parameter tuning costs for large-scale swarms grow exponentially​
%2.4 Definition and Evaluation Dimensions of Swarm Autonomy​
%2.4.1 Hierarchical Classification of Autonomy​
%Basic autonomy: Individual fault self-repair, simple environmental adaptation (e.g., obstacle avoidance)​
%Advanced autonomy: Dynamic adjustment of collective goals, cross-task self-transformation, human-swarm collaborative decision-making​
%2.4.2 Autonomy Boundaries of Existing Solutions​
%Purely distributed swarms: High individual autonomy but rigid collective goals​
%Centralized swarms: Controllable collective goals but lack of individual autonomy​
%Hybrid architectures: Ongoing research progress in dynamic goal adjustment and autonomous decision-making​
%2.4.3 Challenges in Balancing Autonomy and Controllability​
%Risk of unpredictable behavior caused by high autonomy​
%Challenges in designing interfaces for human intervention and swarm autonomous decision-making​
%2.5 Chapter Summary and Research Positioning​
%Core limitations of existing research: Insufficient dynamics of hybrid architectures, difficulty in balancing precision and efficiency for large-scale tasks, and lack of online reprogramming capabilities​
%Entry point of this research (SoNS): How hybrid architectures based on dynamic hierarchies can overcome the above limitations and provide a new path for swarm autonomy and programmability