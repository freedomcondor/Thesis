t \chapter{Related Work}
\label{ch2}

% skeleton
% 1 Hybrid, hierarchical, and leadership approaches in the literature   TODO from Sci-Ro paper, yuwei paper
% 2 Swarm vs. centralized approaches for different tasks (explain what is possible to do with both, for those tasks -- current limitations are implied)
%   1.1 formation control
%   1.2 exploration, search and rescue    TODO from Sci-Ro paper
%   1.3 drone entertainment shows...
%   1.4 (more?)
% 3 The process of swarm programming / re-programming (mostly about how this is done offline, don't worry about the limitations when trying to do this online)
% 4 Swarm autonomy    TODO from Builderbot paper and zotero

\section{Introduction}

\textcolor{blue}{what is swarm, swarm is good}

Swarm intelligence often draws inspiration from collective behaviors in social animals such as ants, birds, and fish
    \cite{bonabeau1999swarm}.  % the book of Swarm intelligence
Early models like Reynolds' Boids have demonstrated that simple local behavioral rules can give rise to coherent large-scale patterns
    \cite{reynolds1987flocks}.
Following these insights, the concept of self-organization refers to the principle that individuals interact with each other locally without centralized control, and complex global behavior emerges.
Swarm Robotics aims to leverage the principles of self-organization to construct decentralized robotic systems that are scalable, adaptive, and fault tolerant
    \cite{dorigo2014swarm,    % scholarpedia on swarm robotics
          dorigo2020reflections,
          dorigo2021swarm}.    %past, present and future
These self-organized systems have shown potential in many applications, especially when large scale or space is needed, including
    environmental monitoring
    \cite{talamali2021less},
    navigation and transportation
    \cite{dorigo2013swarmanoid},
    self-assembly
    \cite{rubenstein2014programmable},
    and collective construction
    \cite{team2012designing, petersen2019review}.
Owing to the absence of centralized control, such systems can be deployed over large spatial domains and can scale up to vast numbers of robots without demanding increasing capabilities of each individual robots.

\textcolor{blue}{but, macro, micro problem, and swarm is not always efficient}

However, despite significant progress, it remains a fundamental scientific question how to design low-level individual robot behaviors to result in the desired high-level collective outcomes, known as micro-marco problem
    \cite{dorigo2021swarm}.
Despite that various theoretical frameworks have been proposed
    \cite{hamann2008framework,
          hamann2010space,
          hamann2018swarm,
          hamann2013towards}
and decentralized controllers have been demonstrated for specific tasks
    \cite{dorigo2004evolving,
          nouyan2009teamwork,
          dorigo2013swarmanoid,
          rubenstein2014programmable,
          li2019decentralized},
yet purely decentralized swarms are often difficult to design and manage, and may take long time to finish the task, especially in tasks that require global information, such as environmental analyzing and decision making.
    \cite{dorigo2021swarm,
          kengyel2015potential}. % pure flat is not good

\textcolor{blue}{Therefore, hierarchical}

These limitations motivate research into hybrid and hierarchical approaches that integrate selective forms of centralization into decentralized systems,
These systems seek to bypass the micro-macro problem to ease the design and raise the performance of a decentralized system and, in the meantime, keep the scalability and fault tolerance features.

\textcolor{blue}{In this chapter, we review}

This chapter reviews prior work in hybrid and hierarchical swarm control,
compares the existing literature on decentralized and centralized approaches,
and examines developments in swarm manageability, including fault tolerance, swarm re-programming, and swarm autonomy.

\section{Heterogeneous, Hybrid, and hierarchical Approaches}

\textcolor{blue}{flat swarms}

Flat and homogeneous swarms, as their classic nature inspiration
    \cite{buhl2006disorder,
          detrain2008collective,
          theraulaz1998origin},
assume that all agents are identical, follow the same behavioral rules, and interact with each other in the same way
    \cite{viragh2014flocking,
          vasarhelyi2018optimized,
          floreano2008bio,
          csahin2004swarm,
          beni1988concept}.
%Such systems excel in robustness and scalability but struggle with tasks that require differentiated roles, long-range coordination, or efficient global organization.

\textcolor{blue}{heterogeneous}

Opposite to that, heterogeneous swarms consist of individuals serving different roles or having different behaviors.
These systems,
with individuals in different physical types or with different information,
can often outperform purely homogeneous systems
    \cite{kengyel2015potential}. % hetero is better

Individuals with more information can influence the behavior of others during tasks such as flocking or collective decision-making
    \cite{firat2020self, % use informal individual to cue the whole swarm to aggreate to a desired shelter
          prasetyo2018best,
          balazs2020adaptive}.
\textcolor{red}{varify: prasetyo2018best}
Some works explicitly name these individuals as leaders
    \cite{gu2009leader,
          amraii2014explicit}.
%\textcolor{red}{consider drop:}
%These leaders may be vulnerable to adversarial detection, a concern explored in    \cite{zheng2020adversarial}.

\textcolor{blue}{hierarchical}

On the other hand, hierarchical approaches impose an additional topological organization on the swarm.
In these systems, hierarchy is formed by pre-defined pairs of leaders and followers,
allowing information to flow in a more structured manner
such as flocking
    \cite{dalmao2011cucker,
          jia2019modelling,
          pignotti2018flocking}
and self-assembly or formation
    \cite{li2019decentralized,
          divband2019photomorphogenesis},
%and thereby improving efficiency in tasks
and thereby ease the design of the behavior and accelerates the convergence time.
Such approaches can be regarded as highly heterogeneous systems in which each individual assumes its own role.

\textcolor{blue}{but hierarchical is not self-organized : not scalable nor fault tolerant}

However,
because the topology organization is usually pre-defined, these systems suffer from low scalability and single points of failure.
No existing approach has provided a comprehensive way to combine self-organization and hierarchical systems.
%maintaining scalability and avoiding single points of failure in these systems often require sophisticated mechanisms,
%such as temporary leadership, explicit role assignments, or distributed authority-transfer schemes.
Constructing hierarchy in a truly self-organized manner remains a major challenge
    \cite{dorigo2020reflections}.
As task complexity increases, researchers increasingly recognize the importance of self-organized hierarchy, a capability still largely absent in classical decentralized swarms.

\section{Swarm vs. centralized approaches for different tasks}

\textcolor{blue}{The point is : }
\textcolor{blue}{1 we want to show there is a gap }
\textcolor{blue}{2 both sides have some downsides }
\textcolor{blue}{ : so we need self-organized hierarchy}

Purely decentralized swarms are highly scalable and fault tolerant but often lack efficiency, especially when global information is required.
Centralized systems, by contrast, achieve high precision and optimal coordination but lose scalability and fault tolerance.
This section reviews these trade-offs across major tasks in multi-robot systems.

\subsection{Formation}

Formation control is one of the most fundamental research topics in multi-robot and swarm systems
    \cite{liu2018survey,
          oh2015survey}. % surveys
Most decentralized formation controllers rely on relative localization, using distance sensors, ultraviolet markers, or fiducial markers
    \cite{walter2018fast, %localization of UaVs using ultraviolet 
          ulrich2022towards}.  %fast fiducial marker with full 6 dof pose estimation

\subsubsection{Fixed-Role Approaches}

Classical methods assume that each robot is preassigned a fixed target position.
During deployment,
robots move according to the positions of predefined neighbors,
making motion control decentralized
but role specification effectively centralized.
These approaches achieve stable rigid-body formations using distance-based constraints
    \cite{yang2018growing,
          mehdifar2018finite,
          stacey2015passivity,
          anderson2018rigid,
          oh2011adistance,
          oh2011bdistance}.
Bearing rigidity theory
    \cite{zhao2019bearing,
          zhao2015bearing}
further allows formation maintenance using only bearing measurements, with two designated leaders controlling scale, rotation, and translation while other robots maintain relative bearings
    \cite{zhao2015translational,
          schiano2016rigidity,
          li2020adaptive,
          li2021adaptive,
          li2020bearing,
          zhao2021finite,
          zhang2022distributed,
          zhang2023bearing}. % for water surface
Switching formations under rigid-body assumptions has also been explored
    \cite{desai1999control,
          desai2002modeling}.

Overall, these methods emphasize control-theoretic guarantees over flexibility or adaptivity, assuming fixed roles and full knowledge of the desired formation.

\subsubsection{Switching Roles}

\textcolor{blue}{Why role switching is important}

One required feature of swarm robotics is self-organized role assignment.
An efficient role assignment can reduce individual robot movements and thereby accelerate formation.
However, in large swarms or environments prone to agent failures, preassigning roles is often impractical.
Some self-assembly approaches allow robots to autonomously select vacant positions, but these methods are often inefficient due to random search
    \cite{rubenstein2014programmable,
          li2019decentralized}.

Centralized algorithms, by contrast, can compute globally optimal role assignments by leveraging full knowledge of robot and formation positions
    \cite{rm2020review,
          macalpine2015scram,
          agarwal2018simultaneous,
          ravichandar2020strata,
          akella2020assignment},
ensuring minimal total displacement or collision-free paths.

\textcolor{blue}{distributed assignment}

Distributed assignment strategies attempt to reduce centralization by dividing computational responsibilities among robots,
using consensus, market-based coordination, or distributed optimization to approximate optimal assignments
    \cite{mosteo2017optimal,
          burger2012distributed,
          chopra2017distributed,
          zavlanos2007distributed,
          alonso2016distributed,
          michael2008distributed,
          montijano2014efficient,
          wang2020shape}.
Despite these efforts, current distributed methods still struggle to scale efficiently with swarm size, as the calculation load of each individual scales up with the swarm scale.
Approaches handling faulty or interchangeable agents also remain limited
    \cite{kambayashi2018distributed}.

\textcolor{blue}{there is a gap}

These observations highlight a critical gap in terms of role assignment in swarm formation:
while centralized approaches provide optimal coordination
and decentralized swarms provide robustness and scalability,
self-organized hierarchy for role allocation remain largely underexplored.
Developing such mechanisms is essential for achieving both efficiency and resilience in large-scale, formation tasks.

\subsection{Coverage}

\textcolor{blue}{what is coverage, why it is important}

Coverage is another classic task in multi-robot systems.
The goal is to ensure that a target region is fully observed, sensed, or explored by a group of agents.
Effective coverage is essential for applications such as environmental monitoring, search and rescue, precision agriculture, and surveillance.
A wide range of strategies has been developed, and most can be classified into three main categories
    \cite{wang2011coverage,     % survey
          galceran2013survey}.  % survey

\textbf{Decentralized Sensor Dispersion}
In these approaches, robots rely solely on local information to spread out and maximize area coverage.
The robots typically use potential fields, repulsion mechanisms, or adaptive local policies to maintain appropriate separation and avoid sensing overlap.
Such methods are inherently decentralized and offer strong scalability and robustness
    \cite{santos2019decentralized,
          luo2018adaptive,
          spanogianopoulos2017fast,
          siligardi2019robust}.

\textbf{Wandering or Random Exploration}
Another family of approaches relies on random motions to gradually explore the environment.
Robots may move randomly, rely on simple reactive collision responses.
These strategies are also decentralized.
They are easy to implement and require only minimal information but often suffer from slow convergence and redundant traversal
    \cite{huang2019exploration,
          ichikawa1999characteristics,
          mcguire2019minimal}.

\textbf{Predefined Sweep Trajectory-Based Coverage}
Sweep-based methods pre-calculate systematic paths, usually S-shaped trajectories, to ensure full coverage of a known region.
These strategies often require predefined roles, global map knowledge, or explicit task allocation, which introduces centralized components but yields efficient and predictable coverage
\cite{almadhoun2019survey, avellar2015multi}.

There are also mixed approaches. For example, \cite{scherer2015autonomous} employs centralized predefined sweep trajectories to generate coverage paths and then relies on decentralized mechanisms to establish communication links for environmental information collection.

\textcolor{blue}{summarize: a gap}

To summarize, 
coverage tasks clearly show the dichotomy between decentralization and centralization.
While decentralized dispersion and exploration are robust and easy to scale, they often lack efficiency and global optimality.
Conversely, sweep-based and centrally coordinated methods achieve highly efficient coverage but depend on predefined assignments or global knowledge, limiting their adaptability in large, dynamic environments.

\subsection{Navigation Path Planning}

In recent applications, especially for drones systems, path planning is a key component of multi-robot and swarm systems,
Existing methods ranges from centralized global planners to decentralized online approaches.

\textbf{Centralized Offline Planning}
Centralized planners assume global map knowledge and compute collision-free trajectories for all robots before execution.
These methods provide high-quality or even optimal solutions but scale poorly with swarm size.
Multi-robot path optimization is usually NP-hard problem,
so researches employ heuristics and approximate solutions to generate global trajectories for robots
    \cite{nazarahari2019multi,
          thabit2018multi,
          yu2016optimal,
          kushleyev2013towards}.

\textbf{Decentralized Online Planning}
Decentralized planners operate using local sensing and real-time optimization, offering better scalability and robustness.
The EGO-Swarm family
    \cite{zhou2020ego,
          zhou2021ego,
          zhou2022swarm}
demonstrates fast, distributed trajectory generation by predicting nearby robots' short-horizon behavior and optimizing smooth, dynamically feasible paths at high frequency.
These approaches enable agile navigation in cluttered environments but lack global optimality guarantees and may struggle in tightly constrained spaces.
And similar to formation control, they do not consider task assignments.
In this term, the target position of each robot is pre-defined and fixed.

\textcolor{blue}{summarize: a gap}

To conclude, similar to formation control, although researches have been well developed to generate high-quality or even optimal solutions,
fully scalable and fault tolerable systems remains to be explored.

\subsection{Light Show}

\textcolor{blue}{centralized : not fault tolerant, not scalable}

Drone light shows are becoming popular in urban life.
It represent a fully centralized form of swarm coordination,
where global trajectories are precomputed offline and broadcast to all agents.
This guarantees precise synchronization and visually coherent formations at large scale
    \cite{waibel2017drone,
          ang2018high}.
However, such architecture relies heavily on global communication and predetermined paths.
It is vulnerable to a single point (ground station) of failure or communication failures,
lacking the scalability or adaptability typically associated with decentralized systems.

\section{Fault tolerance}

Fault tolerance is an inherent property of swarm systems for their redundancy and absence of single points of failures
    \cite{dorigo2014swarm}
    \cite{hamann2018swarm}.

\textcolor{red}{cite/explain some normal fault tolerance, that is not for centralized, predetermined}
Most classical swarm tasks, such as coverage or collective decision making, can tolerate a subset of robots failes.
Robots simplys continue to operate as normal, without knowing or caring about the failure peers \textcolor{red}{cite something here}.

\textcolor{blue}{However, IFs}

However, intermittent faults (IFs) poses a unique challenge for swarm systems.
An IF is a failure that occurs for a short period of time, and then recovers.
If makes them difficult to detect and have bad influence to the swarm
    \cite{zhou2019review}
    \cite{niu2021distributed}.
\textcolor{red}{check what Sinan says, the argument here not strong enough}

Centralized methods are highly effective for identifying IFs, since they have access to system-wide data.
    \cite{sheng2021intermittent}
    \cite{zhang2021intermittent}
    \cite{syed2016novel},
but as discussed in the previous section, centralized methods are vulnerable to single point failures and not scalable.

\textcolor{blue}{Therefore, a gap}

Overall, both centralized and de-centralized swarm system have their own unique challenges in terms of fault tolerance.

\section{Swarm Autonomy and Manageability}

\subsection{Swarm autonomy}

\textcolor{blue}{indivudal autonomy is good, but swarm autonomy is a challenge}

Despite significant advances in individual robot autonomy and artificial cognition
    \cite{cangelosi2022cognition,
          vernon2014artificial},
achieving autonomy in swarm robotics remains an open challenge.
% for swarm, good works 
Existing swarm researches primarily focus on single task such as action, perception, adaptation
    \cite{heinrich2022swarm},
but higher level cognitive capabilities such as learning or anticipation are still largely absent,
and it lacks a overall framework to unify all the attributes.

\textcolor{blue}{Collective decision is a key component for swarm autonomy}

Collective decision making, as a classical task for swarm systems for environmental analysis, poses an important role in swarm automony.
Classical models for collective decision making, such as best-of-N and best-of-two
    \cite{khaluf2019neglected, % decision making
          valentini2017best,   % best of N
          dorigo2014self,
          valentini2016collective,
          shan2020collective,
          ebert2018multi,
          prasetyo2018best,       % best of N
          prasetyo2019collective, % best of N
          valentini2015efficient}, % best of N
demonstrate how simple local rules allow groups to reach agreement without centralized control.
Subsequent work investigates convergence speed, decision accuracy, and robustness under noise or malicious information
    \cite{shan2020collective,            % bayesian
          bartashevich2019benchmarking,  % 
          shan2021discrete, %\textcolor{red}{too tech}
          strobel2018managing}. %\textcolor{red}{ref incomplete} % byzenting

\textcolor{blue}{but Collective decision too simple, swarm needs higher level of cognition}

Despite these progress, decision-making in swarms generally remains a simple reactive task for environmental analysis,
rather than maintaining a broader understanding of the mission context.
In terms of automony, many elements are missing
    \cite{khaluf2019neglected},
such as the swarm needs to detect that a new collective decision is required, or to autonomously adapt the decision-making process to conditions that were not explicitly studied beforehand.
Current frameworks provide no unified mechanisms for learning new behaviors online, anticipating future events, or autonomously restructuring task workflows.

\textcolor{blue}{current swarm is roughly at level 2}

According to the SAE J3016 standard
    \cite{sae2021automated}
defines autonomous level for robot systems.
Existing robot swarm systems would be barely classified as having collective behaviors equivalent to the individual behaviors described in {\it Level 2 (Partial automation)},
as existing robot swarm systems are able to
execute intended tasks, but a human operator is still required to continuously monitor the environment and system and deciding when to intervene to correct or stop the system.
That would be a challenge for swarm systems as will be talked about in the following section.

\subsection{Human-swarm interaction}

\textcolor{blue}{human-swarm interaction is needed}

As discussed in previous section, although robot swarms are generally capable of executing their intended tasks,
human operators still play a crucial role in supervising the system,
issuing high-level commands, and reprogramming the swarm when necessary.
However, as swarm size increases, direct communication with all the robots in the swarm becomes infeasible.
This section reviews existing research on human–swarm interaction and swarm online programming.

Current researches mainly focus on two sides : human interfaces and swarm influence.
    \cite{siean2023opportunities,
          kolling2015human}.

\textcolor{blue}{interface side: gesture control and etc}

A class of researches explore interfaces to manipulate the swarm such as gesture control
    \cite{alonso2015gesture,
          podevijn2013gesturing},
haptic devices
    \cite{lee2013semiautonomous},
or wearable systems
    \cite{jarvis2025first}.
Among these works, by default, operators are able to communicate with all the robots in the swarm to send commands collectively
    \cite{ayanian2014controlling,
          macchini2021personalized,
          abioye2025user}
\textcolor{red}{varify},
which will face scalability problem when the swarm scales up.

\textcolor{blue}{influence side: decentralized, but not efficient}

For decentralized methods where human communicate with only one robot or subset of the swarm, researches forcus on how to make the subset of the swarm influence the whole swarm
    \cite{podevijn2013gesturing, % gesture, subset of the swarm
          zhou2016assistive,  % joysticks
          lee2013semiautonomous}. % haptic feedback
While these methods improve scalability,
they often suffer from slow convergence or limited expressiveness.

\cite{kolling2013human} \textcolor{red}{verify}

\textcolor{blue}{so, a gap}

In general, there lacks a framework that can issue efficient commands to swarm robots without sacrificing scalability.

\subsection{programming / re-programming}

\textcolor{blue}{People needs to easily (and online) program the swarm, but with challenges}

When supervising the swarm, the programmer needs to be able to re-program the swarm easily and online.
However, it faces two problems.
Firstly, designing controllers for swarm systems is often a long trial-and-error process
    \cite{hamann2018swarm,
          brambilla2013swarm}.  
It becomes more challenging when swarms must adapt to new tasks or environments.
Secondly, when the swarm scales up, it becomes difficult for the programmer to communicate with the swarm and deliver the new program.

This section reviews approaches to offline, centralized online, and decentralized online reprogramming.

\textcolor{blue}{Offline Programming}

Classically, swarms are deployed in offline way.
Programs are pre-designed and pre-downloaded to all the robots before deployment, for example,
    \cite{rubenstein2014programmable,
          valentini2016collective,
          werfel2014designing,
          dorigo2013swarmanoid}.
However, offline methods cannot react to unexpected task changes during deployment.
Noteably, automatic design methods falls in this category.
    \cite{francesca2014automode,
          francesca2016automatic,
          birattari2019automatic,
          salman2024automatic}.
They demonstrate how optimization and evolutionary algorithms can produce robust swarm behaviors without manual tuning.
Yet the programs are pre-generated.

\textcolor{blue}{online but centralized}

Centralized online reprogramming techniques update robot behaviors after deployment through a global communication channel.
Over-the-air programming frameworks
    \cite{zyrianoff2024over,
          abadie2024robotap}
enable operators to push new software to all robots.
Large-scale drone light shows
    \cite{waibel2017drone,
          ang2018high}
represent a widely deployed example of centralized online control.
Although they can coordinate thousands of robots,
they typically allow only predefined trajectories and low-level commands.

\textcolor{blue}{online, decentralized}

Decentralized reprogramming methods update robot controllers without relying on a central broadcaster.
Early work in wireless sensor networks
    \cite{xie2011design,
          wang2006reprogramming}
shows the feasibility of distributed software updates.
However, they also highlights fundamental limitations such as slow propagation and high energy consumption resulting from consensus mechanisms
    \cite{de2009energy,
          varadharajan2018over}.

More recent work explores knowledge transfer in multi-robot systems.
For instance, Venkata et al.\ \cite{venkata2023kt} introduce the KT-BT framework, which uses behavior trees and a grammar-based encoding (stringBT) to achieve decentralized transfer of acquired skills, improving collective performance during online adaptation.

\textcolor{blue}{summarize}

To summarize, a unified framework where a human operator can reprogram a swarm in fully scalable and efficient way remains to be explored.

\subsubsection{behavior tree}

Behavior trees (BTs) provide a modular, hierarchical framework for representing robot behaviors
    \cite{colledanchise2018behavior,
          iovino2022survey}
and have been adopted in several multi-robot applications
    \cite{colledanchise2016advantages,
          jeong2022behavior}.
Reprogramming tasks exploit BTs compositionality and readability.

Evolving BTs through evolutionary algorithms—such as genetic programming
    \cite{jones2018evolving,
          kuckling2022automode}
or grammatical evolution
    \cite{neupane2019learning,
          kuckling2022automode}—
enables automated creation of swarm behaviors.
Recent advances support online evolution as well
    \cite{jones2019onboard,
          venkata2023kt},
allowing robots to adapt their BT controllers during deployment.

Dynamic BTs have also been investigated in game AI
    \cite{florez2008dynamic},
though their application to real-world robotics remains to be explored.


\iffalse
\section{Abandon}

\textcolor{blue}{from Ayros paper}

multi-robot fusion problems are well understood, and existing methods are powerful
centralized
\textcolor{red}{About collective sensing : make it its section in applications}
These are all surveys
\cite{yan2013survey} 
\cite{sun2017multi}
\cite{rizk2019cooperative}
\cite{li2021multi}
\textcolor{red}{consider drop this sensor fusion part}


another scenario, about the quality of the block
\cite{khaluf2017edge}  find edge of an area
\cite{wahby2019collective} aggreate to the bright area
\cite{khaluf2020construction} about construction, fusion sensing the density of the building
\cite{capitan2013decentralized} 

not fully de-centralized
\cite{mirzaei2007performance} fixed sensor  \textcolor{red}{a bit tech}
\cite{rodrigues2015beyond} 
\cite{stroupe2001distributed}
\cite{zadorozhny2013information}
\cite{sasaoka2016multi}
\cite{czarnetzki2010handling}
\cite{otte2016collective} This is one a bit different, swarm robot nerval network
\cite{kornienko2005cognitive}
\cite{giusti2012cooperative}
\fi
