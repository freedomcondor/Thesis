\chapter{Related Work}
\label{ch2}

% 1 Hybrid, hierarchical, and leadership approaches in the literature   TODO from Sci-Ro paper, yuwei paper
% 2 Swarm vs. centralized approaches for different tasks (explain what is possible to do with both, for those tasks -- current limitations are implied)
%   1.1 formation control
%   1.2 exploration, search and rescue    TODO from Sci-Ro paper
%   1.3 drone entertainment shows...
%   1.4 (more?)
% 3 The process of swarm programming / re-programming (mostly about how this is done offline, don't worry about the limitations when trying to do this online)
% 4 Swarm autonomy    TODO from Builderbot paper and zotero

\textcolor{blue}{what is swarm}

Swarm robotics is inspired by social animals such as ants, birds, and fish
    \cite{bonabeau1999swarm} % the book of Swarm intelligence
.
Early models such as Boids demonstrated that simple local rules can generate complex collective behaviors
    \cite{reynolds1987flocks}.
Swarm robotics follows this principle, aiming to create decentralized systems capable of scalability, adaptability, and fault-tolerance.
    \cite{dorigo2014swarm} % scholarpedia on swarm robotics
    \cite{dorigo2020reflections}
    \cite{dorigo2021swarm} %past, present and future
Such decentralized systems have shown advantages in applications that requires large scale and space such as
    Environmental monitoring \cite{talamali2021less} \textcolor{red}{varify}, 
    navigation and transport \cite{dorigo2013swarmanoid}, 
    self-assembly \cite{rubenstein2014programmable}, 
    and construction \cite{team2012designing} \cite{petersen2019review}
.


\textcolor{blue}{but swarm is not always efficient}

Despite of great potentials, it is still challenging to connect low level individual reaction and high level collective behavior.
Although researches have been trying to create theoretical framwork
    \cite{hamann2008framework}
    \cite{hamann2010space}
    \cite{hamann2018swarm}
    \cite{hamann2013towards}
and have shown that decentralized methods can direct robots swarm to finish certain tasks
    \cite{dorigo2004evolving}
    \cite{nouyan2009teamwork}
    \cite{dorigo2013swarmanoid}
    \cite{rubenstein2014programmable}
    \cite{li2019decentralized}
,
yet inefficiencies have also been exposed due to the lack of global information held by each individual robot
\textcolor{red}{re-consider these citations, check what Aryo says, maybe use one of Marco's review:}
%for Evaluation:
    \cite{eberhardinger2018approach}
    \cite{eberhardinger2018measuring}
    \cite{kaddoum2010criteria}
%tradeoffs
    \cite{nedic2018network} % Network topology and communication-computation tradeoffs in decentralized optimization
    \cite{jovanovic2016controller} % Tradeoffs ...

\textcolor{blue}{Therefore, hierarchical}

These limitations motivate the use of hierarchical and hybrid strategies.
\textcolor{red}{argue more here : people realize it is not enough for pure, flat, self-organized systems. We still need some centralization}
    \cite{kengyel2015potential} % pure flat is not good


\section{Hybrid, hierarchical, and leadership approaches in the literature}

\textcolor{blue}{flat, homogeneous}

In pure flat and homogeneous swarms, like its nature inspiraion
    \cite{buhl2006disorder}
    \cite{detrain2008collective}
    \cite{theraulaz1998origin}
,
all the individuals look the same and act the same
    \cite{viragh2014flocking}
    \cite{vasarhelyi2018optimized}
    \cite{floreano2008bio}
    \cite{csahin2004swarm}
    \cite{beni1988concept}
.

\textcolor{blue}{heterogeneous}

On the other hand, in many cases, heterogeneous swarms performs better
    \cite{kengyel2015potential} % hetero is better
.
Some individual with more information may improve the swarm behavior.
A approach is to include informed or persistent agents to influence other individuals when flocking
    \cite{firat2020self} % use informal individual to cue the whole swarm to aggreate to a desired shelter
    \cite{valentini2016collective} \textcolor{red}{varify}
    \cite{balazs2020adaptive} \textcolor{red}{varify}
,
or explicitly call the informed individuals "leaders"
    \cite{gu2009leader}
    \cite{amraii2014explicit}
.
Further, there are researches try to hide the trace of the leaders to avoid possible attacks \textcolor{red}{consider drop}
    \cite{zheng2020adversarial}
.

\textcolor{blue}{hierarchical}

Hybrid and hierarchical approaches attempt to balance decentralized flexibility and centralized efficiency.
These approaches means to induce hierarchy structure or leadership in the swarm,
so that the swarm is good at flocking
    \cite{dalmao2011cucker,
          jia2019modelling,
          pignotti2018flocking}
and self-assembly, or formation
    \cite{li2019decentralized,
          divband2019photomorphogenesis}
.

However, to keep scalability and avoid a single point of failure,
it involves mechanisms like temporary leader election, role assignment, and authority transfer
to enable dynamic hierarchies
which is also challenging
    \cite{dorigo2020reflections} \textcolor{red}{varify}.

\textcolor{blue}{Therefore, MNSs} \textcolor{red}{prior work in our research line, not in Ch2, but in Ch1}
MNS tries to form a centralized control structure in a decentralized way.
    \cite{mathews2017mergeable}
Later work extend this idea to formation and coverage.
    \cite{zhu2020formation}
    \cite{zhang2023self}%Yuwei's paper, Self-reconfigurable hierarchical for formation control
    \cite{jamshidpey2020multi}
    \cite{jamshidpey2024centralization}
    \cite{jamshidpey2023reducing}
    \textcolor{red}{my paper ?}
    \cite{zhu2024self}

\textcolor{red}{find new home for these}
    \cite{shan2020collective} : %Distributed Bayesian Hypothesis Testing (DBHT), not sure about leader, consider drop
    \cite{kaiser2022innate} : %not sure about leader, about neuroevolution, consider drop
    \cite{zhou2022swarm}

\section{Swarm vs. centralized approaches for different tasks}

\textcolor{red}{the point is}
\textcolor{red}{1 we want to show there is a gap :  self-organized hierarchy}
\textcolor{red}{2 both sides have some downsides}

In this section, we will discuss the swarm vs. centralized approaches for classic swarm tasks like formation etc.

\subsection{Formation}

Formation control is one of the most actively studied topics in swarm robotics or multi-agent systems
    \cite{liu2018survey}% a survey
    \cite{oh2015survey}% a survey
.
A variety of formation control problems have been studied in the literature for different situation assumptions.

It relies on localization of neighbours
\cite{walter2018fast} %localization of UaVs using ultraviolet 
leD markers
\cite{ulrich2022towards} %fast fiducial marker with full 6 dof pose estimation

\subsubsection{Fixed roles}

These class usually focus in control theory.

\textcolor{blue}{They are decentralized in the sense of movement (robots move according to neighbours), but centralized in the sense of roles (predefined roles).}
These researches usually assumes fixed roles, each robot knows their target positions already.
They focus on control theory in velocity control or acceleration control to converge to the desired formation.

\cite{kushleyev2013towards} micro drones, \textcolor{red}{goes to path planning}.

They talk about how to form a rigid body with each robot detects the distance or bearing of the predefined neighbours
    \cite{yang2018growing}
    \cite{mehdifar2018finite} 
    \cite{stacey2015passivity}.

\textcolor{red}{distance rigidity}
\textcolor{blue}{bearing rigidity}
Bearing rigidity is a way to form a rigid body with only sense of bearing
    \cite{zhao2019bearing}.
In bearing rigidity, there are two leaders, other ones try to keep a certain bearing with neighbours, and the two leaders control the scale, rotation and translation of the formation.

Mathematical researches have developed theories on bearing rigidity from 2D to higher dimensions
    \cite{zhao2015bearing}.
Based on that, many researches focus on control theory methods 
    \cite{zhao2015translational}
    \cite{schiano2016rigidity} bearing formation controller
    \cite{li2020adaptive}
    \cite{li2021adaptive}
    \cite{li2020bearing}
    \cite{zhao2021finite}
    \cite{zhang2022distributed}
    \cite{zhang2023bearing} for water surface

    \textcolor{red}{still to read}
    \cite{arrigoni2018bearing}
    \cite{trinh2018bearing}
    \cite{trinh2021finite}

    \cite{tay1985generating} \textcolor{red}{too many, consider drop}
    \cite{eren2012formation} 
    \cite{trinh2019minimal}
    \cite{karimian2017theory}
    \cite{hou2016elementary}
    \cite{carboni2014rigidity} maybe not about bearing, only rigidity, \textcolor{red}{consider drop}
.

\cite{desai1999control}
\cite{desai2002modeling}
talk about switching formations.


\subsubsection{Switching Roles}

\textcolor{blue}{Why role switching is important}
As talked in Chapter 1, one feature of swarm robotics is that the individuals in the swarm can self-organize their roles.
To predefine the roles of the robots in the formation is not always a good idea, especially when the swarm is large, or we consider the some individual may fail randomly.
One should be able to scatter the swarm randomly and let them configure their position on their own.

Researches on self-assembly satisfy this assumption
    \cite{rubenstein2014programmable}
    \cite{li2019decentralized}
.

But in those approaches, robots moves around to find a vacant spot, which is low efficient.
Researches has been seeking a mathematical model for efficient role assignment.

\textcolor{blue}{centralized assignment}
Centralized assignment can make theoretical best role assignment.
One needs to knows the positions of all the robots and the positions of the whole formation.
After that, we have algorthms to find the best role assignment.
    \cite{rm2020review}
    \cite{mosteo2017optimal} \textcolor{red}{varify}
    %     presents an optimal algorithm for assigning roles and positions to homogeneous robots in freely placeable formations by jointly computing formation translation, rotation, and role assignments to minimize total displacement, with provable correctness and a distributed implementation.
    \cite{macalpine2015scram}
    % introduces SCRAM, a scalable, collision-free role assignment framework that assigns interchangeable robots to target positions to minimize makespan while avoiding collisions, with applications to large-scale multi-robot systems.
    \cite{agarwal2018simultaneous}
    % presents an algorithm that simultaneously optimizes robot assignments and variable goal formation parameters by transforming the problem into a linear sum assignment problem, enabling collision-free trajectories with O(n³) computational complexity.
    \cite{ravichandar2020strata}
    % introduces STRATA, a unified framework for task assignment in large heterogeneous agent teams that accounts for continuous traits, species- and agent-level variability, and ensures tasks' trait requirements are met efficiently.
    \cite{akella2020assignment}
    % presents algorithms for optimally assigning interchangeable robots to variable goal formations by partitioning the formation parameter space into cost-invariant equivalence classes and efficiently solving linear bottleneck assignment problems for each class. (\textcolor{red}{centralized best time assignment})
.

\textcolor{blue}{distributed assignment}
There are also distributed assignment approaches.
In these approaches, each robot carries part of the calculation for the whole assignment algorithm (solving some columns of the matrix).
    \cite{burger2012distributed}
    % introduces a distributed simplex algorithm for solving degenerate linear programs and multi-agent assignment problems over asynchronous peer-to-peer networks, achieving consensus on optimal solutions with linear scalability in communication graph diameter.
    \cite{chopra2017distributed}
    % presents a distributed version of the Hungarian method that enables teams of robots to cooperatively solve assignment and routing problems with spatiotemporal constraints through local computations and communications over a peer-to-peer network.
    \cite{zavlanos2007distributed}
    % presents a distributed formation control framework that combines consensus algorithms, market-based coordination, and artificial potential fields to achieve permutation-invariant formations, ensuring convergence and scalability using only local agent information.
    \cite{alonso2016distributed}
    % proposes a distributed formation control approach that combines consensus and sequential convex optimization to enable teams of robots to navigate and reconfigure safely among static and dynamic obstacles in 2D and 3D environments.
    \cite{michael2008distributed}
    % introduces a distributed, market-based algorithm for dynamic multi-robot task assignment that guarantees efficient convergence to desired task distributions while enabling coordinated formation control, splitting, and merging behaviors.
    \cite{montijano2014efficient}
    % presents a distributed optimization framework for multi-robot formation control that simultaneously solves for optimal positions and role assignments, using averaging and a modified distributed simplex algorithm to guarantee globally optimal configurations.
    \cite{wang2020shape}
    % presents a distributed algorithm for homogeneous robot swarms that concurrently assigns goal locations and plans collision-free paths, enabling fast and reliable shape formation for up to thousands of robots.
.

None of them are scalable : calcualtion grows as the swarm scales up.

\cite{kambayashi2018distributed} follows a unique path to use virtual agents has potential for interchangeable, but can't handle faulty robots.

\subsection{Search and Rescue}

Search and rescue involves in coveraging an area.

Three manners:
    \cite{wang2011coverage} % a survey
    \cite{galceran2013survey} % survey
Make the sensors scatter and cover the area.
They happen in a decentralized way.
    \cite{santos2019decentralized}
    \cite{luo2018adaptive}
    \cite{spanogianopoulos2017fast}
    \cite{siligardi2019robust}
wandering exploration
    \cite{huang2019exploration} \textcolor{red}{varify these three}
    \cite{ichikawa1999characteristics}
    \cite{mcguire2019minimal}

The other is to sweep an area, this usually pre-defines the paths of each robot.
\textcolor{blue}{sweep to cover, S shape path}
\cite{almadhoun2019survey} % a survey, talked about sweep coverage
\cite{avellar2015multi}
This is a offline centralized way for path planning.
\textcolor{blue}{not really about coverage, only from A to B through obstacles}
    \cite{nazarahari2019multi}
    \cite{thabit2018multi}
    \cite{yu2016optimal}

\cite{scherer2015autonomous} is a mix, centralized for path planning, after target, decentralized for connection establishment.

\subsection{Light Show}
\cite{waibel2017drone} light show
\cite{ang2018high} light show

\section{Fault tolerance}

This is a feature

\section{Manageability} \textcolor{red}{adjust this word}

This is a feature

Human needs to supervise a swarm : interact, steer, reprogram.

However the difficulty is, when scale up, human operator can't communicate with all the robot.

\subsection{Human-swarm interaction}

survey:
\cite{siean2023opportunities}
\cite{kolling2015human}

Human interface side
\cite{jarvis2025first}
\cite{alonso2015gesture} 
\cite{abioye2025user} 100 humans managing drones, not steering

human to the whole swarm
\cite{ayanian2014controlling}
\cite{alonso2015gesture}
\cite{macchini2021personalized}

\cite{podevijn2013gesturing} gesture, subset of the swarm
\cite{zhou2016assistive} joysticks
\cite{lee2013semiautonomous} haptic feedback

\cite{kolling2013human}

\subsection{programming / re-programming}

\textcolor{red}{consider a better place for automatic design, go to swarm program section}
    other challenges include automation design:
    \cite{francesca2016automatic}
    \cite{birattari2019automatic}
    \cite{salman2024automatic}

\subsection {offline examples}

mostly offline :

usually, long trial and error: 
\cite{hamann2018swarm} a book
\cite{brambilla2013swarm} a review

automatic design:
 offline automatic design
\cite{francesca2014automode} AutoMoDe
\cite{francesca2016automatic} a survey
\cite{birattari2019automatic} a survey

examples of pre-design
\cite{rubenstein2014programmable} self-assembly
\cite{valentini2016collective} decision making
\cite{werfel2014designing} construction
\cite{dorigo2013swarmanoid} multi task mission

\subsection {online but centralized}
\cite{zyrianoff2024over} over the air
\cite{abadie2024robotap}

\subsection {light show: too big, only simple commands}
\cite{waibel2017drone} light show
\cite{ang2018high} light show

\subsection {online, decentralized}
\cite{xie2011design} wireless sensor network
\cite{wang2006reprogramming} wireless sensor network a survey
but slow : consensus
\cite{de2009energy}
\cite{varadharajan2018over} 

\cite{venkata2023kt} introduces the KT-BT framework, using behavior trees and stringBT grammar, enabling multirobot knowledge transfer, with simulations showing improved group performance.

\section{Swarm Autonomy}

From builderbot paper :
What Is Cognitive Robotics?
\cite{cangelosi2022cognition} 
\cite{vernon2014artificial} 
\cite{heinrich2022swarm}
\cite{khaluf2019neglected}

level standard :
\cite{sae2021automated}

collective decision
\cite{valentini2017best} best of N
\cite{dorigo2014self}
\cite{valentini2016collective}

\textcolor{blue}{from Ayros paper}

\cite{valentini2015efficient} \textcolor{red}{incomplete reference, from Aryo paper 49, belongs to decision making, doesn't belong here}
% 100 kilobots decision making

multi-robot fusion problems are well understood, and existing methods are powerful
centralized
\textcolor{red}{About collective sensing : make it its section in applications}
These are all surveys
\cite{yan2013survey} 
\cite{sun2017multi}
\cite{rizk2019cooperative}
\cite{li2021multi}
\textcolor{red}{consider drop this fusion part}

check colors, many about decision making
\cite{strobel2018managing} \textcolor{red}{ref incomplete}
\cite{ebert2018multi} \textcolor{red}{decision making}
\cite{shan2020collective}
\cite{bartashevich2019benchmarking}
\cite{shan2021discrete} \textcolor{red}{too tech}

another scenario, about the quality of the block
\cite{prasetyo2018best}   best of N
\cite{prasetyo2019collective}  best of N
\cite{khaluf2017edge}  find edge of an area
\cite{wahby2019collective} aggreate to the bright area
\cite{khaluf2020construction} about construction, fusion sensing the density of the building
\cite{capitan2013decentralized} 

not fully de-centralized
\cite{mirzaei2007performance} fixed sensor  \textcolor{red}{a bit tech}
\cite{rodrigues2015beyond} 
\cite{stroupe2001distributed}
\cite{zadorozhny2013information}
\cite{sasaoka2016multi}
\cite{czarnetzki2010handling}
\cite{otte2016collective} This is one a bit different, swarm robot nerval network
\cite{kornienko2005cognitive}
\cite{giusti2012cooperative}

\subsection{behavior tree}

Behavior tree general
\cite{colledanchise2018behavior}
\cite{iovino2022survey}

BT for multi-robot system
\cite{colledanchise2016advantages}
\cite{jeong2022behavior}

Evolving BT:

genetic programming
\cite{jones2018evolving} \textcolor{red}{evolving}
\cite{kuckling2022automode}

grammatical evolution
\cite{neupane2019learning}
\cite{kuckling2022automode}

Online Evolving BT:

\cite{jones2019onboard}
\cite{venkata2023kt}


\cite{florez2008dynamic} about dynamic BT, but haven't been applied to robots, for games



%-------------------------------------------------------------
% P1 start summary :
% Although swarm researches developed over the past decades, researches focus on aspect.
% there lacks a systematic solution to cover overall applications in real scenario.

% P2 an ideal swarm system should be :
%   easy to deploy, change(reprogram)
%   use only local information
%   morphology
%   as a whole, sense the environment, make decision and react (autonomous).

% P3 however decentralized and centralized methods each can fulfill part of those.

% P4 This chapter gives a review of each of those


%subsection : Hybrid, hierarchical
% As centralized - decentralized each has its own pros and cons
% there are researches try to engage hybrid or hierarchical reserachs

%subsection : Task assignment in formation

%subsection : swarm autonomous

%subsection : reprogramming and human swarm interaction

%-----------------------------------------------------------


%Chapter 1 described some concrete problems and why they are important and difficult.
%In this chapter, we review and discuss literatures about them.

% a swarm should be
%As discussed in Ch1, a swarm should be
%    easy to deploy
%    use only local information
%    as a whole, sense the environment and react.

% however difficult
%However, with these constraints, many tasks are difficult to complete.

%\section{self-organization hierarchy}

%Many researches try to combine centralized and decentralized.
%They go hierarchical, with hierarchical, information can easily flow

%\section{Task assignment in formation}

%formation is researched in control theory.

%Task assignment makes efficient formation
%Task assignment calculated hungarian, network flow, n-flex algorithm

%distributed task assignment, each robot calculate a part of it, but calcualtion increases

%\section{swarm autonomous}

%what's in the builderbot paper.

%\section{reprogramming and human swarm interaction}

%what's in the builderbot paper.

%2.1 集群机器人的核心控制范式:从完全分布式到层级化混合架构

%2.1.1 纯分布式集群(Swarm)的理论基础与典型方法
%无领导集群的自组织机制(如基于局部规则的群体行为涌现)
%优势: scalability、容错性、适应动态环境的鲁棒性
%代表性研究:蚁群优化、鸟群模型(Boids)及其在机器人集群中的应用

%2.1.2 集中式控制(Centralized)的技术路径与适用场景
%全局信息感知与统一决策的实现方式(如基于中央服务器的路径规划)
%优势:任务精度高、行为可预测性强、易于编程与调试
%局限性:单点故障风险、扩展性瓶颈、对通信带宽依赖高

%2.1.3 混合架构与层级化领导模式(Hybrid, Hierarchical, Leadership)的演进
%动态层级的构建逻辑:临时领导者选举、角色分配与权限转移机制
%代表性方案:分布式领导(如基于局部优势的动态层级)与半集中式控制(如分区协调)
%核心价值:平衡分布式的灵活性与集中式的高效性

%2.2 任务场景导向的控制策略对比:集群与集中式方案的适用性分析

%2.2.1 队形控制(Formation Control)
%集群方案:基于局部距离 / 方位感知的自组织队形(如 Voronoi 图、人工势场法)
%优势:适应队形动态调整、个体故障不影响整体
%局限:精度依赖局部信息质量、大规模集群易出现累积误差
%集中式方案:全局路径规划与轨迹优化(如模型预测控制 MPC)
%优势:高精度队形保持、复杂图案生成能力强
%局限:计算负荷随规模指数增长、抗干扰能力弱

%2.2.2 探索与搜救任务(Exploration, Search and Rescue)
%集群方案:基于区域覆盖的分布式探索(如随机行走 + 信息共享)
%优势:无盲区覆盖、适应未知环境、多目标并行处理
%局限:任务协调效率低、全局信息整合滞后
%集中式方案:基于全局地图的任务分配(如 A * 算法 + 任务优先级排序)
%优势:任务规划最优性高、资源调度高效
%局限:依赖完整环境先验知识、难以应对突发障碍

%2.2.3 无人机娱乐表演(Drone Entertainment Shows)
%集群方案:基于预编程规则的分布式同步(如时间触发的轨迹对齐)
%优势:单无人机故障不中断整体表演、小规模部署灵活
%局限:复杂图案生成难度大、实时调整能力差
%集中式方案:全局动画分解与个体轨迹预生成
%优势:支持高精度复杂图案、时间同步性强
%局限:对通信延迟敏感、扩展性受限于中央计算能力

%2.2.4 其他典型任务的扩展分析
%物流协同运输:负载分配策略的集群 vs 集中式对比
%环境监测:数据采集与融合方式的效率差异

%2.3 集群编程与重编程的方法论:离线范式与核心挑战

%2.3.1 传统集群编程的离线模式
%行为规则预定义:基于个体算法的群体行为映射(如强化学习训练局部策略)
%仿真验证与参数调优:通过物理引擎模拟优化群体行为参数(如 Swarmulator 等工具)
%部署流程:统一固件更新与任务指令预装(无在线调整能力)

%2.3.2 离线编程的核心技术路径
%基于模板的行为组合:模块化规则库(如避障、聚集、跟随模块的组合调用)
%宏观行为到微观规则的转化方法:解决 “微 - 宏映射” 问题的数学建模(如控制论方法、群体动力学)

%2.3.3 离线范式的局限性隐含
%环境适应性差:预编程规则难以应对未预期场景
%扩展性瓶颈:大规模集群的参数调优成本呈指数增长

%2.4 集群自主性(Swarm Autonomy)的定义与评价维度

%2.4.1 自主性的层级划分
%基础自主性:个体故障自修复、简单环境适应(如避障)
%高级自主性:群体目标动态调整、跨任务自转换、人机协同决策

%2.4.2 现有方案的自主性边界
%纯分布式集群:高个体自主性但群体目标僵化
%集中式集群:群体目标可控但个体自主性缺失
%混合架构:在动态目标调整与自主决策方面的探索进展

%2.4.3 自主性与可控性的平衡难题
%高自主性带来的行为不可预测性风险
%人类干预与集群自主决策的接口设计挑战

%2.5 本章小结与研究定位

%现有研究的核心局限:混合架构的动态性不足、大规模任务的精度与效率难以兼顾、在线重编程能力缺失
%本研究(SoNS)的切入点:基于动态层级的混合架构如何突破上述局限,为集群自主性与可编程性提供新路径

%2.1 Core Control Paradigms in Swarm Robotics: From Fully Distributed to Hierarchical Hybrid Architectures
%2.1.1 Theoretical Foundations and Typical Methods of Purely Distributed Swarms
%Self-organization mechanisms in leaderless swarms (e.g., emergence of collective behavior based on local rules)
%Advantages: Scalability, fault tolerance, and robustness in adapting to dynamic environments
%Representative studies: Ant Colony Optimization, the Boids model, and their applications in robotic swarms​
%2.1.2 Technical Paths and Applicable Scenarios of Centralized Control​
%Implementation of global information perception and unified decision-making (e.g., path planning based on a central server)​
%Advantages: High task precision, strong behavior predictability, and ease of programming and debugging​
%Limitations: Single-point failure risk, scalability bottlenecks, and high dependence on communication bandwidth​
%2.1.3 Evolution of Hybrid Architectures and Hierarchical Leadership Models​
%Logic for constructing dynamic hierarchies: Mechanisms for temporary leader election, role assignment, and authority transfer​
%Representative solutions: Distributed leadership (e.g., dynamic hierarchies based on local superiority) and semi-centralized control (e.g., partitioned coordination)​
%Core value: Balancing the flexibility of distributed systems and the efficiency of centralized systems​
%2.2 Task-Scenario-Oriented Comparison of Control Strategies: Analysis of the Applicability of Swarm vs. Centralized Solutions​
%2.2.1 Formation Control​
%Swarm solutions: Self-organized formations based on local distance/orientation perception (e.g., Voronoi diagrams, artificial potential field method)​
%Advantages: Adaptable to dynamic formation adjustments; individual failures do not affect the whole system​
%Limitations: Precision depends on the quality of local information; cumulative errors tend to occur in large-scale swarms​
%Centralized solutions: Global path planning and trajectory optimization (e.g., Model Predictive Control, MPC)​
%Advantages: High-precision formation maintenance and strong capability for generating complex patterns​
%Limitations: Computational load grows exponentially with scale; weak anti-interference ability​
%2.2.2 Exploration and Search and Rescue Missions​
%Swarm solutions: Distributed exploration based on area coverage (e.g., random walk + information sharing)​
%Advantages: Blind-spot-free coverage, adaptability to unknown environments, and parallel processing of multiple targets​
%Limitations: Low task coordination efficiency; lag in global information integration​
%Centralized solutions: Task assignment based on global maps (e.g., A* algorithm + task priority ranking)​
%Advantages: High optimality of task planning and efficient resource scheduling​
%Limitations: Dependence on complete prior environmental knowledge; difficulty in responding to unexpected obstacles​
%2.2.3 Drone Entertainment Shows​
%Swarm solutions: Distributed synchronization based on preprogrammed rules (e.g., time-triggered trajectory alignment)​
%Advantages: Single-drone failures do not interrupt the overall performance; flexible small-scale deployment​
%Limitations: High difficulty in generating complex patterns; poor real-time adjustment capability​
%Centralized solutions: Global animation decomposition and pre-generation of individual trajectories​
%Advantages: Support for high-precision complex patterns and strong temporal synchronization​
%Limitations: Sensitivity to communication latency; scalability limited by central computing capacity​
%2.2.4 Extended Analysis of Other Typical Tasks​
%Collaborative logistics transportation: Comparison of swarm vs. centralized solutions in load distribution strategies​
%Environmental monitoring: Efficiency differences in data collection and fusion methods​
%2.3 Methodologies for Swarm Programming and Reprogramming: Offline Paradigms and Core Challenges​
%2.3.1 Offline Modes of Traditional Swarm Programming​
%Predefinition of behavioral rules: Mapping of collective behavior based on individual algorithms (e.g., reinforcement learning for training local policies)​
%Simulation verification and parameter tuning: Optimization of collective behavior parameters via physics engine simulations (e.g., tools like Swarmulator)​
%Deployment process: Unified firmware updates and pre-installation of task instructions (no online adjustment capability)​
%2.3.2 Core Technical Paths of Offline Programming​
%Template-based behavior composition: Modular rule libraries (e.g., combined invocation of obstacle avoidance, aggregation, and following modules)​
%Methods for converting macro-level behavior to micro-level rules: Mathematical modeling to address the "micro-macro mapping" problem (e.g., control theory methods, collective dynamics)​
%2.3.3 Implied Limitations of the Offline Paradigm​
%Poor environmental adaptability: Preprogrammed rules struggle to handle unanticipated scenarios​
%Scalability bottlenecks: Parameter tuning costs for large-scale swarms grow exponentially​
%2.4 Definition and Evaluation Dimensions of Swarm Autonomy​
%2.4.1 Hierarchical Classification of Autonomy​
%Basic autonomy: Individual fault self-repair, simple environmental adaptation (e.g., obstacle avoidance)​
%Advanced autonomy: Dynamic adjustment of collective goals, cross-task self-transformation, human-swarm collaborative decision-making​
%2.4.2 Autonomy Boundaries of Existing Solutions​
%Purely distributed swarms: High individual autonomy but rigid collective goals​
%Centralized swarms: Controllable collective goals but lack of individual autonomy​
%Hybrid architectures: Ongoing research progress in dynamic goal adjustment and autonomous decision-making​
%2.4.3 Challenges in Balancing Autonomy and Controllability​
%Risk of unpredictable behavior caused by high autonomy​
%Challenges in designing interfaces for human intervention and swarm autonomous decision-making​
%2.5 Chapter Summary and Research Positioning​
%Core limitations of existing research: Insufficient dynamics of hybrid architectures, difficulty in balancing precision and efficiency for large-scale tasks, and lack of online reprogramming capabilities​
%Entry point of this research (SoNS): How hybrid architectures based on dynamic hierarchies can overcome the above limitations and provide a new path for swarm autonomy and programmability
